\documentclass[]{article}
\usepackage{lmodern}
\usepackage{amssymb,amsmath}
\usepackage{ifxetex,ifluatex}
\usepackage{fixltx2e} % provides \textsubscript
\ifnum 0\ifxetex 1\fi\ifluatex 1\fi=0 % if pdftex
  \usepackage[T1]{fontenc}
  \usepackage[utf8]{inputenc}
\else % if luatex or xelatex
  \ifxetex
    \usepackage{mathspec}
  \else
    \usepackage{fontspec}
  \fi
  \defaultfontfeatures{Ligatures=TeX,Scale=MatchLowercase}
\fi
% use upquote if available, for straight quotes in verbatim environments
\IfFileExists{upquote.sty}{\usepackage{upquote}}{}
% use microtype if available
\IfFileExists{microtype.sty}{%
\usepackage{microtype}
\UseMicrotypeSet[protrusion]{basicmath} % disable protrusion for tt fonts
}{}
\usepackage[margin=1in]{geometry}
\usepackage{hyperref}
\hypersetup{unicode=true,
            pdftitle={NEON Data Product Catalog},
            pdfborder={0 0 0},
            breaklinks=true}
\urlstyle{same}  % don't use monospace font for urls
\usepackage{graphicx,grffile}
\makeatletter
\def\maxwidth{\ifdim\Gin@nat@width>\linewidth\linewidth\else\Gin@nat@width\fi}
\def\maxheight{\ifdim\Gin@nat@height>\textheight\textheight\else\Gin@nat@height\fi}
\makeatother
% Scale images if necessary, so that they will not overflow the page
% margins by default, and it is still possible to overwrite the defaults
% using explicit options in \includegraphics[width, height, ...]{}
\setkeys{Gin}{width=\maxwidth,height=\maxheight,keepaspectratio}
\IfFileExists{parskip.sty}{%
\usepackage{parskip}
}{% else
\setlength{\parindent}{0pt}
\setlength{\parskip}{6pt plus 2pt minus 1pt}
}
\setlength{\emergencystretch}{3em}  % prevent overfull lines
\providecommand{\tightlist}{%
  \setlength{\itemsep}{0pt}\setlength{\parskip}{0pt}}
\setcounter{secnumdepth}{0}
% Redefines (sub)paragraphs to behave more like sections
\ifx\paragraph\undefined\else
\let\oldparagraph\paragraph
\renewcommand{\paragraph}[1]{\oldparagraph{#1}\mbox{}}
\fi
\ifx\subparagraph\undefined\else
\let\oldsubparagraph\subparagraph
\renewcommand{\subparagraph}[1]{\oldsubparagraph{#1}\mbox{}}
\fi

%%% Use protect on footnotes to avoid problems with footnotes in titles
\let\rmarkdownfootnote\footnote%
\def\footnote{\protect\rmarkdownfootnote}

%%% Change title format to be more compact
\usepackage{titling}

% Create subtitle command for use in maketitle
\newcommand{\subtitle}[1]{
  \posttitle{
    \begin{center}\large#1\end{center}
    }
}

\setlength{\droptitle}{-2em}

  \title{NEON Data Product Catalog}
    \pretitle{\vspace{\droptitle}\centering\huge}
  \posttitle{\par}
    \author{}
    \preauthor{}\postauthor{}
      \predate{\centering\large\emph}
  \postdate{\par}
    \date{7/16/2018}


\begin{document}
\maketitle

\newpage

.

\section{DP1.00001.001 2D wind speed and
direction}\label{dp1.00001.001-2d-wind-speed-and-direction}

\begin{center}\rule{0.5\linewidth}{\linethickness}\end{center}

\textbf{Subsystem}

Terrestrial Instrument System (TIS)

\begin{center}\rule{0.5\linewidth}{\linethickness}\end{center}

\textbf{Coverage}

2D wind speed is measured at all NEON terrestrial and aquatic sites.

\begin{center}\rule{0.5\linewidth}{\linethickness}\end{center}

\textbf{Description}

Two-dimensional wind speed and direction, available as two- and
thirty-minute aggregations of 1 Hz observations. Observations are made
by 2-D sonic anemometer sensors located at multiple heights on the tower
infrastructure and by 2-D sonic anemometer sensors located on the
aquatic meteorological station.

\begin{center}\rule{0.5\linewidth}{\linethickness}\end{center}

\textbf{Abstract}

Wind plays an important role in atmospheric and environmental sciences.
A function of differential heating of Earth's surface and subsequent
pressure gradients, horizontal and vertical winds are responsible for
advection of atmospheric pollutants, moisture, heat and momentum (Stull
1988). As such, horizontal and vertical winds will be measured
throughout the Observatory.

\begin{center}\rule{0.5\linewidth}{\linethickness}\end{center}

\textbf{Usage Notes}

Note that the final quality flag for wind direction (windDirFinalQF) in
the basic download package includes a test for calm winds, under which
the computed wind direction is unreliable (due to light wind speed)
though the data quality may still be adequate for scientific purposes.
To distinguish between data flagged for calm winds and other quality
tests, please download the expanded package. \newpage
.

\section{DP1.00002.001 Single aspirated air
temperature}\label{dp1.00002.001-single-aspirated-air-temperature}

\begin{center}\rule{0.5\linewidth}{\linethickness}\end{center}

\textbf{Subsystem}

Terrestrial Instrument System (TIS)

\begin{center}\rule{0.5\linewidth}{\linethickness}\end{center}

\textbf{Coverage}

These data are collected at all NEON terrestrial and aquatic sites.

\begin{center}\rule{0.5\linewidth}{\linethickness}\end{center}

\textbf{Description}

Air temperature, available as one- and thirty-minute averages of 1 Hz
observations. Observations are made by sensors located at multiple
heights on the tower infrastructure and by sensors located on the
aquatic meteorological station. Temperature observations are made using
platinum resistance thermometers, which are housed in a fan aspirated
shield to reduce radiative bias.

\begin{center}\rule{0.5\linewidth}{\linethickness}\end{center}

\textbf{Abstract}

Single aspirated air temperature measurements, available as one- and
thirty-minute averages of 1 Hz observations. Temperature is one of the
most fundamental physical measurements. It is a primary driving factor
for countless physical, chemical, and biological processes. The single
aspirated sensor assembly comprises an individual Platinum Resistance
Thermometer (PRT) housed within an aspirated shield.

\begin{center}\rule{0.5\linewidth}{\linethickness}\end{center}

\textbf{Usage Notes}

Due to an implementation error affecting flow rates within the aspirated
assembly, the flow quality metrics in the expanded data product have
been incorrectly generated. This affects final quality flagging of the
product. Archived (historic) data will be re-processed accordingly over
the next six months. Data from August 17, 2017 and forward will be
archived internally but not publicly released until the reprocessing
framework is in place. \newpage
.

\section{DP1.00003.001 Triple aspirated air
temperature}\label{dp1.00003.001-triple-aspirated-air-temperature}

\begin{center}\rule{0.5\linewidth}{\linethickness}\end{center}

\textbf{Subsystem}

Terrestrial Instrument System (TIS)

\begin{center}\rule{0.5\linewidth}{\linethickness}\end{center}

\textbf{Coverage}

These data are collected at all NEON terrestrial sites.

\begin{center}\rule{0.5\linewidth}{\linethickness}\end{center}

\textbf{Description}

Air temperature, available as one- and thirty-minute averages derived
from triplicate 1 Hz temperature observations. Observations are made by
sensors located at the top of the tower infrastructure. Temperature
observations are made by three platinum resistance thermometers, which
are housed together in a fan aspirated shield to reduce radiative
biases.

\begin{center}\rule{0.5\linewidth}{\linethickness}\end{center}

\textbf{Abstract}

Triple aspirated air temperature measurements, available as one- and
thirty-minute averages of 1 Hz observations. Temperature is one of the
most fundamental physical measurements. It is a primary driving factor
for countless physical, chemical, and biological processes. The triple
aspirated sensor assembly comprises three individual Platinum Resistance
Thermometers (PRTs) housed within an aspirated shield.

\begin{center}\rule{0.5\linewidth}{\linethickness}\end{center}

\textbf{Usage Notes}

Due to an implementation error affecting flow rates within the aspirated
assembly, the flow quality metrics in the expanded data product have
been incorrectly generated. This affects final quality flagging of the
product. Archived (historic) data will be re-processed accordingly over
the next six months. Data from August 17, 2017 and forward will be
archived internally but not publicly released until the reprocessing
framework is in place. \newpage
.

\section{DP1.00004.001 Barometric
pressure}\label{dp1.00004.001-barometric-pressure}

\begin{center}\rule{0.5\linewidth}{\linethickness}\end{center}

\textbf{Subsystem}

Terrestrial Instrument System (TIS)

\begin{center}\rule{0.5\linewidth}{\linethickness}\end{center}

\textbf{Coverage}

At terrestrial sites the pressure sensor will be located on the tower
infrastructure at a site specific installation height (h) above ground
level (AGL). At aquatic sites the pressure sensor will be located on a
field-based met station (tripod) at a standard installation height of
above ground level. Lake sites will have an additional pressure sensor
located on a buoy at a standard installation height above water level
(AWL), but at a different sampling frequency and that data will be
handled in a separate ATBD. Therefore, barometric (station) pressure
will represent the point in space at which the barometer is located.

\begin{center}\rule{0.5\linewidth}{\linethickness}\end{center}

\textbf{Description}

Barometric pressure is available as one- and thirty-minute averages for
station pressure, which is determined from 0.1 Hz observations.
Barometric pressure corrected to sea level and surface level (defined as
water surface at aquatic sites and soil surface at terrestrial sites) is
derived from station pressure averages and available at one- and
thirty-minute increments. Observations are made by a single digital
barometer located on the tower infrastructure and a single digital
barometer located on the aquatic meteorological station.

\begin{center}\rule{0.5\linewidth}{\linethickness}\end{center}

\textbf{Abstract}

Barometric pressure, or static atmospheric pressure, is a vital
measurement for NEON. Barometric pressure is significant in influencing
weather conditions as well as aqueous chemistry (e.g.~the amount of gas
that can dissolve in solution). Recording static atmospheric pressure
will allow atmospheric gas mixing ratios to be converted into mass
quantities. Barometric pressure will be recorded over NEON's entire
operational range. \newpage
.

\section{DP1.00005.001 IR biological
temperature}\label{dp1.00005.001-ir-biological-temperature}

\begin{center}\rule{0.5\linewidth}{\linethickness}\end{center}

\textbf{Subsystem}

Terrestrial Instrument System (TIS)

\begin{center}\rule{0.5\linewidth}{\linethickness}\end{center}

\textbf{Coverage}

These data are collected at all NEON terrestrial sites.

\begin{center}\rule{0.5\linewidth}{\linethickness}\end{center}

\textbf{Description}

Infrared temperature, available as one- and thirty-minute averages of 1
Hz observations. Biological temperature (i.e.~surface temperature) is
measured via IR temperature sensors located in the soil array and at
multiple heights on the tower infrastructure.

\begin{center}\rule{0.5\linewidth}{\linethickness}\end{center}

\textbf{Abstract}

Infrared Biological Temperature (i.e., surface temperature) is available
as one- and thirty-minute averages of 1 Hz observations. Biological
temperature can be used in conjunction with other measurements to draw
conclusions on topics such as plant respiration, evapotranspiration
rates, and stomatal conductance. \newpage
.

\section{DP1.00006.001 Precipitation}\label{dp1.00006.001-precipitation}

\begin{center}\rule{0.5\linewidth}{\linethickness}\end{center}

\textbf{Subsystem}

Terrestrial Instrument System (TIS)

\begin{center}\rule{0.5\linewidth}{\linethickness}\end{center}

\textbf{Coverage}

Some form of precipitation data are collected at all NEON terrestrial
and aquatic sites.

\begin{center}\rule{0.5\linewidth}{\linethickness}\end{center}

\textbf{Description}

Precipitation is observed using one of two sensors. Primary
precipitation is observed using a weighing gauge housed within a small
double fence intercomparison reference, which is generally located
within 0.5 km of the tower infrastructure. Secondary precipitation is
observed using a tipping bucket located on the top of the tower
infrastructure. Ground level precipitation (also known as throughfall)
is also observed using tipping buckets at 3 of 5 soil array locations.
Bulk precipitation is determined at five- and thirty-minute intervals
for primary precipitation and at one- and thirty-minute intervals for
secondary precipitation. AIS sites only include Primary (DFIR at Core
Aquatic sites; currently only 2 or 3) or Secondary (tipping buckets at
relocatable sites; currently about 8). No AIS sites have both; no AIS
sites have the equivalent of throughfall.

\begin{center}\rule{0.5\linewidth}{\linethickness}\end{center}

\textbf{Abstract}

Across NEON sites two methods will be used to determine bulk
precipitation. Bulk precipitation measurements at core sites consist of
a weighing gauge surrounded by a double fence inter-comparison reference
(DFIR). Bulk precipitation measurements at relocatable sites are made
with a tipping bucket. Bulk precipitation measured using a DFIR and a
weighing gauge is known to provide improved results over tipping bucket
measurements. Thus, the weighing gauge surrounded by the DFIR is
considered the ``primary'' method, while the tipping bucket is referred
to as the ``secondary'' method.

\begin{center}\rule{0.5\linewidth}{\linethickness}\end{center}

\textbf{Usage Notes}

NOTICE TO USERS 2018-04-04: All current and previous throughfall
precipitation data should be considered unreliable and of limited use.
The tipping mechanism frequently falls off its rocker, causing
precipitation events to go unrecorded. A more robust design of the
tipping mechanism is currently in production and all sensors will be
retrofitted. All throughfall precipitation data will be quality flagged
as suspect until the sensor retrofit is complete. \newpage
.

\section{DP1.00007.001 3D wind speed, direction and sonic
temperature}\label{dp1.00007.001-3d-wind-speed-direction-and-sonic-temperature}

\begin{center}\rule{0.5\linewidth}{\linethickness}\end{center}

\textbf{Subsystem}

Terrestrial Instrument System (TIS)

\begin{center}\rule{0.5\linewidth}{\linethickness}\end{center}

\textbf{Coverage}

These data are collected at all terrestrial sites to study the
3-dimensional wind speed and wind direction above the ecosystem canopy.

\begin{center}\rule{0.5\linewidth}{\linethickness}\end{center}

\textbf{Description}

Three-dimensional windspeed and direction measured by sonic anemometer;
air temperature measured by the 3-D sonic anemometer. This data product
is bundled into DP4.00200, Bundled data products - eddy covariance, and
is not available as a stand-alone download.

\begin{center}\rule{0.5\linewidth}{\linethickness}\end{center}

\textbf{Abstract}

This data product provides the turbulent wind speed and wind direction
statistics. The key sub-data products include the three wind components
at the tower top. It contains the quality-controlled measurement data
and associated metadata in HDF5 format. It is also used alongside other
data products to generate the momentum flux, sensible heat flux, latent
heat flux and carbon flux data products.

\begin{center}\rule{0.5\linewidth}{\linethickness}\end{center}

\textbf{Usage Notes}

During subsequent nominal operations, we plan to produce and publish the
data products in three phases, to accommodate a variety of use cases:
the initial near-real-time transition, a science reviewed quality
transition, and the epoch yearly transition. The initial near-real-time
transition is scheduled to process daily files at a 5-day delay after
data collection to accommodate a 9-day centered planar-fit window. If
the data has not been received from the field it will attempt to process
daily for 30\,days, and if not all data is available after this window a
force execution is performed populating a HDF5 file with metadata and
filling data with NaN's. The monthly file will be produced after all
daily files are available, no later than 30 days after the last daily
file was initially attempted to be processed. After the initial
transition, the NEON science team has a one month window to manually
flag data that were identified as suspect through field-based problem
tracking and resolution tickets or through additional manual data
quality analysis. Then, the science-reviewed transition will occur, and
the data will be republished to the data portal. The last transition
type is part of the yearly epoch versioning, which provides a fully
quality assured and quality controlled version of the data using the
latest full release of the processing code. This transition is scheduled
to occur 18 months after the initial data collection. \newpage
.

\section{DP1.00010.001 3D wind attitude and motion
reference}\label{dp1.00010.001-3d-wind-attitude-and-motion-reference}

\begin{center}\rule{0.5\linewidth}{\linethickness}\end{center}

\textbf{Subsystem}

Terrestrial Instrument System (TIS)

\begin{center}\rule{0.5\linewidth}{\linethickness}\end{center}

\textbf{Coverage}

These data are collected at all terrestrial sites to study the attitude
and motion of the 3D sonic anemometer at the tower top.

\begin{center}\rule{0.5\linewidth}{\linethickness}\end{center}

\textbf{Description}

Measurement of 3D anemometer attitude and motion. This data product is
bundled into DP4.00200, Bundled data products - eddy covariance, and is
not available as a stand-alone download.

\begin{center}\rule{0.5\linewidth}{\linethickness}\end{center}

\textbf{Abstract}

This data product provides the attitude statistics for the 3D sonic
anemometer. The key sub-data products include the three attitude angles
of the 3D sonic anemometer at the tower top. It contains the
quality-controlled measurement data and associated metadata in HDF5
format. It is also used alongside other data products to generate the
momentum flux, sensible heat flux, latent heat flux and carbon flux data
products. The data are delivered with the Bundled data products - eddy
covariance data product (DP4.00200.001).

\begin{center}\rule{0.5\linewidth}{\linethickness}\end{center}

\textbf{Usage Notes}

During subsequent nominal operations, we plan to produce and publish the
data products in three phases, to accommodate a variety of use cases:
the initial near-real-time transition, a science reviewed quality
transition, and the epoch yearly transition. The initial near-real-time
transition is scheduled to process daily files at a 5-day delay after
data collection to accommodate a 9-day centered planar-fit window. If
the data has not been received from the field it will attempt to process
daily for 30\,days, and if not all data is available after this window a
force execution is performed populating a HDF5 file with metadata and
filling data with NaN's. The monthly file will be produced after all
daily files are available, no later than 30 days after the last daily
file was initially attempted to be processed. After the initial
transition, the NEON science team has a one month window to manually
flag data that were identified as suspect through field-based problem
tracking and resolution tickets or through additional manual data
quality analysis. Then, the science-reviewed transition will occur, and
the data will be republished to the data portal. The last transition
type is part of the yearly epoch versioning, which provides a fully
quality assured and quality controlled version of the data using the
latest full release of the processing code. This transition is scheduled
to occur 18 months after the initial data collection. \newpage
.

\section{DP1.00013.001 Wet deposition chemical
analysis}\label{dp1.00013.001-wet-deposition-chemical-analysis}

\begin{center}\rule{0.5\linewidth}{\linethickness}\end{center}

\textbf{Subsystem}

Terrestrial Instrument System (TIS)

\begin{center}\rule{0.5\linewidth}{\linethickness}\end{center}

\textbf{Coverage}

Measured at select NEON terrestrial and aquatics sites.

\begin{center}\rule{0.5\linewidth}{\linethickness}\end{center}

\textbf{Description}

Total dissolved chemical ion concentrations of SO4 2-, NO3-, Cl-, Br-,
NH4+, PO4 3-, Ca2+, Mg2+, K+, Na+, and pH/Conductivity in precipitation
water; collected at TIS and AIS sites.

\begin{center}\rule{0.5\linewidth}{\linethickness}\end{center}

\textbf{Abstract}

This data product contains the quality-controlled, native sampling
resolution data from NEON's wet deposition protocol. Major ion
concentrations, pH, and conductivity are measured in precipitation
samples.

\begin{center}\rule{0.5\linewidth}{\linethickness}\end{center}

\textbf{Usage Notes}

Queries for this data product will return data from wdp\_collection,
wdp\_collectionChem, and wdp\_chemLab for all dates within the specified
date range. Each record in wdp\_collection is expected to have one child
record in each of wdp\_collectionChem and wdp\_chemLab. The expanded
package returns an additional table, wdp\_sensor, containing automated
data from the collector assembly for all months in the date range
requested. \newpage
.

\section{DP1.00014.001 Shortwave radiation (direct and diffuse
pyranometer)}\label{dp1.00014.001-shortwave-radiation-direct-and-diffuse-pyranometer}

\begin{center}\rule{0.5\linewidth}{\linethickness}\end{center}

\textbf{Subsystem}

Terrestrial Instrument System (TIS)

\begin{center}\rule{0.5\linewidth}{\linethickness}\end{center}

\textbf{Coverage}

These data are collected at all NEON terrestrial sites.

\begin{center}\rule{0.5\linewidth}{\linethickness}\end{center}

\textbf{Description}

Total, direct beam, and diffuse shortwave radiation, available as one-
and thirty-minute averages of 1 Hz observations. Observations are made
by a sensor located at the top of the tower infrastructure.

\begin{center}\rule{0.5\linewidth}{\linethickness}\end{center}

\textbf{Abstract}

Total, direct, and diffuse shortwave radiation, available as one- and
thirty-minute averages of 1 Hz observations. Direct radiation, also
called direct beam radiation, is the solar radiation traveling in a
straight line from the sun to a plane at the Earth's surface oriented
perpendicular to the sun's rays. Diffuse radiation is the solar
radiation scattered by particles in the atmosphere and received at a
horizontal plane at the Earth's surface. Diffuse radiation comes from
the entire sky dome, whereas direct radiation comes from a single
direction. Total solar radiation is the sum of direct and diffuse solar
radiation received at a horizontal plane at the Earth's surface.
\newpage
.

\section{DP1.00017.001 Dust and particulate size
distribution}\label{dp1.00017.001-dust-and-particulate-size-distribution}

\begin{center}\rule{0.5\linewidth}{\linethickness}\end{center}

\textbf{Subsystem}

Terrestrial Instrument System (TIS)

\begin{center}\rule{0.5\linewidth}{\linethickness}\end{center}

\textbf{Coverage}

NEON terrestrial sites

\begin{center}\rule{0.5\linewidth}{\linethickness}\end{center}

\textbf{Description}

Near real-time measurements of PM1.0, PM2.5, PM4, PM10, PM15 and TSP in
the atmosphere using a optical sensor.

\begin{center}\rule{0.5\linewidth}{\linethickness}\end{center}

\textbf{Abstract}

By deploying optical particulate matter analyzers at a total of 6 sites
across three Domains (10, 13, and 15) NEON's aim is to help the
scientific community gain insight on the regional dust transport across
the Rocky Mountain region. Aerosol dust can be composed of numerous
inorganic and organic elements; everything from biological components
such as pollen to the byproducts of incomplete combustion. \newpage
.

\section{DP1.00022.001 Shortwave radiation (primary
pyranometer)}\label{dp1.00022.001-shortwave-radiation-primary-pyranometer}

\begin{center}\rule{0.5\linewidth}{\linethickness}\end{center}

\textbf{Subsystem}

Terrestrial Instrument System (TIS)

\begin{center}\rule{0.5\linewidth}{\linethickness}\end{center}

\textbf{Coverage}

These data are collected at core NEON terrestrial sites.

\begin{center}\rule{0.5\linewidth}{\linethickness}\end{center}

\textbf{Description}

Total shortwave radiation, available as one- and thirty-minute averages
of 1 Hz observations. The primary pyranometer is housed in a heated and
aspirated ventilation unit and observes incoming shortwave radiation at
the top of the tower infrastructure.

\begin{center}\rule{0.5\linewidth}{\linethickness}\end{center}

\textbf{Abstract}

Total incoming solar shortwave radiation, available as one- and
thirty-minute averages of 1 Hz observations. Shortwave radiation is
composed of ultraviolet, visible, and a portion of infra-red
wavelengths. Total shortwave radiation, also called global radiation, is
the incendent shortwave solar radiation (direct and diffuse) received on
a horizontal plane at the Earth's surface. \newpage
.

\section{DP1.00023.001 Shortwave and longwave radiation (net
radiometer)}\label{dp1.00023.001-shortwave-and-longwave-radiation-net-radiometer}

\begin{center}\rule{0.5\linewidth}{\linethickness}\end{center}

\textbf{Subsystem}

Terrestrial Instrument System (TIS)

\begin{center}\rule{0.5\linewidth}{\linethickness}\end{center}

\textbf{Coverage}

These data are collected at all NEON aquatic and terrestrial sites.

\begin{center}\rule{0.5\linewidth}{\linethickness}\end{center}

\textbf{Description}

Net radiation is composed of incoming and outgoing shortwave and
longwave radiation. These data products are available as one- and
thirty-minute averages of 1 Hz observations. Observations of net
shortwave and longwave radiation are made by a sensor located at the top
of the tower infrastructure, while only net longwave radiation is
observed in the soil array. Observations of net shortwave and longwave
radiation are made by a four-component sensor located on the aquatic
meteorological station.

\begin{center}\rule{0.5\linewidth}{\linethickness}\end{center}

\textbf{Abstract}

The four components of net radiation, available as one- and
thirty-minute averages of 1 Hz observations. Net radiation is the
balance between incoming and outgoing shortwave and longwave radiation
on a horizontal plane at the Earth's surface. This data product provides
observations of incoming shortwave, outgoing shortwave, incoming
longwave, and outgoing longwave radiation. \newpage
.

\section{DP1.00024.001 Photosynthetically active radiation
(PAR)}\label{dp1.00024.001-photosynthetically-active-radiation-par}

\begin{center}\rule{0.5\linewidth}{\linethickness}\end{center}

\textbf{Subsystem}

Terrestrial Instrument System (TIS)

\begin{center}\rule{0.5\linewidth}{\linethickness}\end{center}

\textbf{Coverage}

All NEON terrestrial and aquatic sites.

\begin{center}\rule{0.5\linewidth}{\linethickness}\end{center}

\textbf{Description}

Photosynthetically Active Radiation (PAR) observations represent the
radiation flux at wavelengths between 400-700 nm, which constitute the
wavelengths that drive photosynthesis. This data product is available as
one- and thirty-minute averages of 1 Hz observations. Observations are
made by sensors located at multiple heights on the tower infrastructure
and by sensors located on the aquatic meteorological station.

\begin{center}\rule{0.5\linewidth}{\linethickness}\end{center}

\textbf{Abstract}

Photosynthetically Active Radiation (PAR), available as one- and
thirty-minute averages of 1 Hz observations. PAR represents the
radiation at wavelengths between 400-700 nm (visible light), which
constitutes the energy that drives photosynthesis. \newpage
.

\section{DP1.00033.001 Phenology
images}\label{dp1.00033.001-phenology-images}

\begin{center}\rule{0.5\linewidth}{\linethickness}\end{center}

\textbf{Subsystem}

Terrestrial Instrument System (TIS)

\begin{center}\rule{0.5\linewidth}{\linethickness}\end{center}

\textbf{Coverage}

All terrestrial core and relocatable sites; camera is positioned on the
top of each site's tower.

\begin{center}\rule{0.5\linewidth}{\linethickness}\end{center}

\textbf{Description}

RGB and IR images of the plant canopy taken from an automated camera on
the tower top. Images are collected every 15 minutes and closely follow
protocols of the Phenocam Network.

\begin{center}\rule{0.5\linewidth}{\linethickness}\end{center}

\textbf{Abstract}

Phenology is the study of reoccurring life cycle events that are driven
by environmental factors (Morrisette et al., 2009). The timing of these
events is driven by both short- and long-term variability in climate and
is therefore valuable in understanding the effects of climate change
(Richardson et al., 2006). Automated repeat digital images of plant
canopies provide data for the extraction of indices (e.g.~green
chromatic coordinate (gcc)) that can be used to quantify changes in
phenological events over time (Sonnentag et al., 2011).

NEON has deployed a Stardot NetCam on the top of all terrestrial core
and re-locatable towers to study above-canopy phenology. Every 15
minutes each camera captures back-to-back RGB and IR images separated by
30 seconds. Over time, these images can be used to detect seasonal
changes in vegetative canopies (e.g., onset of leaf growth and
senescence). Images are sent to and processed by PhenoCam, a cooperative
network that archives and distributes imagery and derived data products
from digital cameras deployed at research sites across North America and
around the world. NEON's phenocam images are available for viewing and
downloading from the \href{https://phenocam.sr.unh.edu}{PhenoCam
Gallery}, along with images and data from other phenocam sites across
the world. \newpage
.

\section{DP1.00034.001 CO2 concentration -
turbulent}\label{dp1.00034.001-co2-concentration---turbulent}

\begin{center}\rule{0.5\linewidth}{\linethickness}\end{center}

\textbf{Subsystem}

Terrestrial Instrument System (TIS)

\begin{center}\rule{0.5\linewidth}{\linethickness}\end{center}

\textbf{Coverage}

These data are collected at all terrestrial sites to study the turbulent
CO2 concentration above the ecosystem canopy.

\begin{center}\rule{0.5\linewidth}{\linethickness}\end{center}

\textbf{Description}

Concentration of CO2 at the top of the tower; used in calculation of
turbulent terms in eddy covariance calculations of carbon exchange. This
data product is bundled into DP4.00200, Bundled data products - eddy
covariance, and is not available as a stand-alone download.

\begin{center}\rule{0.5\linewidth}{\linethickness}\end{center}

\textbf{Abstract}

This data product provides the turbulent CO2 concentration statistics.
The key sub-data products include CO2 molar fraction in the air at the
tower top. It contains the quality-controlled measurement data and
associated metadata in HDF5 format. It is also used alongside other data
products to generate the CO2 flux data product. The data are delivered
with the Bundled data products - eddy covariance data product
(DP4.00200.001).

\begin{center}\rule{0.5\linewidth}{\linethickness}\end{center}

\textbf{Usage Notes}

During subsequent nominal operations, we plan to produce and publish the
data products in three phases, to accommodate a variety of use cases:
the initial near-real-time transition, a science reviewed quality
transition, and the epoch yearly transition. The initial near-real-time
transition is scheduled to process daily files at a 5-day delay after
data collection to accommodate a 9-day centered planar-fit window. If
the data has not been received from the field it will attempt to process
daily for 30\,days, and if not all data is available after this window a
force execution is performed populating a HDF5 file with metadata and
filling data with NaN's. The monthly file will be produced after all
daily files are available, no later than 30 days after the last daily
file was initially attempted to be processed. After the initial
transition, the NEON science team has a one month window to manually
flag data that were identified as suspect through field-based problem
tracking and resolution tickets or through additional manual data
quality analysis. Then, the science-reviewed transition will occur, and
the data will be republished to the data portal. The last transition
type is part of the yearly epoch versioning, which provides a fully
quality assured and quality controlled version of the data using the
latest full release of the processing code. This transition is scheduled
to occur 18 months after the initial data collection. \newpage
.

\section{DP1.00035.001 H2O concentration -
turbulent}\label{dp1.00035.001-h2o-concentration---turbulent}

\begin{center}\rule{0.5\linewidth}{\linethickness}\end{center}

\textbf{Subsystem}

Terrestrial Instrument System (TIS)

\begin{center}\rule{0.5\linewidth}{\linethickness}\end{center}

\textbf{Coverage}

These data are collected at all terrestrial sites to study the turbulent
H2O concentration above the ecosystem canopy.

\begin{center}\rule{0.5\linewidth}{\linethickness}\end{center}

\textbf{Description}

Concentration of H2O at the top of the tower; used in calculation of
turbulent terms in eddy covariance calculations of water vapor exchange.
This data product is bundled into DP4.00200, Bundled data products -
eddy covariance, and is not available as a stand-alone download.

\begin{center}\rule{0.5\linewidth}{\linethickness}\end{center}

\textbf{Abstract}

This data product provides the turbulent H2O concentration statistics.
The key sub-data products include H2O molar fraction in the air at the
tower top. It contains the quality-controlled measurement data and
associated metadata in HDF5 format. It is also used alongside other data
products to generate the latent heat flux data product. The data are
delivered with the Bundled data products - eddy covariance data product
(DP4.00200.001).

\begin{center}\rule{0.5\linewidth}{\linethickness}\end{center}

\textbf{Usage Notes}

During subsequent nominal operations, we plan to produce and publish the
data products in three phases, to accommodate a variety of use cases:
the initial near-real-time transition, a science reviewed quality
transition, and the epoch yearly transition. The initial near-real-time
transition is scheduled to process daily files at a 5-day delay after
data collection to accommodate a 9-day centered planar-fit window. If
the data has not been received from the field it will attempt to process
daily for 30\,days, and if not all data is available after this window a
force execution is performed populating a HDF5 file with metadata and
filling data with NaN's. The monthly file will be produced after all
daily files are available, no later than 30 days after the last daily
file was initially attempted to be processed. After the initial
transition, the NEON science team has a one month window to manually
flag data that were identified as suspect through field-based problem
tracking and resolution tickets or through additional manual data
quality analysis. Then, the science-reviewed transition will occur, and
the data will be republished to the data portal. The last transition
type is part of the yearly epoch versioning, which provides a fully
quality assured and quality controlled version of the data using the
latest full release of the processing code. This transition is scheduled
to occur 18 months after the initial data collection. \newpage
.

\section{DP1.00036.001 Atmospheric CO2
isotopes}\label{dp1.00036.001-atmospheric-co2-isotopes}

\begin{center}\rule{0.5\linewidth}{\linethickness}\end{center}

\textbf{Subsystem}

Terrestrial Instrument System (TIS)

\begin{center}\rule{0.5\linewidth}{\linethickness}\end{center}

\textbf{Coverage}

This data product is collected at all terrestrial sites. The sensor is
located inside the instrument hut near the bottom of the tower. The air
samples from different measurement heights are pumped through gas tubing
to the sensor for analysis.

\begin{center}\rule{0.5\linewidth}{\linethickness}\end{center}

\textbf{Description}

Profile measurements of CO2 isotope concentration, 13C stable isotope
ratio in CO2, and water vapor concentration at each tower level. This
data product is bundled into DP4.00200, Bundled data products - eddy
covariance, and is not available as a stand-alone download.

\begin{center}\rule{0.5\linewidth}{\linethickness}\end{center}

\textbf{Abstract}

This data product contains the quality-controlled Atmospheric CO2
isotopes measurement data and associated metadata in HDF5 format. The
key sub-data products include CO2 molar fraction, H2O molar fraction and
delta 13C in CO2 in the air at different measurement heights on tower at
all NEON terrestrial sites. The data are delivered with the Bundled data
products - eddy covariance data product (DP4.00200.001).

\begin{center}\rule{0.5\linewidth}{\linethickness}\end{center}

\textbf{Usage Notes}

During subsequent nominal operations, we plan to produce and publish the
data products in three phases, to accommodate a variety of use cases:
the initial near-real-time transition, a science reviewed quality
transition, and the epoch yearly transition. The initial near-real-time
transition is scheduled to process daily files at a 5-day delay after
data collection to accommodate a 9-day centered planar-fit window. If
the data has not been received from the field it will attempt to process
daily for 30\,days, and if not all data is available after this window a
force execution is performed populating a HDF5 file with metadata and
filling data with NaN's. The monthly file will be produced after all
daily files are available, no later than 30 days after the last daily
file was initially attempted to be processed. After the initial
transition, the NEON science team has a one month window to manually
flag data that were identified as suspect through field-based problem
tracking and resolution tickets or through additional manual data
quality analysis. Then, the science-reviewed transition will occur, and
the data will be republished to the data portal. The last transition
type is part of the yearly epoch versioning, which provides a fully
quality assured and quality controlled version of the data using the
latest full release of the processing code. This transition is scheduled
to occur 18 months after the initial data collection. \newpage
.

\section{DP1.00037.001 Atmospheric H2O
isotopes}\label{dp1.00037.001-atmospheric-h2o-isotopes}

\begin{center}\rule{0.5\linewidth}{\linethickness}\end{center}

\textbf{Subsystem}

Terrestrial Instrument System (TIS)

\begin{center}\rule{0.5\linewidth}{\linethickness}\end{center}

\textbf{Coverage}

This data product is collected at all terrestrial core sites plus 1
relocatable site (D19 BARR). The sensor is located inside the instrument
hut near the bottom of the tower. The air samples from different
measurement heights are pumped through gas tubing to the sensor for
analysis.

\begin{center}\rule{0.5\linewidth}{\linethickness}\end{center}

\textbf{Description}

Profile measurements of water vapor isotope concentration, 18O and 2H
stable isotope ratio in water vapor at each tower level. This data
product is bundled into DP4.00200, Bundled data products - eddy
covariance, and is not available as a stand-alone download.

\begin{center}\rule{0.5\linewidth}{\linethickness}\end{center}

\textbf{Abstract}

This data product contains the quality-controlled measurement data and
associated metadata in HDF5 format. The key sub-data products include
H2O molar fraction, delta 18O and delta 2H in water vapor in the air at
different measurement heights on the tower at all terrestrial core sites
and one relocatable site (D19 BARR). The data are delivered with the
Bundled data products - eddy covariance data product (DP4.00200.001).

\begin{center}\rule{0.5\linewidth}{\linethickness}\end{center}

\textbf{Usage Notes}

During subsequent nominal operations, we plan to produce and publish the
data products in three phases, to accommodate a variety of use cases:
the initial near-real-time transition, a science reviewed quality
transition, and the epoch yearly transition. The initial near-real-time
transition is scheduled to process daily files at a 5-day delay after
data collection to accommodate a 9-day centered planar-fit window. If
the data has not been received from the field it will attempt to process
daily for 30\,days, and if not all data is available after this window a
force execution is performed populating a HDF5 file with metadata and
filling data with NaN's. The monthly file will be produced after all
daily files are available, no later than 30 days after the last daily
file was initially attempted to be processed. After the initial
transition, the NEON science team has a one month window to manually
flag data that were identified as suspect through field-based problem
tracking and resolution tickets or through additional manual data
quality analysis. Then, the science-reviewed transition will occur, and
the data will be republished to the data portal. The last transition
type is part of the yearly epoch versioning, which provides a fully
quality assured and quality controlled version of the data using the
latest full release of the processing code. This transition is scheduled
to occur 18 months after the initial data collection. \newpage
.

\section{DP1.00038.001 Stable isotope concentrations in
precipitation}\label{dp1.00038.001-stable-isotope-concentrations-in-precipitation}

\begin{center}\rule{0.5\linewidth}{\linethickness}\end{center}

\textbf{Subsystem}

Terrestrial Instrument System (TIS)

\begin{center}\rule{0.5\linewidth}{\linethickness}\end{center}

\textbf{Coverage}

Measured at select NEON terrestrial and aquatic sites.

\begin{center}\rule{0.5\linewidth}{\linethickness}\end{center}

\textbf{Description}

Stable isotope ratios of 18O and 2H in precipitation water

\begin{center}\rule{0.5\linewidth}{\linethickness}\end{center}

\textbf{Abstract}

This data product contains the quality-controlled, native sampling
resolution data from NEON's stable isotope concentrations in wet
deposition protocol. Deuterium and oxygen-18 concentrations are measured
in precipitation samples.

\begin{center}\rule{0.5\linewidth}{\linethickness}\end{center}

\textbf{Usage Notes}

Queries for this data product will return data from wdi\_collection,
wdi\_collectionIso, wdi\_collectionIsoTest, and wdi\_isoPerSample for
all dates within the specified date range. Each record in
wdi\_collection is expected to have one child record in each of
wdi\_collectionIso, wdi\_collectionIsoTest, and wdi\_isoPerSample. The
expanded package returns two additional tables: wdi\_sensor, which
contains automated data from the collector assembly for all months in
the date range requested, and asi\_externalLabSummaryData, which
contains the most recently updated QA/QC data returned by the analytical
laboratory. \newpage
.

\section{DP1.00040.001 Soil heat flux
plate}\label{dp1.00040.001-soil-heat-flux-plate}

\begin{center}\rule{0.5\linewidth}{\linethickness}\end{center}

\textbf{Subsystem}

Terrestrial Instrument System (TIS)

\begin{center}\rule{0.5\linewidth}{\linethickness}\end{center}

\textbf{Coverage}

Soil heat flux is measured at all of NEON's soil plots at terrestrial
sites.

\begin{center}\rule{0.5\linewidth}{\linethickness}\end{center}

\textbf{Description}

The amount of thermal energy moving by conduction across an area of soil
in a unit of time. Measured as part of the soil array.

\begin{center}\rule{0.5\linewidth}{\linethickness}\end{center}

\textbf{Abstract}

Soil heat flux is the amount of thermal energy that moves by conduction
across an area of soil in a unit of time and usually expressed in Watts
per square meter. This data product represents the soil heat flux at the
locations of the heat flux plates, 0.08 m below the soil surface. It is
reported as 1-minute mean measurements and 30-minute mean values.
\newpage
.

\section{DP1.00041.001 Soil
temperature}\label{dp1.00041.001-soil-temperature}

\begin{center}\rule{0.5\linewidth}{\linethickness}\end{center}

\textbf{Subsystem}

Terrestrial Instrument System (TIS)

\begin{center}\rule{0.5\linewidth}{\linethickness}\end{center}

\textbf{Coverage}

Soil temperature is measured in all five instrumented soil plots at each
terrestrial site.

\begin{center}\rule{0.5\linewidth}{\linethickness}\end{center}

\textbf{Description}

Temperature of the soil at various depth below the soil surface from 2
cm up to 200 cm at non-permafrost sites (up to 300 cm at Alaskan sites).
Data are from all five Instrumented Soil Plots per site and presented as
1-minute and 30-minute averages.

\begin{center}\rule{0.5\linewidth}{\linethickness}\end{center}

\textbf{Abstract}

Soil temperature is measured at various depths below the soil surface
from approximately 2 cm up to 200 cm at non-permafrost sites (up to 300
cm at Alaskan sites). Soil temperature influences the rate of
biogeochemical cycling, decomposition, and root and soil biota activity.
In addition, soil temperature can impact the hydrologic cycle since it
controls whether soil water is in a liquid or solid state. Measurements
are made in vertical profiles consisting of up to nine depths in all
five instrumented soil plots at each terrestrial site, and presented as
1-minute and 30-minute averages. \newpage
.

\section{DP1.00042.001 Snow depth and understory phenology
images}\label{dp1.00042.001-snow-depth-and-understory-phenology-images}

\begin{center}\rule{0.5\linewidth}{\linethickness}\end{center}

\textbf{Subsystem}

Terrestrial Instrument System (TIS)

\begin{center}\rule{0.5\linewidth}{\linethickness}\end{center}

\textbf{Coverage}

All terrestrial core and relocatable sites; camera is positioned at the
bottom of each site's tower.

\begin{center}\rule{0.5\linewidth}{\linethickness}\end{center}

\textbf{Description}

Camera images of snow depth relative to mounted/calibrated depth stakes
when snow is present; images used to track understory phenology when
possible (see NEON.DOM.SITE.DP1.00033).

\begin{center}\rule{0.5\linewidth}{\linethickness}\end{center}

\textbf{Abstract}

Phenology is the study of reoccurring life cycle events that are driven
by environmental factors (Morrisette et al., 2009). The timing of these
events is driven by both short- and long-term variability in climate and
is therefore valuable in understanding the effects of climate change
(Richardson et al., 2006). Automated repeat digital images of plant
canopies provide data for the extraction of indices (e.g.~green
chromatic coordinate (gcc)) that can be used to quantify changes in
phenological events over time (Sonnentag et al., 2011).

NEON has deployed a Stardot NetCam at the bottom of all terrestrial core
and re-locatable towers to study below-canopy phenology and snow depth.
Over time, these images can be used to detect seasonal changes in
understory vegetation (e.g., onset of leaf growth and senescence). The
camera will also capture images of snowdepth stakes. Images are sent to
and processed by PhenoCam, a cooperative network that archives and
distributes imagery and derived data products from digital cameras
deployed at research sites across North America and around the world.
NEON's phenocam images are available for viewing and downloading from
the \href{https://phenocam.sr.unh.edu}{PhenoCam Gallery}, along with
images and data from other phenocam sites across the world. \newpage
.

\section{DP1.00043.001 Spectral sun photometer - calibrated sky
radiances}\label{dp1.00043.001-spectral-sun-photometer---calibrated-sky-radiances}

\begin{center}\rule{0.5\linewidth}{\linethickness}\end{center}

\textbf{Subsystem}

Terrestrial Instrument System (TIS)

\begin{center}\rule{0.5\linewidth}{\linethickness}\end{center}

\textbf{Coverage}

Data are collected for a select subset of NEON terrestrial sites.
Sensors are located on the southeast-most corner of the tower top.

\begin{center}\rule{0.5\linewidth}{\linethickness}\end{center}

\textbf{Description}

Calibrated Sky Radiances; includes Almucantar Radiance Data and
Principal Plane Radiance Data.

\begin{center}\rule{0.5\linewidth}{\linethickness}\end{center}

\textbf{Abstract}

Sun photometer measurements of the direct (collimated) solar radiation
provide information to calculate the columnar aerosol optical depth
(AOD). AOD can then be used to compute columnar water vapor
(Precipitable Water) and estimate the aerosol size using the Angstrom
parameter relationship, and derive other inversion data products. Data
from NEON's sun photometers are uploaded daily to
\href{https://aeronet.gsfc.nasa.gov/}{NASA's Aerosol Robotic Network
(AERONET) program}, where they are checked for quality and processed.
AERONET produces numerous data products in addition to the
\href{http://data.neonscience.org/data-product-view?dpCode=DP1.00043.001}{Spectral
Sun Photometer - Calibrated Sky Radiances} data product, including
Aerosol Optical Depth and Total Column Water Vapor. Clicking on a link
below will open to an AERONET webpage providing a data download service
for the selected NEON site. To discover more AERONET-generated data
products as well as graphing and reporting tools, visit the
\href{https://aeronet.gsfc.nasa.gov/cgi-bin/type_piece_of_map_aod_v3}{AERONET
Data Display Interface} and click on your site of interest in the list.
\newpage
.

\section{DP1.00066.001 Photosynthetically active radiation (quantum
line)}\label{dp1.00066.001-photosynthetically-active-radiation-quantum-line}

\begin{center}\rule{0.5\linewidth}{\linethickness}\end{center}

\textbf{Subsystem}

Terrestrial Instrument System (TIS)

\begin{center}\rule{0.5\linewidth}{\linethickness}\end{center}

\textbf{Coverage}

Sensors will be deployed at all NEON terrestrial sites within the soil
plots.

\begin{center}\rule{0.5\linewidth}{\linethickness}\end{center}

\textbf{Description}

The quantum line sensor provides spatially averaged observations of
photosynthetically active radiation (PAR), i.e., wavelengths between
400-700 nm, at the soil surface over a one meter length. This data
product is available as one- and thirty-minute averages of 1 Hz
observations. Observations are obtained by sensors located throughout
the soil array.

\begin{center}\rule{0.5\linewidth}{\linethickness}\end{center}

\textbf{Abstract}

Photosynthetically active radiation measured at the soil surface via the
quantum line sensor provides information on the light availability at
the ground level. It is reported as 1-minute mean measurements and
30-minute mean values. \newpage
.

\section{DP1.00094.001 Soil water content and water
salinity}\label{dp1.00094.001-soil-water-content-and-water-salinity}

\begin{center}\rule{0.5\linewidth}{\linethickness}\end{center}

\textbf{Subsystem}

Terrestrial Instrument System (TIS)

\begin{center}\rule{0.5\linewidth}{\linethickness}\end{center}

\textbf{Coverage}

Soil water content and an index of salinity are measured in all five
instrumented soil plots at each terrestrial site.

\begin{center}\rule{0.5\linewidth}{\linethickness}\end{center}

\textbf{Description}

Soil volumetric water content and an index of salinity at various depth
below the soil surface from 2 cm up to 200 cm at non-permafrost sites
(up to 300 cm at Alaskan sites). Data are from all five Instrumented
Soil Plots per site and presented as 1-minute and 30-minute averages.

\begin{center}\rule{0.5\linewidth}{\linethickness}\end{center}

\textbf{Abstract}

Soil volumetric water content and an index of soil water ion content
(salinity) are measured at various depths below the soil surface from
approximately 2 cm up to 200 cm at non-permafrost sites (up to 300 cm at
Alaskan sites). Soil moisture is an important component of the
hydrologic cycle and is the dominant source of water for most plants and
soil organisms making it a key indicator of drought. In addition, soil
moisture status influences the severity of flooding and temperature
extremes, as well as physical, chemical and biological processes in the
soil. Measurements are made in vertical profiles consisting of up to
eight depths in all five instrumented soil plots at each terrestrial
site, and presented as 1-minute and 30-minute averages. \newpage
.

\section{DP1.00095.001 Soil CO2
concentration}\label{dp1.00095.001-soil-co2-concentration}

\begin{center}\rule{0.5\linewidth}{\linethickness}\end{center}

\textbf{Subsystem}

Terrestrial Instrument System (TIS)

\begin{center}\rule{0.5\linewidth}{\linethickness}\end{center}

\textbf{Coverage}

Soil CO2 concentration is measured in all five instrumented soil plots
at each terrestrial site.

\begin{center}\rule{0.5\linewidth}{\linethickness}\end{center}

\textbf{Description}

CO2 concentration in soil air at various depth below the soil surface
starting at 2 cm. Data are from all five Instrumented Soil Plots per
site and presented as 1-minute and 30-minute averages.

\begin{center}\rule{0.5\linewidth}{\linethickness}\end{center}

\textbf{Abstract}

CO2 concentrations are measured at different depths in the soil to allow
the gradient method to be used to estimate soil CO2 efflux rates when
combined with other NEON data products. Soil CO2 efflux is an important
component of the carbon cycle because it is one of the largest exchanges
of carbon between terrestrial ecosystems and the atmosphere. In
addition, since the vast majority of soil CO2 is produced by microbial,
root and soil faunal respiration, soil CO2 efflux is an indicator of
total soil biological activity. CO2 concentrations are measured in all
five Instrumented soil plots per terrestrial site and at various depths
below the soil surface starting at approximately 2 cm and data are
presented as 1-minute and 30-minute averages. CO2 sensors at different
depths within a soil plot are typically located within 1 m horizontally
of one another. The CO2 concentration of soil air is measured at three
depths within each plot, starting at approximately 2 cm. \newpage
.

\section{DP1.00096.001 Soil physical properties
(Megapit)}\label{dp1.00096.001-soil-physical-properties-megapit}

\begin{center}\rule{0.5\linewidth}{\linethickness}\end{center}

\textbf{Subsystem}

Terrestrial Instrument System (TIS)

\begin{center}\rule{0.5\linewidth}{\linethickness}\end{center}

\textbf{Coverage}

Soil physical properties are measured at one temporary soil pit at each
terrestrial site

\begin{center}\rule{0.5\linewidth}{\linethickness}\end{center}

\textbf{Description}

Soil taxonomy, horizon names, horizon depths, as well as soil bulk
density, porosity, texture (sand, silt, and clay content) in the
\textless{}= 2 mm soil fraction for each soil horizon. Data were derived
from a sampling location expected to be representative of the area where
the Instrumented Soil Plots per site are located and were collected once
during site construction. Also see distributed soil data products.

\begin{center}\rule{0.5\linewidth}{\linethickness}\end{center}

\textbf{Abstract}

Soil physical properties are measured by horizon from a single temporary
soil pit at each terrestrial site at depths of up to 200 cm at
non-permafrost sites (up to 300 cm at Alaskan sites). Soil properties
affect the movement of soil water and nutrients through the soil profile
and their availability to plants and soil organisms. In addition, these
properties affect the movement of heat and gases into and out of the
soil. The sampling location is expected to be representative of the NEON
sensor-based soil plots and this sampling activity is expected to occur
once at each NEON terrestrial site. Additional soil samples collected
from the same soil pit are archived in the
\href{http://www.neonscience.org/request-megapit-soil-samples}{NEON
Megapit Soil Archive} and are available upon request.

\begin{center}\rule{0.5\linewidth}{\linethickness}\end{center}

\textbf{Usage Notes}

Queries for this data product will return all data for mgp\_permegapit,
mgp\_perhorizon, mgp\_perbiogeosample, mgp\_perbulksample, and
mgp\_perarchivesample during the date range specified. There is expected
to be at least one record for each mgp\_permegapit.pitID and
mgp\_perhorizon.horizonID combination in mgp\_perbiogeosample,
mgp\_perbulksample and mgp\_perarchivesample. Duplicates may exist where
protocol and/or data entry aberrations have occurred; users should check
data carefully for anomalies before analyzing data. \newpage
.

\section{DP1.00097.001 Soil chemical properties
(Megapit)}\label{dp1.00097.001-soil-chemical-properties-megapit}

\begin{center}\rule{0.5\linewidth}{\linethickness}\end{center}

\textbf{Subsystem}

Terrestrial Instrument System (TIS)

\begin{center}\rule{0.5\linewidth}{\linethickness}\end{center}

\textbf{Coverage}

Soil chemical properties are measured at one temporary soil pit at each
terrestrial site.

\begin{center}\rule{0.5\linewidth}{\linethickness}\end{center}

\textbf{Description}

Total content of a range of chemical elements, pH, and electrical
conductivity in the \textless{}= 2 mm soil fraction for each soil
horizon. Data were derived from a sampling location expected to be
representative of the area where the Instrumented Soil Plots per site
are located and were collected once during site construction. Also see
distributed soil data products.

\begin{center}\rule{0.5\linewidth}{\linethickness}\end{center}

\textbf{Abstract}

Soil chemical properties are measured by horizon from a single temporary
soil pit at each terrestrial site at depths of up to 200 cm at
non-permafrost sites (up to 300 cm at Alaskan sites). Soil properties
affect the movement of soil water and nutrients through the soil profile
and their availability to plants and soil organisms. In addition, these
properties relate to the storage and accessibility of nutrients in the
soil and influence biogeochemical cycling rates. The sampling location
is expected to be representative of the NEON sensor-based soil plots and
this sampling activity is expected to occur once at each NEON
terrestrial site. Additional soil samples collected from the same soil
pit are archived in the
\href{http://www.neonscience.org/request-megapit-soil-samples}{NEON
Megapit Soil Archive} and are available upon request.

\begin{center}\rule{0.5\linewidth}{\linethickness}\end{center}

\textbf{Usage Notes}

Queries for this data product will return all data for mgc\_permegapit,
mgc\_perhorizon, mgc\_perbiogeosample, and mgc\_perarchivesample during
the date range specified. There is expected to be at least one record
for each mgc\_permegapit.pitID and mgc\_perhorizon.horizonID combination
in mgc\_perbiogeosample and mgc\_perarchivesample. Duplicates may exist
where protocol and/or data entry aberrations have occurred; users should
check data carefully for anomalies before analyzing data. \newpage
.

\section{DP1.00098.001 Relative
humidity}\label{dp1.00098.001-relative-humidity}

\begin{center}\rule{0.5\linewidth}{\linethickness}\end{center}

\textbf{Subsystem}

Terrestrial Instrument System (TIS)

\begin{center}\rule{0.5\linewidth}{\linethickness}\end{center}

\textbf{Coverage}

Relative humidity is measured at NEON terrestrial and aquatic sites.

\begin{center}\rule{0.5\linewidth}{\linethickness}\end{center}

\textbf{Description}

Relative humidity, temperature, and dew or frost point temperature,
available as one- and thirty-minute averages of 1 Hz observations.
Observations are made by sensors located at the top of the tower
infrastructure, in the soil array, and on the aquatic meteorologic
station.

\begin{center}\rule{0.5\linewidth}{\linethickness}\end{center}

\textbf{Abstract}

This data product contains the relative humidity, air temperature, and
dew point/frost point temperature measurements made at all NEON sites.
It is reported as 1-minute mean measurements and 30-minute mean values.
\newpage
.

\section{DP1.00099.001 CO2 concentration -
storage}\label{dp1.00099.001-co2-concentration---storage}

\begin{center}\rule{0.5\linewidth}{\linethickness}\end{center}

\textbf{Subsystem}

Terrestrial Instrument System (TIS)

\begin{center}\rule{0.5\linewidth}{\linethickness}\end{center}

\textbf{Coverage}

This data product is monitored at all terrestrial sites. Sensors are
located inside the instrument hut near the bottom of the tower. The air
samples from different measurement heights are pumped through gas tubing
to sensors for analysis.

\begin{center}\rule{0.5\linewidth}{\linethickness}\end{center}

\textbf{Description}

Concentration of CO2 in profile of tower; used in calculation of storage
terms in eddy covariance calculations of carbon exchange. This data
product is bundled into DP4.00200, Bundled data products - eddy
covariance, and is not available as a stand-alone download.

\begin{center}\rule{0.5\linewidth}{\linethickness}\end{center}

\textbf{Abstract}

This data product contains the quality-controlled measurement data and
associated metadata in HDF5 format. The key sub-data products include
CO2 molar fraction in the air at different measurement heights on tower
at all NEON terrestrial sites, and sensor associated environmental data.
The data are delivered with the Bundled data products - eddy covariance
data product (DP4.00200.001).

\begin{center}\rule{0.5\linewidth}{\linethickness}\end{center}

\textbf{Usage Notes}

During subsequent nominal operations, we plan to produce and publish the
data products in three phases, to accommodate a variety of use cases:
the initial near-real-time transition, a science reviewed quality
transition, and the epoch yearly transition. The initial near-real-time
transition is scheduled to process daily files at a 5-day delay after
data collection to accommodate a 9-day centered planar-fit window. If
the data has not been received from the field it will attempt to process
daily for 30\,days, and if not all data is available after this window a
force execution is performed populating a HDF5 file with metadata and
filling data with NaN's. The monthly file will be produced after all
daily files are available, no later than 30 days after the last daily
file was initially attempted to be processed. After the initial
transition, the NEON science team has a one month window to manually
flag data that were identified as suspect through field-based problem
tracking and resolution tickets or through additional manual data
quality analysis. Then, the science-reviewed transition will occur, and
the data will be republished to the data portal. The last transition
type is part of the yearly epoch versioning, which provides a fully
quality assured and quality controlled version of the data using the
latest full release of the processing code. This transition is scheduled
to occur 18 months after the initial data collection. \newpage
.

\section{DP1.00100.001 H2O concentration -
storage}\label{dp1.00100.001-h2o-concentration---storage}

\begin{center}\rule{0.5\linewidth}{\linethickness}\end{center}

\textbf{Subsystem}

Terrestrial Instrument System (TIS)

\begin{center}\rule{0.5\linewidth}{\linethickness}\end{center}

\textbf{Coverage}

This data product is monitored at all terrestrial sites. Sensors are
located inside the instrument hut near the bottom of the tower. The air
samples from different measurement heights are pumped through gas tubing
to sensors for analysis.

\begin{center}\rule{0.5\linewidth}{\linethickness}\end{center}

\textbf{Description}

Concentration of H2O in profile; used in calculation of storage terms in
eddy covariance calculations of water vapor exchange. This data product
is bundled into DP4.00200, Bundled data products - eddy covariance, and
is not available as a stand-alone download.

\begin{center}\rule{0.5\linewidth}{\linethickness}\end{center}

\textbf{Abstract}

This data product contains the quality-controlled measurement data and
associated metadata in HDF5 format. The key sub-data products include
H2O molar fraction in the air at different measurement heights on the
tower at all NEON terrestrial sites, and sensor-associated environmental
data. The data are delivered with the Bundled data products - eddy
covariance data product (DP4.00200.001).

\begin{center}\rule{0.5\linewidth}{\linethickness}\end{center}

\textbf{Usage Notes}

During subsequent nominal operations, we plan to produce and publish the
data products in three phases, to accommodate a variety of use cases:
the initial near-real-time transition, a science reviewed quality
transition, and the epoch yearly transition. The initial near-real-time
transition is scheduled to process daily files at a 5-day delay after
data collection to accommodate a 9-day centered planar-fit window. If
the data has not been received from the field it will attempt to process
daily for 30\,days, and if not all data is available after this window a
force execution is performed populating a HDF5 file with metadata and
filling data with NaN's. The monthly file will be produced after all
daily files are available, no later than 30 days after the last daily
file was initially attempted to be processed. After the initial
transition, the NEON science team has a one month window to manually
flag data that were identified as suspect through field-based problem
tracking and resolution tickets or through additional manual data
quality analysis. Then, the science-reviewed transition will occur, and
the data will be republished to the data portal. The last transition
type is part of the yearly epoch versioning, which provides a fully
quality assured and quality controlled version of the data using the
latest full release of the processing code. This transition is scheduled
to occur 18 months after the initial data collection. \newpage
.

\section{DP1.00101.001 Particulate
mass}\label{dp1.00101.001-particulate-mass}

\begin{center}\rule{0.5\linewidth}{\linethickness}\end{center}

\textbf{Subsystem}

Terrestrial Instrument System (TIS)

\begin{center}\rule{0.5\linewidth}{\linethickness}\end{center}

\textbf{Coverage}

These data are collected at six NEON terrestrial sites, on the eastern
and western slopes of the Rocky Mountains.

\begin{center}\rule{0.5\linewidth}{\linethickness}\end{center}

\textbf{Description}

Dust mass and density measured by a high volume dust sampler and quartz
filters.

\begin{center}\rule{0.5\linewidth}{\linethickness}\end{center}

\textbf{Abstract}

This product contains the quality-controlled, native sampling resolution
data from NEON's particulate mass sampling protocol. Samples are
collected by an automated assembly that pulls air through a quartz
microfiber filter with a porosity of 10 micrometers, to collect PM10.
Filters are weighed at high precision pre- and post-deployment to
determine dust deposition mass. In addition to determining mass
concentration of PM10, filters from the particulate mass analyzers will
be archived in laboratory storage. Subsamples of the filters will be
available to science community upon request to enable the assessment of
chemical and nutrient inputs in the region.

\begin{center}\rule{0.5\linewidth}{\linethickness}\end{center}

\textbf{Usage Notes}

Queries for this data product will return data from dpm\_field and
dpm\_lab for all dates within the specified date range. Each record in
dpm\_field is expected to have one child record in dpm\_lab. The
expanded package returns an additional table, dpm\_sensor, containing
automated data from the collector assembly for all months in the date
range requested. Duplicates may exist where protocol and/or data entry
aberrations have occurred; users should check carefully for anomalies
before analyzing data. \newpage
.

\section{DP1.10003.001 Breeding landbird point
counts}\label{dp1.10003.001-breeding-landbird-point-counts}

\begin{center}\rule{0.5\linewidth}{\linethickness}\end{center}

\textbf{Subsystem}

Terrestrial Observation System (TOS)

\begin{center}\rule{0.5\linewidth}{\linethickness}\end{center}

\textbf{Coverage}

This sampling occurs at all NEON terrestrial sites.

\begin{center}\rule{0.5\linewidth}{\linethickness}\end{center}

\textbf{Description}

Count, distance from observer, and taxonomic identification of breeding
landbirds observed during point counts

\begin{center}\rule{0.5\linewidth}{\linethickness}\end{center}

\textbf{Abstract}

This data product contains the quality-controlled, native sampling
resolution data from NEON's breeding landbird sampling. Breeding
landbirds are defined as ``smaller birds (usually exclusive of raptors
and upland game birds) not usually associated with aquatic habitats''
(Ralph et al. 1993). The breeding landbird point counts product provides
records of species identification of all individuals observed during the
6-minute count period, as well as metadata which can be used to model
detectability, e.g., weather, distances from observers to birds, and
detection methods. The NEON point count method is adapted from the
Integrated Monitoring in Bird Conservation Regions (IMBCR): Field
protocol for spatially-balanced sampling of landbird populations (Hanni
et al. 2017; \url{http://bit.ly/2u2ChUB}). For additional details, see
protocol
\href{http://data.neonscience.org/api/v0/documents/NEON.DOC.014041vF}{NEON.DOC.014041}:
TOS Protocol and Procedure: Breeding Landbird Abundance and Diversity
and science design
\href{http://data.neonscience.org/api/v0/documents/NEON.DOC.000916vB}{NEON.DOC.000916}:
TOS Science Design for Breeding Landbird Abundance and Diversity.

\begin{center}\rule{0.5\linewidth}{\linethickness}\end{center}

\textbf{Usage Notes}

Queries for this data product will return data collected during the date
range specified for brd\_perpoint and brd\_countdata, but will return
data from all dates for brd\_personnel (quiz scores may occur over time
periods which are distinct from when sampling occurs) and
brd\_references (which apply to a broad range of sampling dates). A
record from brd\_perPoint should have 6+ child records in
brd\_countdata, at least one per pointCountMinute. Duplicates or missing
data may exist where protocol and/or data entry aberrations have
occurred; users should check data carefully for anomalies before joining
tables. Taxonomic IDs of species of concern have been `fuzzed'; see data
package readme files for more information. \newpage
.

\section{DP1.10008.001}\label{dp1.10008.001}

Soil chemical properties (Distributed initial characterization)

\begin{center}\rule{0.5\linewidth}{\linethickness}\end{center}

\textbf{Subsystem}

Terrestrial Observation System (TOS)

\begin{center}\rule{0.5\linewidth}{\linethickness}\end{center}

\textbf{Coverage}

These data are collected at all NEON terrestrial sites.

\begin{center}\rule{0.5\linewidth}{\linethickness}\end{center}

\textbf{Description}

Soil chemical properties of a soil core that was sampled by the NRCS as
part of initial site characterization activities at the NEON site. Data
are reported by horizon for the top 1m of the soil profile. Also see
distributed periodic and megapit soil data.

\begin{center}\rule{0.5\linewidth}{\linethickness}\end{center}

\textbf{Abstract}

This data product contains quality-controlled, native sampling
resolution chemistry data from soils measured during the course of an
initial soil characterization effort at each NEON site. This effort is
executed by the Soil Science Division of the Natural Resources
Conservation Service (NRCS), in partnership with the USDA Agriculture
Research Service (ARS). Queries for this data product will return soil
chemistry data on a per horizon basis. Associated with these data are
soil pedon descriptions and narrative summary documents, which place the
plot-level data into site-level context. These documents can be found in
the \href{http://data.neonscience.org/documents}{NEON Document Library},
in the folder Soil Characterization Summaries \textgreater{} Distributed
plots.

\begin{center}\rule{0.5\linewidth}{\linethickness}\end{center}

\textbf{Usage Notes}

One record is expected to appear in spc\_biogeochem for each unique
value of biogeoIDnrcs, but duplicates and/or missing data may exist
where protocol and/or data entry aberrations have occurred. Soil pit and
horizon metadata associated with these records can be found in the Soil
physical properties (Distributed initial characterization) data product.
\newpage
.

\section{DP1.10010.001 Coarse downed wood log
survey}\label{dp1.10010.001-coarse-downed-wood-log-survey}

\begin{center}\rule{0.5\linewidth}{\linethickness}\end{center}

\textbf{Subsystem}

Terrestrial Observation System (TOS)

\begin{center}\rule{0.5\linewidth}{\linethickness}\end{center}

\textbf{Coverage}

Tallies for Coarse Downed Wood are conducted at all terrestrial NEON
sites at which qualifying logs greater than or equal to 2 cm diameter
are found. Functionally, surveyed sites include forested sites, and
somes sites dominated by woody shrub/scrub vegetation.

\begin{center}\rule{0.5\linewidth}{\linethickness}\end{center}

\textbf{Description}

Tally and raw measurement of coarse downed wood \textgreater{}= 2 cm
diameter

\begin{center}\rule{0.5\linewidth}{\linethickness}\end{center}

\textbf{Abstract}

The Coarse Downed Wood log survey data product contains the
quality-controlled, native sampling resolution data from in-situ tallies
and measurements of downed logs from each of NEON's terrestrial sites at
which qualifying logs are present. Qualifying logs are tallied within
each plot according to the Line Intercept Distance Sampling (LIDS)
method, and additional diameter, length and decay class characteristics
are measured for each tallied log. Data are reported per log per plot,
and when forked logs are tallied, additional diameter data are reported
for each qualifying log fork. For additional details, see protocol
\href{http://data.neonscience.org/api/v0/documents/NEON.DOC.001711vD}{NEON.DOC.001711vD}:
TOS Protocol and Procedure: Coarse Downed Wood, and Science Design
\href{http://data.neonscience.org/api/v0/documents/NEON.DOC.000914vA}{NEON.DOC.000914}:
TOS Science Design for Plant Biomass, Productivity and Leaf Area Index.

\begin{center}\rule{0.5\linewidth}{\linethickness}\end{center}

\textbf{Usage Notes}

Queries for this data product will return data from the cdw\_fieldtally
table, subset to data collected within the specified date range. Data
are provided in monthly download files; queries including any part of a
month will return data from the entire month. For plots surveyed within
a given bout that have no qualifying CDW logs, there will be a minimum
of three records in cdw\_fieldtally, one for each of the three transects
surveyed indicating targetTaxaPresent = N. When qualifying logs are
tallied within a plot (i.e., targetTaxaPresent = Y), there will be one
record in cdw\_fieldtally for each qualifying log. If a tallied log was
originally tagged for the Vegetation Structure protocol when it was
alive, the logID will correspond to at least one tagID in the
vst\_mapping table in the vegetation structure data product. Logs
tallied in the cdw\_fieldtally table are specifically avoided for bulk
density sampling, so logIDs recorded in cdw\_fieldtally should NOT be
duplicated in the CDW Bulk Density data product. Taxonomic IDs of
species of concern have been `fuzzed'; see data package readme files for
more information. \newpage
.

\section{DP1.10014.001 Coarse downed wood bulk density
sampling}\label{dp1.10014.001-coarse-downed-wood-bulk-density-sampling}

\begin{center}\rule{0.5\linewidth}{\linethickness}\end{center}

\textbf{Subsystem}

Terrestrial Observation System (TOS)

\begin{center}\rule{0.5\linewidth}{\linethickness}\end{center}

\textbf{Coverage}

Sampling for Coarse Downed Wood bulk density is conducted at all
terrestrial sites with qualifying logs where repeat CDW Survey tally
sampling is performed. Functionally, this includes forested sites, and
some sites dominated by woody shrub/scrub vegetation.

\begin{center}\rule{0.5\linewidth}{\linethickness}\end{center}

\textbf{Description}

Raw bulk density measurements of coarse downed wood \textgreater{}= 2 cm
diameter

\begin{center}\rule{0.5\linewidth}{\linethickness}\end{center}

\textbf{Abstract}

The Coarse Downed Wood bulk density sampling data product contains the
quality-controlled, native sampling resolution volume, mass and
calculated bulk density data from cross-sectional disks cut from downed
logs at each of NEON's terrestrial sites at which qualifying logs are
present. Disks are preferentially collected from logs that fall into the
most abundant `decayClass x sizeCategory x taxonID' combinations, as
informed by the Coarse Downed Wood survey data product. In addition to
bulk density from each collected disk, log-level decay, size category
and taxonID information is also recorded for each log from which disks
are sampled. Data are reported per disk per log, and when multiple disks
are collected from the same log, multiple records will exist in the
data. Logs are typically associated with a plotID, and disks may also be
collected from logs that fall outside of NEON plots. For additional
details, see protocol
\href{http://data.neonscience.org/api/v0/documents/NEON.DOC.001711vD}{NEON.DOC.001711vD}:
TOS Protocol and Procedure: Coarse Downed Wood, and Science Design
\href{http://data.neonscience.org/api/v0/documents/NEON.DOC.000914vA}{NEON.DOC.000914}:
TOS Science Design for Plant Biomass, Productivity and Leaf Area Index.

\begin{center}\rule{0.5\linewidth}{\linethickness}\end{center}

\textbf{Usage Notes}

The protocol specifies that for each sampleID in the cdw\_densitylog
table, there will be a maximum of two disk-level child records with
unique subsampleIDs in the cdw\_densitydisk table. If a sampled log was
originally tagged for the Vegetation Structure protocol when it was
alive, the logID will correspond to at least one tagID in the
vst\_mapping table. Logs for which bulk density samples are collected
are specifically targeted to avoid those logs that are tallied in the
cdw\_fieldtally table, and logIDs recorded in the cdw\_densitylog table
should NOT be duplicated in the CDW Log Survey data product. Queries for
this data product will return data from the cdw\_densitylog and
cdw\_densitydisk tables, subset to data collected for the user-specified
date range. Data are provided in monthly download files; queries
including any part of a month will return data from the entire month.
Taxonomic IDs of species of concern have been `fuzzed'; see data package
readme files for more information. \newpage
.

\section{DP1.10017.001 Digital hemispheric photos of plot
vegetation}\label{dp1.10017.001-digital-hemispheric-photos-of-plot-vegetation}

\begin{center}\rule{0.5\linewidth}{\linethickness}\end{center}

\textbf{Subsystem}

Terrestrial Observation System (TOS)

\begin{center}\rule{0.5\linewidth}{\linethickness}\end{center}

\textbf{Coverage}

These data are collected at NEON terrestrial sites.

\begin{center}\rule{0.5\linewidth}{\linethickness}\end{center}

\textbf{Description}

Upward and/or downward facing digital 180-degree images of vegetation in
plots used to calculate leaf area index

\begin{center}\rule{0.5\linewidth}{\linethickness}\end{center}

\textbf{Abstract}

This data product contains the quality-controlled, native sampling
resolution field data and 180 degree hemispherical images that enable
ground-based calculation of Leaf Area Index (LAI) and/or Plant Area
Index (PAI). For forests, both upward-facing photos of canopy
vegetation, and photos of understory vegetation are collected. For
shorter-stature ecosystems, only downward-facing images of `understory'
vegetation are collected, where `understory' is defined to include all
vegetation. Photos are acquired with a full-frame DSLR camera equipped
with a fisheye lens, and are provided in RAW image format. For
additional details, see protocol
\href{http://data.neonscience.org/api/v0/documents/NEON.DOC.014039vH}{NEON.DOC.014039}:
TOS Protocol and Procedure: Measurement of Leaf Area Index, and Science
Design
\href{http://data.neonscience.org/api/v0/documents/NEON.DOC.000914vA}{NEON.DOC.000914}:
TOS Science Design for Plant Biomass, Productivity and Leaf Area Index.

\begin{center}\rule{0.5\linewidth}{\linethickness}\end{center}

\textbf{Usage Notes}

Queries for this data product will return data from the dhp\_perbout and
dhp\_perimagefile tables, subset to data collected for the
user-specified range. For each record in the dhp\_perbout table, there
should be either 12 or 24 child records in the dhp\_perimagefile table,
depending on whether only understory, or both understory and overstory
images are acquired at a given plot. Data are provided in monthly
download files; queries including any part of a month will return data
from the entire month. Images may be accessed via two mechanisms: Direct
download of individual images from a cloud-storage location, or as part
of a packaged .zip file downloaded via the NEON Data Portal. The
imageFileUrl field in the `basic' download package can be used to
directly access individual images, and when downloaded via this
mechanism, file names correspond to values in the imageFileName field.
When downloaded as part of a .zip package, files are renamed to provide
more information about each file, and the name corresponds to values in
the downloadFileName field.

There may be fewer than the 12 images expected for a given boutID x
plotID x imageType combination if some images were culled during image
QC checks, e.g., out of focus images. Duplicates may exist where
protocol and/or data entry aberrations have occurred; users should check
data carefully for anomalies before joining tables. \newpage
.

\section{DP1.10020.001 Ground beetle sequences DNA
barcode}\label{dp1.10020.001-ground-beetle-sequences-dna-barcode}

\begin{center}\rule{0.5\linewidth}{\linethickness}\end{center}

\textbf{Subsystem}

Terrestrial Observation System (TOS)

\begin{center}\rule{0.5\linewidth}{\linethickness}\end{center}

\textbf{Coverage}

These data are collected at NEON terrestrial sites.

\begin{center}\rule{0.5\linewidth}{\linethickness}\end{center}

\textbf{Description}

CO1 DNA sequences from select ground beetles

\begin{center}\rule{0.5\linewidth}{\linethickness}\end{center}

\textbf{Abstract}

This data product contains the quality-controlled laboratory metadata
and QA results for NEON's cytochrome oxidase I (COI) barcoding of ground
beetles sequences. The DNA barcoding procedure involves the removal of
tissue, extracting and sequencing DNA from the tissue, and matching that
sequence data to sequences from previously identified voucher specimens.
DNA analysis serves a number of purposes, including verification of
taxonomy of specimens that do not receive expert identification,
clarification of the taxonomy of rare or cryptic species, and
characterization of diversity using molecular markers. For additional
details on ground beetle collection, see protocol
\href{http://data.neonscience.org/api/v0/documents/NEON.DOC.014050vK}{NEON.DOC.014050}:
TOS Protocol and Procedure: Ground Beetle Sampling and science design
\href{http://data.neonscience.org/api/v0/documents/NEON.DOC.NEON.DOC.000909vA}{NEON.DOC.000909vA}:
TOS Science Design for Ground Beetle Abundance and Diversity. Queries
for this data product will return metadata tables formatted for
submission to the Barcode of Life Database. These queries will also
provide links to the actual sequence data, which are publicly available
on the Barcode of Life Datasystems (BOLD,
\url{http://www.barcodinglife.com/}). The sequence data can be obtained
by following the links from the NEON data portal, or by directly
querying NEON data sets on the BOLD server. From the NEON portal, the
link ``BOLD Project: Ground beetle sequences DNA barcode'' redirects to
a page on the BOLD public data portal for the queried data. This is a
dynamic link and will automatically update based on the user query.

\begin{center}\rule{0.5\linewidth}{\linethickness}\end{center}

\textbf{Usage Notes}

Taxonomic IDs of species of concern have been `fuzzed'; see data package
readme files for more information. \newpage
.

\section{DP1.10022.001 Ground beetles sampled from pitfall
traps}\label{dp1.10022.001-ground-beetles-sampled-from-pitfall-traps}

\begin{center}\rule{0.5\linewidth}{\linethickness}\end{center}

\textbf{Subsystem}

Terrestrial Observation System (TOS)

\begin{center}\rule{0.5\linewidth}{\linethickness}\end{center}

\textbf{Coverage}

These data are collected at all NEON terrestrial sites.

\begin{center}\rule{0.5\linewidth}{\linethickness}\end{center}

\textbf{Description}

Taxonomically identified ground beetles and the plots and times from
which they were collected.

\begin{center}\rule{0.5\linewidth}{\linethickness}\end{center}

\textbf{Abstract}

This data product contains the quality-controlled, native sampling
resolution data from NEON's ground beetle sampling and specimen
processing protocols. Ground beetle abundance and diversity are sampled
via pitfall trapping at regular intervals by NEON field technicians at
core and relocatable sites. Following trap collection, all beetles from
the family Carabidae are sorted by NEON technicians and identified to
species or morphospecies. A subset of collected Carabidae are pointed or
pinned, while other specimens (non-pinned/non-pointed carabids,
invertebrate bycatch, and vertebrate bycatch) are stored in 95\% ethanol
for archiving, and may be pooled into a single archive vial per plot.
Regardless of storage method, all collections data are reported at a per
trap resolution. A subset of pinned ground beetles (up to 467 per site
per year) are sent to an expert taxonomist for secondary identification.
Identifications performed on these individuals may be used to estimate
uncertainty in parataxonomist identification by NEON technicians.

\begin{center}\rule{0.5\linewidth}{\linethickness}\end{center}

\textbf{Usage Notes}

The protocol dictates that each trap is collected once per bout (one
expected record per trapID per plotID per collectDate in
bet\_fielddata). A record from bet\_fielddata may have zero (if no
sample collected) or multiple child records in bet\_sorting depending on
number of taxa contained in the sampleID. A record from from
bet\_sorting may have zero (if no contents of the subsampleID pinned) or
multiple child records in bet\_parataxonomistID depending on the number
of individuals selected for pinning from each subsampleID. A record in
bet\_archivepooling may correspond to one or more records in
bet\_subsampling, where multiple subsampleIDs are pooled into a single
archiveVial. Each record in bet\_IDandpinning should have zero or one
corresponding records in bet\_expertTaxonomistIDProcessed, depending on
whether that individualID was selected for professional identification.
Each record in bet\_IDandpinning should also have zero or one
corresponding records in bet\_expertTaxonomistIDRaw. All beetles must be
sorted prior to pinning, so the total number of beetles collected can be
calculated as the sum of individualCount in bet\_sorting, though further
identifications may be updated based on the downstream workflow.
Taxonomic IDs of species of concern have been `fuzzed'; see data package
readme files for more information. \newpage
.

\section{DP1.10023.001 Herbaceous clip
harvest}\label{dp1.10023.001-herbaceous-clip-harvest}

\begin{center}\rule{0.5\linewidth}{\linethickness}\end{center}

\textbf{Subsystem}

Terrestrial Observation System (TOS)

\begin{center}\rule{0.5\linewidth}{\linethickness}\end{center}

\textbf{Coverage}

These data are collected at NEON terrestrial sites.

\begin{center}\rule{0.5\linewidth}{\linethickness}\end{center}

\textbf{Description}

Dry weight of herbaceous vegetation harvested from individual clip
strips, by functional type

\begin{center}\rule{0.5\linewidth}{\linethickness}\end{center}

\textbf{Abstract}

This data product contains the quality-controlled, native sampling
resolution data from NEON's Herbaceous biomass clip harvest sampling.
Herbaceous vegetation is operationally defined in this protocol as
non-woody plants (i.e.~grasses, sedges, forbs, some bryophytes, and
non-woody vines such as Convolvulus spp. and certain Rubus spp.), as
well as woody-stemmed plants with diameter at decimeter height (ddh)
\textless{} 1 cm at the time of sampling. For additional details, see
protocol
\href{http://data.neonscience.org/api/v0/documents/NEON.DOC.014037vG}{NEON.DOC.014037}:
TOS Protocol and Procedure: Measurement of Herbaceous Biomass and
Science Design
\href{http://data.neonscience.org/api/v0/documents/NEON.DOC.000914vA}{NEON.DOC.000914}:
TOS Science Design for Plant Biomass, Productivity, and Leaf Area Index.

\begin{center}\rule{0.5\linewidth}{\linethickness}\end{center}

\textbf{Usage Notes}

Queries for this data product for hbp\_perbout and hbp\_massdata files
will be subset to data collected during the date range specified. A
given hbp\_perbout.clipID is expected to be sampled zero or one times
per collectDate (local time). A record from hbp\_perbout may have zero
or more child records in hbp\_massdata, depending on whether or not
hebaceous material is present in the selected clip cell (indicated by
the hbp\_perbout.targetTaxaPresent field), whether clipped material was
sorted to functional group (hbp\_massdata.herbGroup), and whether
reweighing occurred for QA purposes (hbp\_massdata.qaDryMass).
Duplicates may exist where protocol and/or data entry abberations have
occurred; users should check data carefully for anomalies before joining
tables. \newpage
.

\section{DP1.10026.001 Plant foliar physical and chemical
properties}\label{dp1.10026.001-plant-foliar-physical-and-chemical-properties}

\begin{center}\rule{0.5\linewidth}{\linethickness}\end{center}

\textbf{Subsystem}

Terrestrial Observation System (TOS)

\begin{center}\rule{0.5\linewidth}{\linethickness}\end{center}

\textbf{Coverage}

Foliar sampling is conducted at all NEON terrestrial sites.

\begin{center}\rule{0.5\linewidth}{\linethickness}\end{center}

\textbf{Description}

Plant sun-lit canopy foliar physical (e.g., leaf mass per area) and
chemical properties reported at the level of the individual (woody
plants) or community (herbaceous plants).

\begin{center}\rule{0.5\linewidth}{\linethickness}\end{center}

\textbf{Abstract}

This data product contains the quality-controlled, native sampling
resolution data from NEON's measurement of foliar traits in sun-lit,
georeferenced vegetation samples, including leaf mass per area,
chlorophyll, major and minor elements, and lignin. Whenever possible,
foliar data are collected in conjunction with overflights of the NEON
Airborne Observation Platform (AOP). For additional details on sampling
procedures, see
\href{http://data.neonscience.org/api/v0/documents/NEON.DOC.001024vD}{NEON.DOC.001024}:
TOS Protocol and Procedure: Canopy Foliage Sampling. Queries for this
data product will return field collection data and physical and chemical
measurements on a per sample basis. For carbon and nitrogen stable
isotope ratios measured in the same samples, see the Plant foliar stable
isotopes data product.

\begin{center}\rule{0.5\linewidth}{\linethickness}\end{center}

\textbf{Usage Notes}

Queries for this data product will return data collected during the date
range specified. Each foliage sample has a unique identifer (sampleID)
that is expected to appear once in the cfc\_fieldData table. A record
from cfc\_fieldData is then expected to appear once in cfc\_LMA and
several times in cfc\_chemistrySubsampling. Each record from
cfc\_chemistrySubsampling is expected to appear one to two times (if
analytical replicates were conducted) in cfc\_chlorophyll,
cfc\_carbonNitrogen, cfc\_elements and cfc\_lignin. There should only be
one instance per sampleID x analyticalRepNumber combination in each
table, but duplicates may exist where protocol and/or data entry
aberrations have occurred. Users should check data carefully for
anomalies before analyzing data. \newpage
.

\section{DP1.10031.001 Litter chemical
properties}\label{dp1.10031.001-litter-chemical-properties}

\begin{center}\rule{0.5\linewidth}{\linethickness}\end{center}

\textbf{Subsystem}

Terrestrial Observation System (TOS)

\begin{center}\rule{0.5\linewidth}{\linethickness}\end{center}

\textbf{Coverage}

These data are collected at NEON terrestrial sites with overstory
vegetation.

\begin{center}\rule{0.5\linewidth}{\linethickness}\end{center}

\textbf{Description}

Bulk litter chemistry at the scale of a plot. Data are reported by
functional group (leaves vs.~needles).

\begin{center}\rule{0.5\linewidth}{\linethickness}\end{center}

\textbf{Abstract}

This data product contains the quality-controlled, native sampling
resolution data from NEON's measurement of carbon, nitrogen, and lignin
concentrations in litterfall. Litter is defined as material that is
dropped from the forest canopy and has a butt end diameter \textless{}2
cm and a length \textless{}50 cm; this material is collected in elevated
0.5 m2 PVC traps. After sorting by functional group, needles and leaves
are sent for chemical analysis. For additional details, see protocol
\href{http://data.neonscience.org/api/v0/documents/NEON.DOC.001710vE}{NEON.DOC.001710}:
TOS Field and Lab Protocol for Litterfall and Fine Woody Debris.

\begin{center}\rule{0.5\linewidth}{\linethickness}\end{center}

\textbf{Usage Notes}

Queries for this data product will return data collected during the date
range specified. A cnSampleID, the unique identifier for a litter
subsample analyzed for carbon and nitrogen concentrations, may appear
one to two times in ltr\_litterCarbonNitrogen, depending on whether an
analytical replicate was run. A ligninSampleID, the unique identifier
for a litter subsample analyzed for lignin concentration, may appear one
to two times in ltr\_litterLignin, depending on whether an analytical
replicate was run. There should only be one instance per cnSampleID OR
ligninSampleID x analyticalRepNumber combination, but duplicates may
exist where protocol and/or data entry abberrations have occurred. Users
should check data carefully for anomalies before analyzing data.
\newpage
.

\section{DP1.10033.001 Litterfall and fine woody debris
sampling}\label{dp1.10033.001-litterfall-and-fine-woody-debris-sampling}

\begin{center}\rule{0.5\linewidth}{\linethickness}\end{center}

\textbf{Subsystem}

Terrestrial Observation System (TOS)

\begin{center}\rule{0.5\linewidth}{\linethickness}\end{center}

\textbf{Coverage}

These data are collected at NEON terrestrial sites with overstory
vegetation.

\begin{center}\rule{0.5\linewidth}{\linethickness}\end{center}

\textbf{Description}

Dry weight of litterfall and fine woody debris collected from elevated
litter traps and ground traps, by functional group

\begin{center}\rule{0.5\linewidth}{\linethickness}\end{center}

\textbf{Abstract}

This data product contains the quality-controlled, native sampling
resolution data from NEON's Litterfall and fine woody debris sampling.
Litter is defined as material that is dropped from the forest canopy and
has a butt end diameter \textless{}2cm and a length \textless{}50 cm;
this material is collected in elevated 0.5m2 PVC traps. Fine woody
debris is defined as material that is dropped from the forest canopy and
has a butt end diameter \textless{}2cm and a length \textgreater{}50 cm;
this material is collected in ground traps as longer material is not
reliably collected by the elevated traps. For additional details, see
protocol
\href{http://data.neonscience.org/api/v0/documents/NEON.DOC.001710vE}{NEON.DOC.001710}:
TOS Field and Lab Protocol for Litterfall and Fine Woody Debris and
science design
\href{http://data.neonscience.org/api/v0/documents/NEON.DOC.000914vA}{NEON.DOC.000914}:
TOS Science Design for Plant Biomass, Productivity, and Leaf Area Index.

\begin{center}\rule{0.5\linewidth}{\linethickness}\end{center}

\textbf{Usage Notes}

Queries for this data product will return data from all dates for
ltr\_pertrap (which may be established many years before a litter
collection event), whereas ltr\_fielddata, ltr\_massdata, and
ltr\_chemistrysubsampling files will be subset to data collected during
the date range specified. The protocol dictates that each trap is
established once (one expected record per trapID in ltr\_pertrap). A
record from ltr\_pertrap may have zero or more child records in
ltr\_fielddata.trapID, depending on the date range of the data
downloaded; a given ltr\_fielddata.trapID is expected to occur be
sampled zero or one times per collectDate (local time). A record from
from ltr\_fielddata may have zero (if no litter present) or more child
records in ltr\_massdata depending on the functional groups contained in
the trap and whether reweighing occurred for QA purposes. A record from
from ltr\_massdata may have zero (if not sent for chemistry analyses) or
one child records in ltr\_chemistrySubsampling. Duplicates may exist
where protocol and/or data entry aberrations have occurred; users should
check data carefully for anomalies before joining tables. \newpage
.

\section{DP1.10035.001 Bryophyte clip
harvest}\label{dp1.10035.001-bryophyte-clip-harvest}

\begin{center}\rule{0.5\linewidth}{\linethickness}\end{center}

\textbf{Subsystem}

Terrestrial Observation System (TOS)

\begin{center}\rule{0.5\linewidth}{\linethickness}\end{center}

\textbf{Coverage}

These data are collected at NEON terrestrial sites with at least 20\%
bryophyte cover, averaged across all tower plots. Bryophyte clip harvest
is not performed in Distributed Plots.

\begin{center}\rule{0.5\linewidth}{\linethickness}\end{center}

\textbf{Description}

Dry weight of bryophyte biomass samples

\begin{center}\rule{0.5\linewidth}{\linethickness}\end{center}

\textbf{Abstract}

This data product contains the quality-controlled, native sampling
resolution data from NEON's Bryophyte productivity clip harvest
sampling. Bryophytes are operationally defined in this protocol as
mosses (including Sphagnum sp.) and liverworts. The sampling period for
each collection is a year. For additional details, see
\href{http://data.neonscience.org/api/v0/documents/NEON.DOC.001709vB}{NEON.DOC.001709}:
TOS Protocol and Procedure: Bryophyte Productivity and
\href{http://data.neonscience.org/api/v0/documents/NEON.DOC.000914vA}{NEON.DOC.000914}:
TOS Science Design for Plant Biomass, Productivity, and Leaf Area Index.

\begin{center}\rule{0.5\linewidth}{\linethickness}\end{center}

\textbf{Usage Notes}

Queries for this data product will be subset to data collected during
the date range specified. A given clipID in the bry\_productivity table
is intended to be harvested only once. The unique sampleID is generated
from the clipID and the collectDate. Each sampleID created results in
one record in the bry\_productivity table. Duplicates and/or missing
data may exist where protocol and/or data entry aberrations have
occurred; users should check data carefully for anomalies before joining
tables. \newpage
.

\section{DP1.10038.001 Mosquito sequences DNA
barcode}\label{dp1.10038.001-mosquito-sequences-dna-barcode}

\begin{center}\rule{0.5\linewidth}{\linethickness}\end{center}

\textbf{Subsystem}

Terrestrial Observation System (TOS)

\begin{center}\rule{0.5\linewidth}{\linethickness}\end{center}

\textbf{Coverage}

These data are collected at NEON terrestrial sites.

\begin{center}\rule{0.5\linewidth}{\linethickness}\end{center}

\textbf{Description}

CO1 DNA sequences from select mosquitoes

\begin{center}\rule{0.5\linewidth}{\linethickness}\end{center}

\textbf{Abstract}

This data product contains the quality-controlled laboratory metadata
and QA results for NEON's cytochrome oxidase I (COI) barcoding of
mosquito sequences. The DNA barcoding procedure involves the removal of
leg tissue from pinned specimens, extracting and sequencing DNA from the
tissue, and matching that sequence data to sequences from previously
identified voucher specimens. DNA analysis serves a number of purposes,
including verification of taxonomy of specimens that do not receive
expert identification, clarification of the taxonomy of rare or cryptic
species, and characterization of diversity using molecular markers. For
additional details on mosquito collection, see protocol
\href{http://data.neonscience.org/api/v0/documents/NEON.DOC.014049vH}{NEON.DOC.014049}:
TOS Protocol and Procedure: Mosquito Sampling and science design
\href{http://data.neonscience.org/api/v0/documents/NEON.DOC.NEON.DOC.000908vA}{NEON.DOC.000908}:
TOS Science Design for Terrestrial Microbial Diversity. Queries for this
data product will return metadata tables formatted for submission to the
Barcode of Life Database. These queries will also provide links to the
actual sequence data, which are publicly available on the Barcode of
Life Datasystem (BOLD, \url{http://www.barcodinglife.com/}). The
sequence data can be obtained by following the links from the NEON data
portal, or by directly querying NEON data sets on the BOLD server. From
the NEON portal, the link ``BOLD Project: Mosquito sequences DNA
barcode'' redirects to a page on the BOLD public data portal for the
queried data. This is a dynamic link and will automatically update based
on the user query.

\begin{center}\rule{0.5\linewidth}{\linethickness}\end{center}

\textbf{Usage Notes}

Taxonomic IDs of species of concern have been `fuzzed'; see data package
readme files for more information. \newpage
.

\section{DP1.10041.001 Mosquito-borne pathogen
status}\label{dp1.10041.001-mosquito-borne-pathogen-status}

\begin{center}\rule{0.5\linewidth}{\linethickness}\end{center}

\textbf{Subsystem}

Terrestrial Observation System (TOS)

\begin{center}\rule{0.5\linewidth}{\linethickness}\end{center}

\textbf{Coverage}

These data are collected at NEON terrestrial sites.

\begin{center}\rule{0.5\linewidth}{\linethickness}\end{center}

\textbf{Description}

Presence/absence of a pathogen in a single mosquito sample (pool)

\begin{center}\rule{0.5\linewidth}{\linethickness}\end{center}

\textbf{Abstract}

This data product contains the quality-controlled, native sampling
resolution data derived from NEON's mosquito pathogen testing. Products
resulting from this sampling include pathogen test results of mosquito
pools derived from identified mosquitoes collected during NEON mosquito
sampling. See NEON Product Mosquitoes sampled from CO2 traps
(DP1.10043.001) for data on the abundance and diversity of mosquitoes
collected at NEON sites. Following collection, samples are sent to a
professional taxonomist where a subsample of the catch generated from
each trap is identified to species and sex. A subset of
postively-identified mosquitoes are later processed for pathogen
testing. Only female mosquitoes identified to the species-level and
captured in sufficient quantity over a season from likely vector species
are eligible for pathogen testing. Mosquitoes that meet pathogen-testing
criteria collected within the same site and sampling bout are
homogenized into a large pool of female conspecifics and then subdivided
into testing vials of appropriate pool sizes for pathogen testing. Each
vial is tested one or more times using a variety of methods using a
variety of methods which may include RT-PCR, Vero cell culture, and melt
curve assays. These methods vary in target specificity, from general
(e.g., Vero cell culture) to specific viral species (e.g., RT-PCR). Most
pools of mosquitoes are negative because pathogens are rare; when pools
are determined to be positive for any virus, the identit(ies) of the
virus(es) are determined to the species-level, if possible. Test results
yield data on the presence of important mosquito pathogens (e.g., West
Nile virus, Eastern equine encephalitis virus, Dengue, etc) in a subset
of species that are known vectors of disease.. For additional details,
see science design
\href{htttp://data.neonscience.org/api/v0/documents/NEON.DOC.000911vA}{NEON.DOC.000911}:TOS
Science Design for Vectors and Pathogens.

\begin{center}\rule{0.5\linewidth}{\linethickness}\end{center}

\textbf{Usage Notes}

Queries for this data product will return data collected during the date
range specified. The protocol dictates that each
mos\_pathogen\_pooling.testingVialID should be recorded once, providing
linkages back to source collections in the Mosquitoes sampled from CO2
traps product (NEON.DP1.10043.001). A record from mos\_pathogenpooling
may have one or more child records in mos\_pathogenresults, depending on
the number of pathogen tests applied to the vial. Duplicates may exist
where protocol and/or data entry abberations have occurred; users should
check data carefully for anomalies before joining tables. \newpage
.

\section{DP1.10043.001 Mosquitoes sampled from CO2
traps}\label{dp1.10043.001-mosquitoes-sampled-from-co2-traps}

\begin{center}\rule{0.5\linewidth}{\linethickness}\end{center}

\textbf{Subsystem}

Terrestrial Observation System (TOS)

\begin{center}\rule{0.5\linewidth}{\linethickness}\end{center}

\textbf{Coverage}

These data are collected at NEON terrestrial sites.

\begin{center}\rule{0.5\linewidth}{\linethickness}\end{center}

\textbf{Description}

Taxonomically identified mosquitoes and the plots and times from which
they were collected

\begin{center}\rule{0.5\linewidth}{\linethickness}\end{center}

\textbf{Abstract}

This data product contains the quality-controlled, native sampling
resolution data from NEON's mosquito sampling protocol. Mosquito
abundance and diversity are sampled at regular intervals by NEON field
technicians at core and relocatable sites. For additional details on
protocol, see the TOS Protocol and Procedure: Mosquito Sampling.
Following collection, samples are sent to a professional taxonomist
where a subsample each catch generated from each trap is identified to
species and sex. Identified mosquitoes are then processed for pathogen
analysis or preserved for final archiving. Products resulting from this
sampling and processing include records of when mosquitoes were sampled,
the taxonomic and abundance data for a subset of mosquitoes captured,
and information about the material archived from the sample. For
additional details, see protocol
\href{http://data.neonscience.org/api/v0/documents/NEON.DOC.014049vH}{NEON.DOC.014049}:
TOS Protocol and Procedure: Mosquito Sampling and science design
\href{http://data.neonscience.org/api/v0/documents/NEON.DOC.000910vA}{NEON.DOC.000910}:
TOS Science Design for Mosquito Abundance, Diversity and Phenology.

\begin{center}\rule{0.5\linewidth}{\linethickness}\end{center}

\textbf{Usage Notes}

Queries for this data product will return data collected during the date
range specified. The protocol dictates that each mos\_trapping.sampleID
should be recorded once. A record in mos\_trapping should may have 0 (if
no mosquitoes collected) or one child record in mos\_sorting, which
generates a single subsampleID. Each mos\_identification.subsampleID
generates a single record in mos\_identification for each scientificName
* sex combination. The value(s) contained in the
mos\_identification.archiveID may be used to find associated records in
mos\_archivepooling. The value(s) contained in
mos\_identification.individualIDList may be used to find associated
records in the Mosquito sequences DNA barcode product (DP1.10038.001).
The values in mos\_identification.testingID may be used to find
associated records in the Mosquito-borne pathogen status product
(DP1.10041.001). Duplicates may exist where protocol and/or data entry
abberations have occurred; users should check data carefully for
anomalies before joining tables. \newpage
.

\section{DP1.10045.001 Non-herbaceous perennial vegetation
structure}\label{dp1.10045.001-non-herbaceous-perennial-vegetation-structure}

\begin{center}\rule{0.5\linewidth}{\linethickness}\end{center}

\textbf{Subsystem}

Terrestrial Observation System (TOS)

\begin{center}\rule{0.5\linewidth}{\linethickness}\end{center}

\textbf{Coverage}

These data are collected at all NEON terrestrial sites at which
qualifying growth forms are present at 10\% cover or greater, or in the
case of tree palms, when individuals are present in 10\% or more of
designated plots.

\begin{center}\rule{0.5\linewidth}{\linethickness}\end{center}

\textbf{Description}

Field measurements of individual non-herbaceous perennial plants
(e.g.~cacti, ferns)

\begin{center}\rule{0.5\linewidth}{\linethickness}\end{center}

\textbf{Abstract}

This data product contains the quality-controlled, native sampling
resolution data from in-situ structural measurements of live and
standing dead non-herbaceous perennial plants, from all terrestrial NEON
sites with qualifying vegetation. Non-herbaceous perennial plants
include agave, cactus, ferns, ocotillo, palms, xerophyllum and yucca
species. The exact measurements collected per individual depend on
growth form, and these measurements are focused on biomass and
productivity estimation, when suitable allometries exist, and estimation
of volume. Tree Palms are the only non-herbaceous perennial individuals
that are consistently mapped; smaller palms such as Serenoa repens may
be mapped if no taller overstory is present in a given plot. Tagging is
also generally dependent on growth form, and may also depend on species.
For example, palms are always tagged, large-stature cactus are tagged,
small-stature cactus are not tagged, and ferns are never tagged.
Individuals of all growth forms except tree palms may be subsampled
according to a nested subplot approach in order to standardize the per
plot sampling effort. Structure and mapping data are reported per
individual per plot; sampling metadata, such as per growth form sampling
area, are reported per plot. For additional details, see protocol
\href{http://data.neonscience.org/api/v0/documents/NEON.DOC.000987vG}{NEON.DOC.000987vG}:
TOS Protocol and Procedure: Measurement of Vegetation Structure, and
Science Design
\href{http://data.neonscience.org/api/v0/documents/NEON.DOC.000914vA}{NEON.DOC.000914}:
TOS Science Design for Plant Biomass, Productivity and Leaf Area Index.

\begin{center}\rule{0.5\linewidth}{\linethickness}\end{center}

\textbf{Usage Notes}

Queries for this data product will return data from vst\_perplotperyear
and nst\_perindividual, subset to data collected during the date range
specified. Data are provided in monthly download files; queries
including any part of a month will return data from the entire month.
For mapped locations of palms, users must download the
vst\_mappingandtagging table from the related Woody Plant Vegetation
Structure data product (NEON.DP1.10098), and join on the individualID
variable. In the vst\_perplotperyear table, there should be one record
per plotID per eventID, and data in this table describe the
presence/absence of non-herbaceous perennial growth forms, as well as
the sampling area utilized. The nst\_perindividual table contains one
record per individualID per eventID, and contains growth form, structure
and status data that may be linked to vst\_perplotperyear via the plotID
and eventID fields. The nst\_perindividual table may also contain
records with no individualID, since some individuals are measured but
not tagged (e.g., ferns). For all tables, duplicates may exist where
protocol and/or data entry aberrations have occurred; users should check
data carefully for anomalies before joining tables. Taxonomic IDs of
species of concern have been `fuzzed'; see data package readme files for
more information. \newpage
.

\section{DP1.10047.001}\label{dp1.10047.001}

Soil physical properties (Distributed initial characterization)

\begin{center}\rule{0.5\linewidth}{\linethickness}\end{center}

\textbf{Subsystem}

Terrestrial Observation System (TOS)

\begin{center}\rule{0.5\linewidth}{\linethickness}\end{center}

\textbf{Coverage}

These data are collected at all NEON terrestrial sites.

\begin{center}\rule{0.5\linewidth}{\linethickness}\end{center}

\textbf{Description}

Soil physical properties of a soil core that was sampled by the NRCS as
part of initial site characterization activities at the NEON site. Data
are reported by horizon for the top 1m of the soil profile. Also see
distributed periodic and megapit soil data.

\begin{center}\rule{0.5\linewidth}{\linethickness}\end{center}

\textbf{Abstract}

This data product contains quality-controlled, native sampling
resolution taxonomic and physical data from soils measured during the
course of an initial soil characterization effort at each NEON site.
This effort is executed by the Soil Science Division of the Natural
Resources Conservation Service (NRCS), in partnership with the USDA
Agriculture Research Service (ARS). Queries for this data product will
return field collection, bulk density, and particle size distribution
data on a per horizon basis. Associated with these data are soil pedon
descriptions and narrative summary documents, which place the plot-level
data into site-level context. These documents can be found in the
\href{http://data.neonscience.org/documents}{NEON Document Library}, in
the folder Soil Characterization Summaries \textgreater{} Distributed
plots.

\begin{center}\rule{0.5\linewidth}{\linethickness}\end{center}

\textbf{Usage Notes}

Each soil pit sampled yields a unique pitID in the spc\_perplot table. A
record from spc\_perplot then has several child records, one for each
horizon in the pit, in spc\_perhorizon. Each horizon record from
spc\_perhorizon will then have zero or one child records in
spc\_bulkdensity and spc\_particlesize. Duplicates may exist where
protocol and/or data entry abberrations have occurred; users should
check data carefully for anomalies before joining tables. \newpage
.

\section{DP1.10053.001 Plant foliar stable
isotopes}\label{dp1.10053.001-plant-foliar-stable-isotopes}

\begin{center}\rule{0.5\linewidth}{\linethickness}\end{center}

\textbf{Subsystem}

Terrestrial Observation System (TOS)

\begin{center}\rule{0.5\linewidth}{\linethickness}\end{center}

\textbf{Coverage}

Foliar sampling is conducted at all NEON terrestrial sites.

\begin{center}\rule{0.5\linewidth}{\linethickness}\end{center}

\textbf{Description}

Plant sun-lit canopy foliar stable isotope values reported at the level
of the individual (woody plants) or community (herbaceous plants).

\begin{center}\rule{0.5\linewidth}{\linethickness}\end{center}

\textbf{Abstract}

This data product contains the quality-controlled, native sampling
resolution data from NEON's measurement of foliar stable isotope ratios
in sun-lit, georeferenced vegetation samples, including delta 15N and
delta 13C values. Whenever possible, foliar stable isotope ratios are
collected in conjunction with overflights of the NEON Airborne
Observation Platform (AOP). For additional details on sampling
procedures, see {[}NEON.DOC.001024: TOS Protocol and Procedure: Canopy
Foliage Sampling{]}
(\url{http://data.neonscience.org/api/v0/documents/NEON.DOC.001024vD}).
Queries for this data product will return carbon and nitrogen stable
isotope ratios on a per sample basis. For field collection metadata and
foliar physical and chemical properties (leaf mass per area, lignin,
chlorophyll, major and minor elements) measured on the same samples, see
the Plant foliar physical and chemical properties data product.

\begin{center}\rule{0.5\linewidth}{\linethickness}\end{center}

\textbf{Usage Notes}

Queries for this data product will return data collected during the date
range specified. Upon collection, each foliage sample receives a unique
identifer (sampleID) that is expected to appear once in the
cfc\_fieldData table in Plant foliar physical and chemical properties.
Each record from cfc\_fieldData is expected to appear one to two times
(if analytical replicates were conducted) in cfc\_foliarStableIsotopes.
There should only be one instance per sampleID x analyticalRepNumber
combination, but duplicates may exist where protocol and/or data entry
aberrations have occurred. Users should check data carefully for
anomalies before analyzing data. Taxonomic IDs of species of concern
have been `fuzzed'; see data package readme files for more information.
\newpage
.

\section{DP1.10055.001 Plant phenology
observations}\label{dp1.10055.001-plant-phenology-observations}

\begin{center}\rule{0.5\linewidth}{\linethickness}\end{center}

\textbf{Subsystem}

Terrestrial Observation System (TOS)

\begin{center}\rule{0.5\linewidth}{\linethickness}\end{center}

\textbf{Coverage}

Plant phenology is monitored at NEON terrestrial sites, typically within
the tower airshed.

\begin{center}\rule{0.5\linewidth}{\linethickness}\end{center}

\textbf{Description}

Phenophase status and intensity of tagged plants

\begin{center}\rule{0.5\linewidth}{\linethickness}\end{center}

\textbf{Abstract}

This data product contains the quality-controlled, native sampling
resolution data from in-situ observations of plant leaf development and
reproductive phenophases, at each of NEON's terrestrial sites.
Phenophase status and intensity definitions follow those of the USA
National Phenology Network (USA-NPN). Status and intensity data are
reported per phenophase per individual or patch, for each day observed.
For additional details, see protocol
\href{http://data.neonscience.org/api/v0/documents/NEON.DOC.014040vJ}{NEON.DOC.014040}:
TOS Protocol and Procedure: Plant Phenology, and Science Design
\href{http://data.neonscience.org/api/v0/documents/NEON.DOC.000907vA}{NEON.DOC.000907}:
TOS Science Design for Plant Phenology.

\begin{center}\rule{0.5\linewidth}{\linethickness}\end{center}

\textbf{Usage Notes}

Queries for this data product will return data from all dates for
phe\_perindividual (which may be tagged for sampling many years before a
phenology observation), whereas phe\_perindividualperyear and
phe\_statusintensity files will be subset to data collected during the
date range specified. The protocol dictates that each individual is
established once (one expected record per individualID in
phe\_perindividual from initial establishment), but additional records
in phe\_perindividual for a given individualID may occur when subsequent
visits determine an update to the taxonomic identification or relative
position is warranted. Each actively monitored individual is intended to
be measured once per year for size and disease status, leading to one
record in phe\_perindividualperyear per calendar year for each
individualID in phe\_perindividual. Individuals that have died or
otherwise been dropped for monitoring may have 0 records in
phe\_perindividualperyear. An record from phe\_perindividual may have
zero or one child records in phe\_statusintensity per date (local time),
depending on the date range of the data downloaded; a given
phe\_perindividual.individualID is expected to be sampled for phenophase
status and intensity a maximum of once per day. Duplicates may exist
where protocol and/or data entry abberations have occurred; users should
check data carefully for anomalies before joining tables. Taxonomic IDs
of species of concern have been `fuzzed'; see data package readme files
for more information. \newpage
.

\section{DP1.10058.001 Plant presence and percent
cover}\label{dp1.10058.001-plant-presence-and-percent-cover}

\begin{center}\rule{0.5\linewidth}{\linethickness}\end{center}

\textbf{Subsystem}

Terrestrial Observation System (TOS)

\begin{center}\rule{0.5\linewidth}{\linethickness}\end{center}

\textbf{Coverage}

These data are collected at NEON terrestrial sites.

\begin{center}\rule{0.5\linewidth}{\linethickness}\end{center}

\textbf{Description}

Plant species presence as observed in multi-scale plots: species and
associated percent cover at 1-m2 and plant species presence at 10-m2,
100-m2 and 400-m2

\begin{center}\rule{0.5\linewidth}{\linethickness}\end{center}

\textbf{Abstract}

This data product contains the quality-controlled, native sampling
resolution data from NEON's plant diversity sampling. The presence and
percent cover of species is documented in square, multi-scale plots. The
presence and percent cover of plant species and ground cover is observed
in eight 1m\^{}2 subplots per plot. The presence of species is observed
in eight 10m\^{}2 subplots and four 100m\^{}2 subplots per plot, which
can be combined for a list of species at the 400m\^{}2 plot scale. For
additional details, see protocol
\href{http://data.neonscience.org/api/v0/documents/NEON.DOC.014042vG}{NEON.DOC.014042}:
TOS Protocol and Procedure: Plant Diversity Sampling; and science design
\href{http://data.neonscience.org/api/v0/documents/NEON.DOC.000912vA}{NEON.DOC.000912}:
TOS Science Design for Plant Diversity.

\begin{center}\rule{0.5\linewidth}{\linethickness}\end{center}

\textbf{Usage Notes}

Queries for this data will return data from all dates for
div\_voucher\_pub (which may have been collected many years before a
plant presence and absence or genetic material collection event),
whereas div\_1m2Data\_pub, div\_10m2Data100m2Data\_pub, and
div\_geneticarchive\_pub files will be subset to the data collected
during the date range specified. The protocol dictates that each target
plot is sampled once or twice annually (\textgreater{}1 expected record
per subplotID per plotID per year in div\_1m2Data\_pub and
div\_10m2Data100m2Data\_pub). A record in div\_1m2Data\_pub or
div\_10m2Data100m2Data\_pub may have zero (if genetic material or
vouchers not collected from that plot) or more corresponding plotID
records in div\_geneticarchive\_pub and div\_voucher\_pub depending on
date range of the data downloaded, and records in
div\_geneticarchive\_pub div\_voucher\_pub may have zero (if genetic
material or voucher not collected in a plot) or one corresponding record
in div\_geneticarchive\_pub and/or div\_voucher\_pub. The protocol
dictates observations of species presence in nested plots. A record
(species present) from div\_1m2Data\_pub is -- based on the protocol --
also present in corresponding subplots in div\_10m2Data100m2Data\_pub
but the records are not always populated in div\_10m2Data100m2Data\_pub
(a species in subplot 31.1.1 from file div\_1m2Data\_pub is also in
31.1.10 and 31 from file div\_10m2Data100m2Data\_pub but is not provided
in the data). Records where plotID is equivalent in div\_1m2Data\_pub
and div\_10m2Data100m2Data\_pub reflect total species presence at the
400m2 plot scale. Duplicates may exist where protocol and/or data entry
abberations have occurred; users should check data carefully for
anomalies before joining tables. Taxonomic IDs of species of concern
have been `fuzzed'; see data package readme files for more information.
\newpage
.

\section{DP1.10064.001 Rodent-borne pathogen
status}\label{dp1.10064.001-rodent-borne-pathogen-status}

\begin{center}\rule{0.5\linewidth}{\linethickness}\end{center}

\textbf{Subsystem}

Terrestrial Observation System (TOS)

\begin{center}\rule{0.5\linewidth}{\linethickness}\end{center}

\textbf{Coverage}

These data are collected at all NEON terrestrial sites, except for PUUM
(Puu Makaala Natural Area Reserve) and YELL (Yellowstone National Park).

\begin{center}\rule{0.5\linewidth}{\linethickness}\end{center}

\textbf{Description}

Presence/absence of a pathogen (or antibodies to a pathogen) in each
single rodent sample

\begin{center}\rule{0.5\linewidth}{\linethickness}\end{center}

\textbf{Abstract}

This data product contains the quality-controlled, native sampling
resolution data from NEON's testing of blood samples from individual
small mammals for seroposivity to Hantaviruses. The blood samples are
collected as part of the mark-recapture, box trapping effort for small
mammals (i.e., rodents (Rodentia) \textless{} 600 grams), with the field
capture results available separately via the Small mammal box trapping
data product (NEON.DP1.10072). Small mammals are sampled at regular
intervals by NEON field technicians at core and relocatable sites. Blood
samples are collected from individuals of target species of rodents,
including all species in the families Cricetidae, Muridae, and
Dipodidae, if an individual weighs more than 10 grams and is in good
physical condition. In a typical year, only a subset of blood samples
collected will be tested, up to approximately 140 samples per NEON site,
with the remaining samples archived. The sample identifiers for the
blood samples allow users to link the pathogen test results provided in
this data product to the field collection information, included
taxonomic identification of the capture, to the Small mammal box
trapping data product (NEON.DP1.10072). For additional details, see
protocol
\href{http://data.neonscience.org/api/v0/documents/NEON.DOC.000481vJ}{NEON.DOC.000481}:
TOS Protocol and Procedure: Small Mammal Sampling and science design
\href{http://data.neonscience.org/api/v0/documents/NEON.DOC.000911vA}{NEON.DOC.000911}:
TOS Science Design for Vectors and Pathogens.

\begin{center}\rule{0.5\linewidth}{\linethickness}\end{center}

\textbf{Usage Notes}

Queries for this data product will return data collected during the date
range specified. Testing data are provided in the rpt\_bloodtesting
table in this product. Users are advised to download the Small mammal
box trapping data product (DP1.10072) from the same date/location
combinations to access relevant information about the small mammals from
which these blood samples were taken (mam\_pertrapnight table) and
ancillary information about the trapping events (mam\_perplotnight
table). Data from the rpt\_bloodtesting table can be joined to the
mam\_pertrapnight table provided in the Small mammal box trapping data
product (DP1.10072) by the bloodSampleID field. Duplicates may exist
where protocol and/or data entry abberations have occurred; users should
check data carefully for anomalies before joining tables. \newpage
.

\section{DP1.10066.001 Root sampling
(Megapit)}\label{dp1.10066.001-root-sampling-megapit}

\begin{center}\rule{0.5\linewidth}{\linethickness}\end{center}

\textbf{Subsystem}

Terrestrial Observation System (TOS)

\begin{center}\rule{0.5\linewidth}{\linethickness}\end{center}

\textbf{Coverage}

These data are collected at all NEON terrestrial sites.

\begin{center}\rule{0.5\linewidth}{\linethickness}\end{center}

\textbf{Description}

Fine root biomass in 10cm increments (first 1m depth) and 20cm
increments (from 1m to 2m depth) from soil pit sampling

\begin{center}\rule{0.5\linewidth}{\linethickness}\end{center}

\textbf{Abstract}

This data product contains the quality-controlled, native sampling
resolution data from NEON's belowground plant biomass sampled from the
megapit. Soil samples were collected at 10 cm depth increments to the
first 100 cm below the surface, then 20 cm depth increments thereafter,
along three vertical profiles from a single temporary soil pit at a
location expected to be representative of NEON sensor-based soil plots.
This sampling activity is expected to occur once at each NEON
terrestrial site. Additional belowground plant biomass samples collected
from the same soil pit are archived in the NEON Megapit Soil Archive and
are available upon request. For additional details on the sampling
protocol, see
\href{http://data.neonscience.org/api/v0/documents/NEON.DOC.001708vA}{NEON.DOC.001708}:
TOS Protocol and Procedure: Soil Pit Sampling for Plant Belowground
Biomass. Products resulting from this sampling include root biomass by
size class and status.

\begin{center}\rule{0.5\linewidth}{\linethickness}\end{center}

\textbf{Usage Notes}

Queries for this data product will return data collected during the date
range specified. The protocol dictates that up to 12 records per soil
pit are created in the mpr\_perPitProfile table, corresponding to 3
vertical profiles, 2 root size classes, and 2 root statuses. Each
profile (up to 4 records) in mpr\_perPitProfile corresponds to up to 15
child records in mpr\_perDepthIncrement, one for each of up to 15 depth
increments per vertical profile. Each record in mpr\_perPitProfile
creates up to 15 child records in mpr\_perRootSample, one record per
profile per depth increment per root size class per root status.
\newpage
.

\section{DP1.10067.001 Root sampling tower
plots}\label{dp1.10067.001-root-sampling-tower-plots}

\begin{center}\rule{0.5\linewidth}{\linethickness}\end{center}

\textbf{Subsystem}

Terrestrial Observation System (TOS)

\begin{center}\rule{0.5\linewidth}{\linethickness}\end{center}

\textbf{Coverage}

These data are collected at all NEON terrestrial sites.

\begin{center}\rule{0.5\linewidth}{\linethickness}\end{center}

\textbf{Description}

Fine root biomass to 30cm depth via soil core sampling

\begin{center}\rule{0.5\linewidth}{\linethickness}\end{center}

\textbf{Abstract}

This data product contains the quality-controlled, native sampling
resolution data from NEON's belowground biomass protocol. These data
enable estimation of the amount of belowground plant biomass
\textless{}= 10 mm diameter within the area surrounding the NEON eddy
covariance tower. At many sites this will also be the dominant
vegetation type(s). NEON uses a 3-inch outside diameter (6.65 cm inside
diameter) soil corer for belowground biomass sampling, with samples
cored to 30 cm depth in order to be consistent with the sampling depth
used for soil biogeochemistry and microbe sampling. Roots are sorted to
two status classes (live or dead) and the following size category bins:
\textless{} 0.5 mm, 0.5-1 mm, 1-2 mm, and 2-10 mm. Additional
belowground plant biomass samples collected from the same soil cores are
archived in the NEON Soil Archive and are available upon request. For
additional details on the sampling protocol, see
\href{http://data.neonscience.org/api/v0/documents/NEON.DOC.014038vE}{NEON.DOC.014038}:
TOS Protocol and Procedure: Core Sampling for Plant Belowground Biomass
(AD{[}06{]}). Products resulting from this sampling include root biomass
by size class and status.

\begin{center}\rule{0.5\linewidth}{\linethickness}\end{center}

\textbf{Usage Notes}

Queries for this data product will return all data collected during the
date range specified. The bbc\_percore table includes two records
(sampleIDs) per clipID. The bbc\_rootmass table includes up to eight (4
size classes x 2 status classes) records (subsampleIDs) per sampleID.
The bbc\_dilution table inclues zero or one record (dilutionSubsampleID)
per subsampleID. The bbc\_chemistryPooling table includes up to four
(pooled live size classes) records (cnSampleIDs) per clipID. \newpage
.

\section{DP1.10072.001 Small mammal box
trapping}\label{dp1.10072.001-small-mammal-box-trapping}

\begin{center}\rule{0.5\linewidth}{\linethickness}\end{center}

\textbf{Subsystem}

Terrestrial Observation System (TOS)

\begin{center}\rule{0.5\linewidth}{\linethickness}\end{center}

\textbf{Coverage}

These data are collected at all NEON terrestrial sites, except the site
in Hawaii.

\begin{center}\rule{0.5\linewidth}{\linethickness}\end{center}

\textbf{Description}

Individual- and trap-level data collected using box traps designed to
capture small mammals

\begin{center}\rule{0.5\linewidth}{\linethickness}\end{center}

\textbf{Abstract}

This data product contains the quality-controlled, native sampling
resolution data from NEON's small mammal sampling protocol. Small mammal
abundance and diversity are sampled at regular intervals by NEON field
technicians at core and relocatable sites. Here small mammals are
defined based on a combination of behavioral, dietary, and size
constraints, as the NEON design is limited to species sampled by box
traps. This definition includes any mammal that is (1) nonvolant; (2)
nocturnally active; (3) forages predominantly aboveground; and (4) is
greater than 5 grams but less than approximately 500-600 g. In North
America, this includes cricetids, heteromyids, small sciurids, and
introduced murids. It does not include shrews, large squirrels, rabbits,
or weasels, despite the fact that individuals of these species may be
incidentally captured. Products resulting from this sampling include the
species identification and unique identifier for each individual
captured, as well as a suite of standard size measurements and
reproductive condition data. Sample identifiers for any blood, ear,
hair, whisker, fecal, and/or voucher samples collected are also
provided.

For additional details, see protocol
\href{http://data.neonscience.org/api/v0/documents/NEON.DOC.000481vJ}{NEON.DOC.000481}:
TOS Protocol and Procedure: Small Mammal Sampling and science design
\href{http://data.neonscience.org/api/v0/documents/NEON.DOC.000915vA}{NEON.DOC.000914}:
TOS Science Design for Small Mammal Abundance and Diversity. For spatial
data (text and shapefiles), download
\href{http://data.neonscience.org/api/v0/documents/NEON_TOS_Plots_V4}{NEON\_TOS\_Plots}.

\begin{center}\rule{0.5\linewidth}{\linethickness}\end{center}

\textbf{Usage Notes}

Queries for this data product will return data collected during the date
range specified. Per trapping grid data are provided in the table,
mam\_perplotnight, with associated per trap per night data provided in
the mam\_pertrapnight table. The tables can be joined by the nightuid
field. Duplicates may exist where protocol and/or data entry abberations
have occurred; users should check data carefully for anomalies before
joining tables. Taxonomic IDs of species of concern have been `fuzzed';
see data package readme files for more information. \newpage
.

\section{DP1.10076.001 Small mammal sequences DNA
barcode}\label{dp1.10076.001-small-mammal-sequences-dna-barcode}

\begin{center}\rule{0.5\linewidth}{\linethickness}\end{center}

\textbf{Subsystem}

Terrestrial Observation System (TOS)

\begin{center}\rule{0.5\linewidth}{\linethickness}\end{center}

\textbf{Coverage}

These data are collected at NEON terrestrial sites.

\begin{center}\rule{0.5\linewidth}{\linethickness}\end{center}

\textbf{Description}

CO1 DNA sequences from select small mammals

\begin{center}\rule{0.5\linewidth}{\linethickness}\end{center}

\textbf{Abstract}

This data product contains the quality-controlled laboratory metadata
and QA results for NEON's cytochrome oxidase I (COI) barcoding of small
mammal sequences. The DNA barcoding procedure involves the
non-distructive collection of tissue from live specimens in the field,
extracting and sequencing DNA from the tissue, and matching that
sequence data to sequences from previously identified voucher specimens.
DNA analysis serves a number of purposes, including verification of
taxonomy of specimens that do not receive expert identification,
clarification of the taxonomy of rare or cryptic species, and
characterization of diversity using molecular markers. For additional
details on small mammal collection, see protocol
\href{http://data.neonscience.org/api/v0/documents/NEON.DOC.000481vJ}{NEON.DOC.000481}:
TOS Protocol and Procedure: Small Mammal Sampling and science design
\href{http://data.neonscience.org/api/v0/documents/NEON.DOC.NEON.DOC.000915vA}{NEON.DOC.000915}:
TOS Science Design for Small Mammal Abundance and Diversity. Queries for
this data product will return metadata tables formatted for submission
to the Barcode of Life Database. These queries will also provide links
to the actual sequence data, which are publicly available on the Barcode
of Life Datasystem (BOLD, \url{http://www.barcodinglife.com/}). The
sequence data can be obtained by following the links from the NEON data
portal, or by directly querying NEON data sets on the BOLD server. From
the NEON portal, the link ``BOLD Project: Small Mammal sequences DNA
barcode'' redirects to a page on the BOLD public data portal for the
queried data. This is a dynamic link and will automatically update based
on the user query.

\begin{center}\rule{0.5\linewidth}{\linethickness}\end{center}

\textbf{Usage Notes}

Taxonomic IDs of species of concern have been `fuzzed'; see data package
readme files for more information. \newpage
.

\section{DP1.10078.001 Soil chemical properties (Distributed
periodic)}\label{dp1.10078.001-soil-chemical-properties-distributed-periodic}

\begin{center}\rule{0.5\linewidth}{\linethickness}\end{center}

\textbf{Subsystem}

Terrestrial Observation System (TOS)

\begin{center}\rule{0.5\linewidth}{\linethickness}\end{center}

\textbf{Coverage}

These data are collected at all NEON terrestrial sites with sufficient
soil depths to enable sampling.

\begin{center}\rule{0.5\linewidth}{\linethickness}\end{center}

\textbf{Description}

Soil chemistry from the top 30 cm of the profile from periodic soil core
collections. Data are reported by horizon (mineral vs.~organic). See
initial characterization and megapit products for additional soil data.

\begin{center}\rule{0.5\linewidth}{\linethickness}\end{center}

\textbf{Abstract}

Total organic carbon and total nitrogen concentrations in surface soils
sampled from NEON plots. Soils are sampled by horizon type (organic or
mineral) to a maximum depth of 30 cm. For additional details, see
protocol
\href{http://data.neonscience.org/api/v0/documents/NEON.DOC.014048vG}{NEON.DOC.014048}:
TOS Protocol and Procedure for Soil Biogeochemical and Microbial
Sampling and science design
\href{http://data.neonscience.org/api/v0/documents/NEON.DOC.000906vA}{NEON.DOC.000906}:
TOS Science Design for Terrestrial Biogeochemistry. Queries for this
data product will return elemental data only. For stable isotope data
measured concurrently with concentrations, see Soil stable isotopes
(Distributed periodic) data product. For field metadata associated with
these samples, see Soil physical properties (Distributed periodic) data
product.

\begin{center}\rule{0.5\linewidth}{\linethickness}\end{center}

\textbf{Usage Notes}

A cnSampleID may appear from one to four times in sls\_soilChemistry,
depending on whether analytical replicates were conducted and whether
the sample was acidified to remove carbonate prior to analysis. There
should only be one instance per cnSampleID x analyticalRepNumber x
acidTreatment combination, yet duplicates may exist where protocol
and/or data entry aberrations have occurred. Users should check data
carefully for anomalies before analyzing data. \newpage
.

\section{DP1.10080.001 Soil inorganic nitrogen pools and
transformations}\label{dp1.10080.001-soil-inorganic-nitrogen-pools-and-transformations}

\begin{center}\rule{0.5\linewidth}{\linethickness}\end{center}

\textbf{Subsystem}

Terrestrial Observation System (TOS)

\begin{center}\rule{0.5\linewidth}{\linethickness}\end{center}

\textbf{Coverage}

These data are collected at all NEON terrestrial sites.

\begin{center}\rule{0.5\linewidth}{\linethickness}\end{center}

\textbf{Description}

Soil inorganic nitrogen concentrations and rates of net nitrogen
mineralization and net nitrification from the top 30 cm of the profile.
Data are reported by horizon (organic vs.~mineral) within a soil core.

\begin{center}\rule{0.5\linewidth}{\linethickness}\end{center}

\textbf{Abstract}

This data product contains the quality-controlled, native sampling
resolution data from NEON's measurement of soil inorganic nitrogen (N)
pools and net N transformation rates. Soils are sampled by horizon type
(organic or mineral) to a maximum depth of 30 cm. For additional
details, see protocol
\href{http://data.neonscience.org/api/v0/documents/NEON.DOC.014048vJ}{NEON.DOC.014048:
TOS Protocol and Procedure for Soil Biogeochemical and Microbial
Sampling}. Queries for this data product will return laboratory metadata
and results of chemical analyses. For field metadata associated with
these samples, see the Soil physical properties (Distributed periodic)
data product.

\begin{center}\rule{0.5\linewidth}{\linethickness}\end{center}

\textbf{Usage Notes}

Queries for this data product will return data collected during the date
range specified. Each soil sampling location yields a unique sampleID
per horizon per collectDate (day of year, local time). For paired
initial and final cores, sampleIDs will differ only by collectDate. Each
sample is then subsampled and extracted for inorganic N, yielding a
corresponding kclSampleID. Thus, a record from sls\_soilCoreCollection
in Soil physical properties (Distributed periodic) may have zero or one
child records in the ntr\_internalLab table in Soil inorganic nitrogen
pools and transformations. The information needed to link procedural
blanks with samples is provided in ntr\_externalLabBlanks. Each child
record may appear from zero to two times in the ntr\_externalLab table.
Most will appear once, but some may appear twice if analytical
replicates were conducted. Duplicates and/or missing data may exist
where protocol and/or data entry abberations have occurred; users should
check data carefully for anomalies before joining tables. \newpage
.

\section{DP1.10081.001 Soil microbe community
composition}\label{dp1.10081.001-soil-microbe-community-composition}

\begin{center}\rule{0.5\linewidth}{\linethickness}\end{center}

\textbf{Subsystem}

Terrestrial Observation System (TOS)

\begin{center}\rule{0.5\linewidth}{\linethickness}\end{center}

\textbf{Coverage}

These data are collected at all NEON terrestrial sites.

\begin{center}\rule{0.5\linewidth}{\linethickness}\end{center}

\textbf{Description}

Counts and relative abundances of archaeal, bacterial, and fungal taxa
observed in soil microbial communities

\begin{center}\rule{0.5\linewidth}{\linethickness}\end{center}

\textbf{Abstract}

This data product contains the quality-controlled laboratory data and
metadata for NEON's soil bacterial, archaeal, and fungal community
composition analysis derived from soil microbial sampling. Taxon tables
are derived from the 16S and ITS marker gene sequencing data product,
NEON.DP1.10108. Taxonomic data are generated for sequence data using
standard bioinformatics software. For additional details about sampling
methods and design, see
\href{http://data.neonscience.org/api/v0/documents/NEON.DOC.014048vK}{NEON.DOC.014048}:
TOS Protocol and Procedure for Soil Biogeochemical and Microbial
Sampling; and science design
\href{http://data.neonscience.org/api/v0/documents/NEON.DOC.000908vA}{NEON.DOC.000908vA}:
TOS Science Design for Terrestrial Microbial Diversity.

\begin{center}\rule{0.5\linewidth}{\linethickness}\end{center}

\textbf{Usage Notes}

Queries for the basic download data product will return data from
mcc\_soilTaxonTableMetadata\_16S and mcc\_soilTaxonTableMetadata\_ITS
for all dates within the specified date range. In addition, the expanded
download also includes per-sample taxon tables for all dates within the
specified date range and data from all dates for the
mcc\_taxonTableLabSummary. A given dnaSampleID for
mcc\_soilTaxonTableMetadata\_16S(ITS) is expected to generate one record
for each completeTaxonomy, and the number of records per sample should
equal the number of unique values for completeTaxonomy. Duplicate
samples and/or missing data may exist where protocol and/or data entry
aberrations have occurred; users should check data carefully for
anomalies before joining tables. \newpage
.

\section{DP1.10086.001 Soil physical properties (Distributed
periodic)}\label{dp1.10086.001-soil-physical-properties-distributed-periodic}

\begin{center}\rule{0.5\linewidth}{\linethickness}\end{center}

\textbf{Subsystem}

Terrestrial Observation System (TOS)

\begin{center}\rule{0.5\linewidth}{\linethickness}\end{center}

\textbf{Coverage}

These data are collected at all NEON terrestrial sites with sufficient
soil depths to enable sampling.

\begin{center}\rule{0.5\linewidth}{\linethickness}\end{center}

\textbf{Description}

Soil physical properties from the top 30 cm of the profile from periodic
soil core collections. Data are reported by horizon (mineral
vs.~organic). See initial characterization and megapit products for
additional soil data.

\begin{center}\rule{0.5\linewidth}{\linethickness}\end{center}

\textbf{Abstract}

This data product contains the quality-controlled, native sampling
resolution data from NEON's soil sampling, pH, and moisture
measurements. Samples collected as part of this product are also used
for microbial and biogeochemical measurements; those data can be found
in associated data products.

Soil is defined as the upper layer of the earth's surface where plants
grow and consists of decomposing organic material and inorganic
particles such as clay and rock. Soils are sampled by horizon type
(organic or mineral) to a maximum depth of 30cm. For additional details,
see
\href{http://data.neonscience.org/api/v0/documents/NEON.DOC.014048vG}{NEON.DOC.014048}:
TOS Protocol and Procedure for Soil Biogeochemical and Microbial
Sampling;
\href{http://data.neonscience.org/api/v0/documents/NEON.DOC.000906vA}{NEON.DOC.000906}:
TOS Science Design for Terrestrial Biogeochemistry; and
\href{http://data.neonscience.org/api/v0/documents/NEON.DOC.000908vB}{NEON.DOC.000908}:
TOS Science Design for Terrestrial Microbial Diversity. Queries for this
data product will return field collection, pH, and moisture data with
collection dates within the months of the specified date range. This
data product provides primary field and laboratory metadata that can be
associated with soil microbial data products, soil chemistry and stable
isotope data products, and soil nitrogen transformations data products.
See related data products section for the complete list.

\begin{center}\rule{0.5\linewidth}{\linethickness}\end{center}

\textbf{Usage Notes}

A record from sls\_soilCoreCollection may have zero or one child records
in sls\_soilpH and sls\_soilMoisture; a given
sls\_soilCoreCollection.sampleID is expected to be sampled one time per
collectDate (local time). Depending on the type of bout and time of
year, a record from sls\_soilCoreCollection may have zero or one child
records in sls\_metagenomicsPooling and in sls\_bgcSubsampling.
Duplicates may exist where protocol and/or data entry abberations have
occurred; users should check data carefully for anomalies before joining
tables. \newpage
.

\section{DP1.10092.001 Tick-borne pathogen
status}\label{dp1.10092.001-tick-borne-pathogen-status}

\begin{center}\rule{0.5\linewidth}{\linethickness}\end{center}

\textbf{Subsystem}

Terrestrial Observation System (TOS)

\begin{center}\rule{0.5\linewidth}{\linethickness}\end{center}

\textbf{Coverage}

These data are collected at NEON terrestrial sites.

\begin{center}\rule{0.5\linewidth}{\linethickness}\end{center}

\textbf{Description}

Presence/absence of a pathogen in each single tick sample

\begin{center}\rule{0.5\linewidth}{\linethickness}\end{center}

\textbf{Abstract}

This data product contains the quality-controlled, native sampling
resolution data derived from NEON's tick pathogen testing. Products
resulting from this sampling include results of testing individual ticks
collected during NEON tick sampling for pathogen presence/absence. See
NEON Product Ticks sampled using drag cloths (DP1.10093.001) for data on
the abundance and diversity of ticks collected at NEON sites. Following
collection, tick samples are sent to a professional taxonomist where
ticks are identified to species and sex. A subset of
postively-identified nymphal ticks are tested for the presence of viral
and protozoan pathogens. For additional details see science design
\href{http://data.neonscience.org/api/v0/documents/NEON.DOC.000911vA}{NEON.DOC.000911}:
TOS Science Design for Vectors and Pathogens.

\begin{center}\rule{0.5\linewidth}{\linethickness}\end{center}

\textbf{Usage Notes}

Queries for this data product will return data from tck\_pathogen subset
to data collected during the date range specified, but when the expanded
package is requested, all batches from the relevant laboratory will be
included in the tck\_pathogenqa file. The protocol dictates that each
testingID is tested once per pathogen (one expected record per testingID
per testPathogenName in tck\_pathogen). An record from tck\_pathogenqa
may have zero or more related records in tck\_pathogen, depending on the
date range of the data downloaded. Duplicates may exist where protocol
and/or data entry abberations have occurred; users should check data
carefully for anomalies before joining tables. \newpage
.

\section{DP1.10093.001 Ticks sampled using drag
cloths}\label{dp1.10093.001-ticks-sampled-using-drag-cloths}

\begin{center}\rule{0.5\linewidth}{\linethickness}\end{center}

\textbf{Subsystem}

Terrestrial Observation System (TOS)

\begin{center}\rule{0.5\linewidth}{\linethickness}\end{center}

\textbf{Coverage}

These data are collected at NEON terrestrial sites.

\begin{center}\rule{0.5\linewidth}{\linethickness}\end{center}

\textbf{Description}

Abundance and density of ticks collected by drag and/or flag sampling
(by species and/or lifestage)

\begin{center}\rule{0.5\linewidth}{\linethickness}\end{center}

\textbf{Abstract}

This data product contains the quality-controlled, native sampling
resolution data from Tick and Tick-Borne Pathogen Sampling protocol.
Tick abundance and diversity are sampled at regular intervals by NEON
field technicians at core and relocatable sites using drag or flag
sampling techniques. For additional details on protocol, see the TOS
Protocol and Procedure: Tick and Tick-Borne Pathogen Sampling. Following
collection, samples are sent to a professional taxonomist where ticks
are identified to species and lifestage and/or sex. Identified ticks are
then processed for pathogen analysis or preserved for final archiving.
Products resulting from this sampling and processing include records of
when ticks were sampled and the taxonomic and abundance data of ticks
captured. For additional details, see protocol
\href{http://data.neonscience.org/api/v0/documents/NEON.DOC.014045vH}{NEON.DOC.014045}:
TOS Protocol and Procedure: Tick and Tick-Borne Pathogen Sampling and
science design
\href{http://data.neonscience.org/api/v0/documents/NEON.DOC.000911vA}{NEON.DOC.000911}:
TOS Science Design for Vectors and Pathogens.

\begin{center}\rule{0.5\linewidth}{\linethickness}\end{center}

\textbf{Usage Notes}

Queries for this data product will return data from tck\_fielddata and
tckTaxonomy subset to data collected during the date range specified.
The protocol dictates that each of 6 plotIDs per site is sampled during
each eventID (six expected record per eventID). An record from
tck\_fielddata may have zero or more related records in tck\_taxonomy,
depending on whether ticks were present in the sample. Duplicates may
exist where protocol and/or data entry abberations have occurred; users
should check data carefully for anomalies before joining tables.
\newpage
.

\section{DP1.10098.001 Woody plant vegetation
structure}\label{dp1.10098.001-woody-plant-vegetation-structure}

\begin{center}\rule{0.5\linewidth}{\linethickness}\end{center}

\textbf{Subsystem}

Terrestrial Observation System (TOS)

\begin{center}\rule{0.5\linewidth}{\linethickness}\end{center}

\textbf{Coverage}

These data are collected at all NEON terrestrial sites at which
qualifying smaller woody individuals (individuals with DBH \textless{}
10 cm) are present at 10\% cover or greater, or when larger individuals
(individuals with DBH ≥ 10 cm) are present in 10\% or more of designated
plots . Functionally, sampling occurs at forested sites, and sites with
shrub/scrub vegetation.

\begin{center}\rule{0.5\linewidth}{\linethickness}\end{center}

\textbf{Description}

Structure measurements, including height, canopy diameter, and stem
diameter, as well as mapped position of individual woody plants

\begin{center}\rule{0.5\linewidth}{\linethickness}\end{center}

\textbf{Abstract}

This data product contains the quality-controlled, native sampling
resolution data from in-situ measurements of live and standing dead
woody individuals and shrub groups, from all terrestrial NEON sites with
qualifying woody vegetation. The exact measurements collected per
individual depend on growth form, and these measurements are focused on
enabling biomass and productivity estimation, estimation of shrub volume
and biomass, and calibration / validation of multiple NEON airborne
remote-sensing data products. In general, comparatively large
individuals that are visible to remote-sensing instruments are mapped,
tagged and measured, and other smaller individuals are tagged and
measured but not mapped. Smaller individuals may be subsampled according
to a nested subplot approach in order to standardize the per plot
sampling effort. Structure and mapping data are reported per individual
per plot; sampling metadata, such as per growth form sampling area, are
reported per plot. For additional details, see protocol
\href{http://data.neonscience.org/api/v0/documents/NEON.DOC.000987vG}{NEON.DOC.000987vG}:
TOS Protocol and Procedure: Measurement of Vegetation Structure, and
Science Design
\href{http://data.neonscience.org/api/v0/documents/NEON.DOC.000914vA}{NEON.DOC.000914}:
TOS Science Design for Plant Biomass, Productivity and Leaf Area Index.

\begin{center}\rule{0.5\linewidth}{\linethickness}\end{center}

\textbf{Usage Notes}

Queries for this data product will return data from all dates for
vst\_mappingandtagging (since individuals maybe tagged and mapped many
years before a given vegetation structure sampling bout), whereas the
vst\_perplotperyear, vst\_apparentindividual and vst\_shrubgroup tables
will be subset to data collected during the date range specified. Data
are provided in monthly download files; queries including any part of a
month will return data from the entire month. In the vst\_perplotperyear
table, there should be one record per plotID per eventID, and data in
this table describe the presence/absence of woody growth forms, as well
as the sampling area utilized for each growth form. The
vst\_mappingandtagging table contains at least one record per
individualID, and provides data that are invariant through time,
including tagID, taxonID and mapped location (if applicable). Duplicates
in vst\_mappingandtagging may exist at the individualID level if errors
have been corrected after ingest of the original record; in this
instance, users are advised to use the most recent record. Records in
vst\_mappingandtagging may be linked to vst\_perplotperyear via the
plotID and eventID fields. The vst\_apparentindividual table contains
one record per individualID per eventID, and includes growth form,
structure and status data that may be linked to vst\_mappingandtagging
records via individualID; records may also be linked to
vst\_perplotperyear via the plotID and eventID fields. For allometric
measurements on tree palms and other large, nonwoody individuals, users
must download the nst\_perindividual table from the related
Non-herbaceous perennial vegetation structure data product
(NEON.DP1.10045), and join on the individualID variable. The
vst\_shrubgroup table contains a minimum of one record per groupID per
plotID per eventID; multiple records with the same groupID may exist if
a given shrub group is comprised of more than one taxonID. Data provided
in the vst\_shrubgroup table allow calculation of live and dead volume
per taxonID within each shrub group, and records may be linked with
vst\_perplotperyear via the plotID and eventID fields.

For all tables, duplicates may exist where protocol and/or data entry
aberrations have occurred; users should check data carefully for
anomalies before joining tables. Taxonomic IDs of species of concern
have been `fuzzed'; see data package readme files for more information.
\newpage
.

\section{DP1.10099.001 Root stable
isotopes}\label{dp1.10099.001-root-stable-isotopes}

\begin{center}\rule{0.5\linewidth}{\linethickness}\end{center}

\textbf{Subsystem}

Terrestrial Observation System (TOS)

\begin{center}\rule{0.5\linewidth}{\linethickness}\end{center}

\textbf{Coverage}

These data are collected at all NEON terrestrial sites.

\begin{center}\rule{0.5\linewidth}{\linethickness}\end{center}

\textbf{Description}

Fine root stable isotope values from cores or pits. Data are reported by
size class.

\begin{center}\rule{0.5\linewidth}{\linethickness}\end{center}

\textbf{Abstract}

This data product contains the quality-controlled, native sampling
resolution data from NEON's measurement of stable isotope values of
carbon and nitrogen in root biomass. Samples are either collected every
five years using surface soil cores (top 30 cm) taken from each Tower
base plot, or once from a single deep (2 meter) pit in the vicinity of
the NEON Tower and soil sensors in 10-20 cm depth increments. For
surface soil core sampling, roots are sieved, sorted into four size
categories, and only live roots are analyzed for chemistry. For soil pit
sampling, roots are sieved, sorted into two size categories, and both
live and dead roots are analyzed. If sufficient mass is present,
additional root material is archived and available upon request. For
additional details on the sampling protocols, see
\href{http://data.neonscience.org/api/v0/documents/NEON.DOC.014038vE}{NEON.DOC.014038}:
TOS Protocol and Procedure: Core Sampling for Plant Belowground Biomass
and
\href{http://data.neonscience.org/api/v0/documents/NEON.DOC.001708vA}{NEON.DOC.001708}:
TOS Protocol and Procedure: Soil Pit Sampling for Plant Belowground
Biomass.

\begin{center}\rule{0.5\linewidth}{\linethickness}\end{center}

\textbf{Usage Notes}

Queries for this data product will return data collected during the date
range specified. A cnSampleID, the unique identifier for a root sample,
may appear one to two times in bbc\_rootStableIsotopes, depending on
whether an analytical replicate was run. There should only be one
instance per cnSampleID x analyticalRepNumber combination, yet
duplicates may exist where protocol and/or data entry abberations have
occurred. Users should check data carefully for anomalies before
analyzing data. \newpage
.

\section{DP1.10100.001 Soil stable isotopes (Distributed
periodic)}\label{dp1.10100.001-soil-stable-isotopes-distributed-periodic}

\begin{center}\rule{0.5\linewidth}{\linethickness}\end{center}

\textbf{Subsystem}

Terrestrial Observation System (TOS)

\begin{center}\rule{0.5\linewidth}{\linethickness}\end{center}

\textbf{Coverage}

These data are collected at all NEON terrestrial sites with sufficient
soil depths to enable sampling.

\begin{center}\rule{0.5\linewidth}{\linethickness}\end{center}

\textbf{Description}

Soil stable isotope values from the top 30 cm of the profile from
periodic soil core collections. Data are reported by horizon (mineral
vs.~organic). See initial characterization and megapit products for
additional soil data.

\begin{center}\rule{0.5\linewidth}{\linethickness}\end{center}

\textbf{Abstract}

Stable isotope content of total organic carbon and total nitrogen pools
in surface soils sampled from NEON plots. Soils are sampled by horizon
type (organic or mineral) to a maximum depth of 30 cm. For additional
details, see protocol
\href{http://data.neonscience.org/api/v0/documents/NEON.DOC.014048vJ}{NEON.DOC.014048}:
TOS Protocol and Procedure for Soil Biogeochemical and Microbial
Sampling and science design
\href{http://data.neonscience.org/api/v0/documents/NEON.DOC.000906vA}{NEON.DOC.000906}:
TOS Science Design for Terrestrial Biogeochemistry. Queries for this
data product will return isotope data only. For concentration data
measured concurrently with stable isotopes, see Soil chemical properties
(Distributed periodic) data product. For field metadata associated with
these samples, see Soil physical properties (Distributed periodic) data
product.

\begin{center}\rule{0.5\linewidth}{\linethickness}\end{center}

\textbf{Usage Notes}

A cnSampleID may appear from one to four times in
sls\_soilStableIsotopes, depending on whether analytical replicates were
conducted and whether the sample was acidified to remove carbonate prior
to analysis. There should only be one instance per cnSampleID x
analyticalRepNumber x acidTreatment combination, yet duplicates may
exist where protocol and/or data entry abberations have occurred. Users
should check data carefully for anomalies before analyzing data.
\newpage
.

\section{DP1.10101.001 Litter stable
isotopes}\label{dp1.10101.001-litter-stable-isotopes}

\begin{center}\rule{0.5\linewidth}{\linethickness}\end{center}

\textbf{Subsystem}

Terrestrial Observation System (TOS)

\begin{center}\rule{0.5\linewidth}{\linethickness}\end{center}

\textbf{Coverage}

These data are collected at NEON terrestrial sites with overstory
vegetation.

\begin{center}\rule{0.5\linewidth}{\linethickness}\end{center}

\textbf{Description}

Bulk litter stable isotope values at the scale of a plot. Data are
reported by functional group (leaves vs.~needles).

\begin{center}\rule{0.5\linewidth}{\linethickness}\end{center}

\textbf{Abstract}

This data product contains the quality-controlled, native sampling
resolution data from NEON's measurement of stable isotopes in
litterfall. Litter is defined as material that is dropped from the
forest canopy and has a butt end diameter \textless{}2 cm and a length
\textless{}50 cm; this material is collected in elevated 0.5 m2 PVC
traps. After sorting by functional group, needles and leaves are sent
for isotope analysis. For additional details, see protocol
\href{http://data.neonscience.org/api/v0/documents/NEON.DOC.001710vE}{NEON.DOC.001710}:
TOS Field and Lab Protocol for Litterfall and Fine Woody Debris.

\begin{center}\rule{0.5\linewidth}{\linethickness}\end{center}

\textbf{Usage Notes}

Queries for this data product will return data collected during the date
range specified. A cnSampleID, the unique identifier for an analytical
litter sample, may appear one to two times in ltr\_litterStableIsotopes,
depending on whether an analytical replicate was run. There should only
be one instance per cnSampleID x analyticalRepNumber combination, but
duplicates may exist where protocol and/or data entry abberrations have
occurred. Users should check data carefully for anomalies before
analyzing data. \newpage
.

\section{DP1.10102.001 Root chemical
properties}\label{dp1.10102.001-root-chemical-properties}

\begin{center}\rule{0.5\linewidth}{\linethickness}\end{center}

\textbf{Subsystem}

Terrestrial Observation System (TOS)

\begin{center}\rule{0.5\linewidth}{\linethickness}\end{center}

\textbf{Coverage}

These data are collected at all NEON terrestrial sites.

\begin{center}\rule{0.5\linewidth}{\linethickness}\end{center}

\textbf{Description}

Fine root chemistry from cores or pits. Data are reported by size class.

\begin{center}\rule{0.5\linewidth}{\linethickness}\end{center}

\textbf{Abstract}

This data product contains the quality-controlled, native sampling
resolution data from NEON's measurement of carbon and nitrogen
concentrations in root biomass. Samples are either collected every five
years using surface soil cores (top 30 cm) taken from each Tower base
plot, or once from a single deep (2 meter) pit in the vicinity of the
NEON Tower and soil sensors in 10-20 cm depth increments. For surface
soil core sampling, roots are sieved, sorted into four size categories,
and only live roots are analyzed for chemistry. For soil pit sampling,
roots are sieved, sorted into two size categories, and both live and
dead roots are analyzed. If sufficient mass is present, additional root
material is archived and available upon request. For additional details
on the sampling protocols, see
\href{http://data.neonscience.org/api/v0/documents/NEON.DOC.014038vE}{NEON.DOC.014038}:
TOS Protocol and Procedure: Core Sampling for Plant Belowground Biomass
and
\href{http://data.neonscience.org/api/v0/documents/NEON.DOC.001708vA}{NEON.DOC.001708}:
TOS Protocol and Procedure: Soil Pit Sampling for Plant Belowground
Biomass.

\begin{center}\rule{0.5\linewidth}{\linethickness}\end{center}

\textbf{Usage Notes}

Queries for this data product will return data collected during the date
range specified. A cnSampleID, the unique identifier for a root sample,
may appear one to two times in bbc\_rootChemistry, depending on whether
an analytical replicate was run. There should only be one instance per
cnSampleID x analyticalRepNumber combination, yet duplicates may exist
where protocol and/or data entry abberrations have occurred. Users
should check data carefully for anomalies before analyzing data.
\newpage
.

\section{DP1.10104.001 Soil microbe
biomass}\label{dp1.10104.001-soil-microbe-biomass}

\begin{center}\rule{0.5\linewidth}{\linethickness}\end{center}

\textbf{Subsystem}

Terrestrial Observation System (TOS)

\begin{center}\rule{0.5\linewidth}{\linethickness}\end{center}

\textbf{Coverage}

These data are collected at all NEON terrestrial sites.

\begin{center}\rule{0.5\linewidth}{\linethickness}\end{center}

\textbf{Description}

Quantitative abundance of total microbes in soil samples

\begin{center}\rule{0.5\linewidth}{\linethickness}\end{center}

\textbf{Abstract}

This data product contains the quality-controlled laboratory data and
metadata for microbial biomass derived from soil microbial sampling.
Microbial biomass is measured by phospholipid fatty acid (PLFA)
analysis, in which a set of microbial lipid biomarkers is extracted and
quantified using Gas Chromatography-Mass Spectrometry (GS-MS). For
additional details about sampling methods and design, see
\href{http://data.neonscience.org/api/v0/documents/NEON.DOC.014048vJ}{NEON.DOC.014048:
TOS Protocol and Procedure for Soil Biogeochemical and Microbial
Sampling}; and
\href{http://data.neonscience.org/api/v0/documents/NEON.DOC.000908vA}{NEON.DOC.000908vA:
TOS Science Design for Terrestrial Microbial Diversity}.

\begin{center}\rule{0.5\linewidth}{\linethickness}\end{center}

\textbf{Usage Notes}

Queries for the basic package of this data product will return data from
sme\_microbialBiomass for all dates within the specified date range.
Queries for the expanded package will also return data from all dates
for sme\_batchResults and sme\_labSummary. A given biomassID for
sme\_microbialBiomass is expected to generate one record. The
sme\_microbialBiomass table can be joined to other soil sampling data
via the sampleID field, which matches the sampleID in
sls\_soilCoreCollection, in the Soil physical properties (Distributed
periodic) data product. Duplicate samples and/or missing data may exist
where protocol and/or data entry aberrations have occurred; users should
check data carefully for anomalies before joining tables. \newpage
.

\section{DP1.10107.001 Soil microbe metagenome
sequences}\label{dp1.10107.001-soil-microbe-metagenome-sequences}

\begin{center}\rule{0.5\linewidth}{\linethickness}\end{center}

\textbf{Subsystem}

Terrestrial Observation System (TOS)

\begin{center}\rule{0.5\linewidth}{\linethickness}\end{center}

\textbf{Coverage}

These data are collected at all NEON terrestrial sites.

\begin{center}\rule{0.5\linewidth}{\linethickness}\end{center}

\textbf{Description}

Metagenomic sequence data from soil samples

\begin{center}\rule{0.5\linewidth}{\linethickness}\end{center}

\textbf{Abstract}

This data product contains the quality-controlled laboratory metadata
and QA results for NEON's shotgun metagenomics sequences derived from
soil microbial sampling. Typically, measurements are done on plot-level
composite samples and represent up to 3 randomly selected sampling
locations within a plot. For additional details, see protocol
\href{http://data.neonscience.org/api/v0/documents/NEON.DOC.014048vJ}{NEON.DOC.014048}:
TOS Protocol and Procedure for Soil Biogeochemical and Microbial
Sampling; and science design
\href{http://data.neonscience.org/api/v0/documents/NEON.DOC.000908vA}{NEON.DOC.000908vA}:
TOS Science Design for Terrestrial Microbial Diversity. Queries for this
data product will return metadata tables that include laboratory methods
and results from DNA extraction, sample preparation, and DNA sequencing
for samples from the specified sites and within the specified date
range. The actual sequence data are publicly available and may be
queried on the \href{http://metagenomics.anl.gov/}{Metagenomics Rapid
Annotation using Subsystem Technology (MG-RAST)} server. There may be
lags between publication of metadata on the NEON data portal and
availability of sequence data on the public sequence repository.

\begin{center}\rule{0.5\linewidth}{\linethickness}\end{center}

\textbf{Usage Notes}

Queries for this data product will return data from
mms\_metagenomeDnaExtraction and mms\_metagenomeSequencing for all dates
within the specified date range. The mms\_metagenomeDnaExtraction table
is generic for all microbial genetic data products: non-target samples
may be included and can be filtered using the field
``sequenceAnalysisType'' (filter to values of ``metagenomes'' and
``marker gene and metagenomes''). Each record in
mms\_metagenomeDnaExtraction may have one child record in
mms\_metagenomeSequencing. A given
mms\_metagenomeDnaExtraction.dnaSampleID is expected to be sampled one
time per collectDate (local time). Duplicate samples may exist where
protocol and/or data entry aberrations have occurred; users should check
data carefully for anomalies before joining tables. \newpage
.

\section{DP1.10108.001 Soil microbe marker gene
sequences}\label{dp1.10108.001-soil-microbe-marker-gene-sequences}

\begin{center}\rule{0.5\linewidth}{\linethickness}\end{center}

\textbf{Subsystem}

Terrestrial Observation System (TOS)

\begin{center}\rule{0.5\linewidth}{\linethickness}\end{center}

\textbf{Coverage}

These data are collected at all NEON terrestrial sites.

\begin{center}\rule{0.5\linewidth}{\linethickness}\end{center}

\textbf{Description}

DNA sequence data from ribosomal RNA marker genes from soil samples

\begin{center}\rule{0.5\linewidth}{\linethickness}\end{center}

\textbf{Abstract}

This data product contains the quality-controlled laboratory metadata
and 16S and ITS marker gene sequences derived from NEON's soil microbial
sampling. For additional details, see
\href{http://data.neonscience.org/api/v0/documents/NEON.DOC.014048vB}{NEON.DOC.014048:
TOS Protocol and Procedure for Soil Biogeochemical and Microbial
Sampling} and
\href{http://data.neonscience.org/api/v0/documents/NEON.DOC.000908vA}{NEON.DOC.000908:
TOS Science Design for Terrestrial Microbial Diversity}.

Queries for this data product return a downloadable data package with
laboratory methods and DNA extraction, PCR amplification, and sequencing
metadata for samples from the queried sites and date range. The actual
sequence data are publicly available and may be queried on the
\href{http://metagenomics.anl.gov/}{Metagenomics Rapid Annotation using
Subsystem Technology (MG-RAST)} server. There may be lags between
publication of metadata on the NEON data portal and availability of
sequence data on the public sequence repository. Sequence data may also
be obtained by querying NEON data sets at the
\href{https://www.ncbi.nlm.nih.gov/sra}{NCBI Sequence Read Archive (NCBI
SRA)} and the \href{https://www.ebi.ac.uk/}{European Bioinformatics
Institute (EMBL-EBI)}.

\begin{center}\rule{0.5\linewidth}{\linethickness}\end{center}

\textbf{Usage Notes}

Queries for this data product will return data for the tables
mmg\_soilDnaExtraction, mmg\_soilPcrAmplification\_16S (and ITS) and
mmg\_soilMarkerGeneSequencing\_16S (and ITS) for all dates within the
specified date range. A given mmg\_soilMarkerGeneSequencing\_16S(or
ITS).dnaSampleID is expected to generate one record for each
targetTaxonGroup. Duplicate samples and/or missing data may exist where
protocol and/or data entry aberrations have occurred; users should check
data carefully for anomalies before joining tables. \newpage
.

\section{DP1.10109.001 Soil microbe group
abundances}\label{dp1.10109.001-soil-microbe-group-abundances}

\begin{center}\rule{0.5\linewidth}{\linethickness}\end{center}

\textbf{Subsystem}

Terrestrial Observation System (TOS)

\begin{center}\rule{0.5\linewidth}{\linethickness}\end{center}

\textbf{Coverage}

These data are collected at all NEON terrestrial sites.

\begin{center}\rule{0.5\linewidth}{\linethickness}\end{center}

\textbf{Description}

Counts and relative abundances of marker genes from total archaea,
bacteria, and fungi observed by qPCR in soil microbial communities

\begin{center}\rule{0.5\linewidth}{\linethickness}\end{center}

\textbf{Abstract}

This data product contains the quality-controlled laboratory data and
metadata for NEON's soil bacterial, archaeal, and fungal group
abundances analysis, which are derived from soil microbial sampling.
Group abundances are quantified via qPCR on frozen, field-collected
soils. For additional details, see protocol
\href{http://data.neonscience.org/api/v0/documents/NEON.DOC.014048vJ}{NEON.DOC.014048}:
TOS Protocol and Procedure for Soil Biogeochemical and Microbial
Sampling; and science design
\href{http://data.neonscience.org/api/v0/documents/NEON.DOC.000908vA}{NEON.DOC.000908vA}:
TOS Science Design for Terrestrial Microbial Diversity.

\begin{center}\rule{0.5\linewidth}{\linethickness}\end{center}

\textbf{Usage Notes}

Queries for this data product will return data from
mga\_soilGroupAbundances for all dates within the specified date range,
as well as data from all dates for the mga\_batchResults and
mga\_labSummary (if the expanded package is selected). Note that both
soil and aquatic samples may be analyzed in the same batch and using the
same methods. As such, batch and lab summary tables may include data for
soil and aquatic samples. A given mga\_soilGroupAbundances.dnaSampleID
is expected to generate one record per targetTaxonGroup. Duplicate
samples and/or missing data may exist where protocol and/or data entry
aberrations have occurred; users should check data carefully for
anomalies before joining tables. \newpage
.

\section{DP1.20002.001 Land-water interface
images}\label{dp1.20002.001-land-water-interface-images}

\begin{center}\rule{0.5\linewidth}{\linethickness}\end{center}

\textbf{Subsystem}

Aquatic Instrument System (AIS)

\begin{center}\rule{0.5\linewidth}{\linethickness}\end{center}

\textbf{Coverage}

Aquatic stream gauge images are recorded at all aquatic sites.

\begin{center}\rule{0.5\linewidth}{\linethickness}\end{center}

\textbf{Description}

RGB and IR images of the lake, river, or stream vegetation, stream
surface, and stream gauge (where possible) taken from an automated
camera. Images are collected every 15 minutes.

\begin{center}\rule{0.5\linewidth}{\linethickness}\end{center}

\textbf{Abstract}

Physical characteristics of a water body, e.g., flow data in streams and
water level in lakes, are critical to interpreting chemical and
biological measurements or estimating fluxes into and out of systems. At
all NEON aquatic sites, water level is assessed via readings made from a
fixed staff gauge with 1 cm resolution prior to and following the
execution of an aquatic protocol. Photos may also be used for
qualitative estimates of snow cover, riparian characteristics, or
weather. At stream sites, a stage-discharge rating curve can be
developed for a specific, fixed cross section by collecting multiple
measurements of discharge, channel area, and gauge height over a range
of discharge levels. In lakes, water level will be tracked with gauge
height in addition to using pressure transducer sensors at identified
inflow and outflow locations.

Images are sent to and processed by PhenoCam, a cooperative network that
archives and distributes imagery and derived data products from digital
cameras deployed at research sites across North America and around the
world. These images are available for viewing and downloading from the
\href{https://phenocam.sr.unh.edu}{PhenoCam Gallery}. \newpage
.

\section{DP1.20004.001 Barometric pressure above water
on-buoy}\label{dp1.20004.001-barometric-pressure-above-water-on-buoy}

\begin{center}\rule{0.5\linewidth}{\linethickness}\end{center}

\textbf{Subsystem}

Aquatic Instrument System (AIS)

\begin{center}\rule{0.5\linewidth}{\linethickness}\end{center}

\textbf{Coverage}

Buoys will be deployed at all~lake and large river sites within NEON.

\begin{center}\rule{0.5\linewidth}{\linethickness}\end{center}

\textbf{Description}

Barometric pressure, available as one- and thirty-minute averages for
both station pressure and pressure reduced to sea level. Observations
are made on the meteorology station on the buoy in lakes and rivers.

\begin{center}\rule{0.5\linewidth}{\linethickness}\end{center}

\textbf{Abstract}

Barometric pressure on buoys is measured every minute and is reported as
1-minute instantaneous measurements and 30-minute mean values. Other
than the data collection frequency, this data product has the same data
streams and processing as barometric pressure measured at aquatic met
stations.

\begin{center}\rule{0.5\linewidth}{\linethickness}\end{center}

\textbf{Usage Notes}

The Valid Calibration Flag is currently blank, but will be incorporated
in future updates to this data product. \newpage
.

\section{DP1.20015.001 Specific conductivity in
groundwater}\label{dp1.20015.001-specific-conductivity-in-groundwater}

\begin{center}\rule{0.5\linewidth}{\linethickness}\end{center}

\textbf{Subsystem}

Aquatic Instrument System (AIS)

\begin{center}\rule{0.5\linewidth}{\linethickness}\end{center}

\textbf{Coverage}

These data are collected in the fall and spring at all NEON aquatic
sites except for MCRA, CUPE, and TECR where there are no groundwater
wells.

\begin{center}\rule{0.5\linewidth}{\linethickness}\end{center}

\textbf{Description}

In situ sensor-based measurements of specific conductance of groundwater
in wells

\begin{center}\rule{0.5\linewidth}{\linethickness}\end{center}

\textbf{Abstract}

Specific conductance is a proxy for the level of total dissolved solids
of the groundwater as well as related to the redox potential.
Conductivity may also be used as a tracer of distinct water masses for
understanding flow. NEON measurements of groundwater conductivity at
high temporal resolution. Up to eight wells (and as low as three) are
available per aquatic site. From NEON groundwater elevation
measurements, the magnitude and direction of groundwater flow can be
calculated, which may be coupled to better understand the exchange
between groundwater and surface water. This data product includes
continuous quality-controlled groundwater temperature captured every 5
minute and reported as 5-minute instantaneous measurements and 30-minute
averages. \newpage
.

\section{DP1.20016.001 Elevation of surface
water}\label{dp1.20016.001-elevation-of-surface-water}

\begin{center}\rule{0.5\linewidth}{\linethickness}\end{center}

\textbf{Subsystem}

Aquatic Instrument System (AIS)

\begin{center}\rule{0.5\linewidth}{\linethickness}\end{center}

\textbf{Coverage}

Elevation of surface water is measured at all aquatic sites. It is
measured at the upstream and downstream sensor stations in wadeable
streams; at a single station in rivers and at the inlet and outlet of
lake sites.

\begin{center}\rule{0.5\linewidth}{\linethickness}\end{center}

\textbf{Description}

Measurements of water surface elevation, available as one-, five-, and
thirty-minute averages in lakes and wadeable streams. Based on sensor
measurements of water pressure.

\begin{center}\rule{0.5\linewidth}{\linethickness}\end{center}

\textbf{Abstract}

Surface water elevation is controlled by precipitation at both the
landscape and channel scales, overland flow, interflow and groundwater
flow. It is correlated to discharge and is critical to understanding how
water moves through the environment, carrying nutrients and sediment,
modulating aquatic ecosystem structure and function. This data product
contains continuous, quality-controlled, surface water depth converted
to elevation above mean sea level. Measurements are captured once per
minute and reported as 1-minute instantaneous measurements and 30-minute
averages. \newpage
.

\section{DP1.20032.001}\label{dp1.20032.001}

Shortwave and longwave radiation above water on-buoy (net radiometer)

\begin{center}\rule{0.5\linewidth}{\linethickness}\end{center}

\textbf{Subsystem}

Aquatic Instrument System (AIS)

\begin{center}\rule{0.5\linewidth}{\linethickness}\end{center}

\textbf{Coverage}

Buoys will be deployed at all~lake and large river sites within NEON.

\begin{center}\rule{0.5\linewidth}{\linethickness}\end{center}

\textbf{Description}

Net radiation is composed of incoming and outgoing shortwave and
longwave radiation. These data products are available as one- and
thirty-minute averages. Observations of net shortwave and longwave
radiation are made by a sensor located on the meteorology station on the
buoy in lakes and rivers.

\begin{center}\rule{0.5\linewidth}{\linethickness}\end{center}

\textbf{Abstract}

Net shortwave and longwave radiation on buoys is measured every minute
and is reported as 1-minute instantaneous measurements and 30-minute
mean values. Other than the data collection frequency, this data product
has the same data streams and processing as net radiation at aquatic met
stations and on the terrestrial instrument towers. \newpage
.

\section{DP1.20033.001 Nitrate in surface
water}\label{dp1.20033.001-nitrate-in-surface-water}

\begin{center}\rule{0.5\linewidth}{\linethickness}\end{center}

\textbf{Subsystem}

Aquatic Instrument System (AIS)

\begin{center}\rule{0.5\linewidth}{\linethickness}\end{center}

\textbf{Coverage}

Upstream (S1) and downstream (S2) sensor stations are at all stream
sites within NEON. Buoys (S1) will be deployed as a single station at
all~lake and large river (non-wadeable) sites within NEON.

\begin{center}\rule{0.5\linewidth}{\linethickness}\end{center}

\textbf{Description}

In situ sensor-based nitrate concentration, available as fifteen- and
sixty-minute averages in surface water in lakes, wadeable and
non-wadeable streams

\begin{center}\rule{0.5\linewidth}{\linethickness}\end{center}

\textbf{Abstract}

Nitrate is measured using an optical sensor at the downstream (S2)
sensor station in streams and on the buoy (S1) at lake and river sites.
It is reported as a 15-minute mean value from 50 measurements made
during a 2-3 minute sampling burst every 15 minutes. \newpage
.

\section{DP1.20042.001 Photosynthetically active radiation at water
surface}\label{dp1.20042.001-photosynthetically-active-radiation-at-water-surface}

\begin{center}\rule{0.5\linewidth}{\linethickness}\end{center}

\textbf{Subsystem}

Aquatic Instrument System (AIS)

\begin{center}\rule{0.5\linewidth}{\linethickness}\end{center}

\textbf{Coverage}

S1 (upstream) and S2 (downstream) sensor sets are at all stream sites
within NEON. Buoys will be deployed at all~lake and large river sites
within NEON.

\begin{center}\rule{0.5\linewidth}{\linethickness}\end{center}

\textbf{Description}

Photosynthetically Active Radiation (PAR) observations represent the
radiation flux at wavelengths between 400-700 nm, which constitute the
wavelengths that drive photosynthesis. This data product is available as
one-, five-, and thirty-minute averages. Observations are made at the
aquatic sensor set location at lakes, non-wadeable streams, and wadeable
streams.

\begin{center}\rule{0.5\linewidth}{\linethickness}\end{center}

\textbf{Abstract}

Photosynthetically active radiation at water surface is measured at 1 Hz
at stream sensor sets and twice per minute on buoys at lake and river
sites. It is reported as 1-minute mean measurements and 30-minute mean
values.

\begin{center}\rule{0.5\linewidth}{\linethickness}\end{center}

\textbf{Usage Notes}

Due to the nature of the floating platform, above water PAR sensors on
the buoy will not meet the following requirement at this time:
NEON.AIS.4.1334 ``All radiation sensors shall be mounted to remain level
to within ±1°''. \newpage
.

\section{DP1.20046.001 Air temperature above water
on-buoy}\label{dp1.20046.001-air-temperature-above-water-on-buoy}

\begin{center}\rule{0.5\linewidth}{\linethickness}\end{center}

\textbf{Subsystem}

Aquatic Instrument System (AIS)

\begin{center}\rule{0.5\linewidth}{\linethickness}\end{center}

\textbf{Coverage}

Buoys will be deployed at all~lake and large river sites within NEON.

\begin{center}\rule{0.5\linewidth}{\linethickness}\end{center}

\textbf{Description}

Air temperature, available as one- and thirty-minute averages.
Observations are made by a sensor located on the meteorology station on
the buoy in lakes and rivers. Temperature observations are made using
platinum resistance thermometers, which are housed in a passive shield
to reduce radiative bias.

\begin{center}\rule{0.5\linewidth}{\linethickness}\end{center}

\textbf{Abstract}

Air temperature on buoys is measured every minute and is reported as
1-minute instantaneous measurements and 30-minute mean values. Unlike
other locations, such as aquatic met stations and the terrestrial tower,
there is not a separate sensor for air temperature. The temperature
measurements from the relative humidity sensor are used for this buoy
data product and these data are a subset of the Relative Humidity of the
air above lakes on buoy (DP1.20271.001). \newpage
.

\section{DP1.20048.001 Stream discharge field
collection}\label{dp1.20048.001-stream-discharge-field-collection}

\begin{center}\rule{0.5\linewidth}{\linethickness}\end{center}

\textbf{Subsystem}

Aquatic Observation System (AOS)

\begin{center}\rule{0.5\linewidth}{\linethickness}\end{center}

\textbf{Coverage}

These data are collected in aquatic wadeable streams and rivers.

\begin{center}\rule{0.5\linewidth}{\linethickness}\end{center}

\textbf{Description}

Discharge measurements from field-based surveys

\begin{center}\rule{0.5\linewidth}{\linethickness}\end{center}

\textbf{Abstract}

This data product contains the quality-controlled, native sampling
resolution data from NEON's stream discharge field collection protocol.
Individual discharge measurements are conducted by means of surveys that
occur in wadeable streams and rivers along permanently benchmarked
cross-sections at NEON aquatic sites. During discharge measurements the
waterbody is divided into lateral sub-sections (of which there are
typically 20-25 per cross-section). Within each subsection, an
instantaneous velocity magnitude is obtained and transformed to a
volumetric discharge magnitude by applying the velocity across the full
subsection area. Total stream discharge is then calculated by a
flowmeter (in wadeable streams) or an acoustic doppler current profiler
(in rivers), which sums the discrete volumetric discharges for each
subsection. Further details with regards to wadeable streams can be
found in
\href{data.neonscience.org/api/v0/documents/NEON.DOC.001085vD}{NEON.DOC.001085}
AOS Protocol and Procedure: Stream Discharge. A protocol for river
discharge is forthcoming.

\begin{center}\rule{0.5\linewidth}{\linethickness}\end{center}

\textbf{Usage Notes}

Queries for this data product will return all data collected within the
date range specified. Data are provided in individual measurement files.
The protocol dictates that stream discharge field collection will take
place once at each site per event (one record expected per siteID and
collectDate combination in dsc\_fieldData). Each record in
dsc\_fieldData will have a variable number of child records in
dsc\_individualFieldData. The number of records in
dsc\_individualFieldData per siteID per collectDate varies depending on
the width of the waterbody and usually ranges from 10 to 25 records.
Duplicates may exist where protocol and/or data entry aberrations have
occurred; users should check data carefully for anomalies before
analyzing data. \newpage
.

\section{DP1.20053.001 Temperature (PRT) in surface
water}\label{dp1.20053.001-temperature-prt-in-surface-water}

\begin{center}\rule{0.5\linewidth}{\linethickness}\end{center}

\textbf{Subsystem}

Aquatic Instrument System (AIS)

\begin{center}\rule{0.5\linewidth}{\linethickness}\end{center}

\textbf{Coverage}

All wadeable stream sites monitor stream water temperature at both the
upstream and downstream sensor set locations.

\begin{center}\rule{0.5\linewidth}{\linethickness}\end{center}

\textbf{Description}

Surface water temperature, available as one-, five-, and thirty-minute
averages, measured by a platinum resistance thermometer at the sensor
location in lakes, wadeable and non-wadeable streams

\begin{center}\rule{0.5\linewidth}{\linethickness}\end{center}

\textbf{Abstract}

This data product contains quality-controlled continuous surface water
temperature readings from a sensor within each of NEON's~wade-able
stream sensor sets. \newpage
.

\section{DP1.20059.001 Windspeed and direction above water
on-buoy}\label{dp1.20059.001-windspeed-and-direction-above-water-on-buoy}

\begin{center}\rule{0.5\linewidth}{\linethickness}\end{center}

\textbf{Subsystem}

Aquatic Instrument System (AIS)

\begin{center}\rule{0.5\linewidth}{\linethickness}\end{center}

\textbf{Coverage}

Buoys will be deployed at all~lake and large river sites within NEON.

\begin{center}\rule{0.5\linewidth}{\linethickness}\end{center}

\textbf{Description}

Wind speed and direction; observations are made by 2-D sonic anemometer
sensors located on lake and river buoys.

\begin{center}\rule{0.5\linewidth}{\linethickness}\end{center}

\textbf{Abstract}

Wind speed and direction on buoys are measured 11 times per minute and
reported as 2- and 30-minute mean values. The buoy wind sensor, and
therefore data processing, is different than other wind sensors at
aquatic met stations and on the terrestrial tower. \newpage
.

\section{DP1.20063.001 Aquatic plant bryophyte chemical
properties}\label{dp1.20063.001-aquatic-plant-bryophyte-chemical-properties}

\begin{center}\rule{0.5\linewidth}{\linethickness}\end{center}

\textbf{Subsystem}

Aquatic Observation System (AOS)

\begin{center}\rule{0.5\linewidth}{\linethickness}\end{center}

\textbf{Coverage}

Measured at all NEON aquatic sites (wadeable streams, lakes, and
non-wadeable streams).

\begin{center}\rule{0.5\linewidth}{\linethickness}\end{center}

\textbf{Description}

C and N concentrations of aquatic plant and bryophytes from benthic
collections in lakes, non-wadeable streams, and wadeable streams

\begin{center}\rule{0.5\linewidth}{\linethickness}\end{center}

\textbf{Abstract}

This data product contains the quality-controlled, field sampling
metadata for aquatic plant and bryophyte carbon and nitrogen analyses.
Benthic field samples are collected in wadeable streams, rivers, and
lakes, and processed at the domain support facility. Aquatic plant and
bryophyte chemistry samples are derived from clip harvest samples,
collected once per year during the mid-summer aquatic biological
sampling bout. Samples are collected from a known benthic area,
separated by taxon in the domain lab, identified, and ground and shipped
to a contracting laboratory for chemical analyses. For additional
details, see
\href{http://data.neonscience.org/api/v0/documents/NEON.DOC.003039vB}{NEON.DOC.003039}:
AOS Protocol and Procedure: Aquatic Plant, Bryophyte, Lichen and
Macroalgae Sampling and
\href{http://data.neonscience.org/api/v0/documents/NEON.DOC.001152vA}{NEON.DOC.001152}:
NEON Aquatic Sampling Strategy.

\begin{center}\rule{0.5\linewidth}{\linethickness}\end{center}

\textbf{Usage Notes}

Queries for this data product will return all data for apl\_clipHarvest,
apl\_domainLabChemistry, and apl\_algaeExternalLabDataPerSample during
the date range specified. Each record in apl\_clipHarvest may have
several child records in apl\_domainLabChemistry and
apl\_plantExternalLabDataPerSample, where each record represents a
unique sampleID - analyte - replicate combination. The expanded package
also returns summary data for each analytical method from the contractor
in asi\_externalLabPOMSummaryData with one record per date range
(startDate - endDate), analyte, instrument, and method. Duplicates may
exist where protocol and/or data entry abberrations have occurred; users
should check data carefully for anomalies before analyzing data.
Taxonomic IDs of species of concern have been `fuzzed'; see data package
readme files for more information. \newpage
.

\section{DP1.20066.001 Aquatic plant bryophyte macroalgae clip
harvest}\label{dp1.20066.001-aquatic-plant-bryophyte-macroalgae-clip-harvest}

\begin{center}\rule{0.5\linewidth}{\linethickness}\end{center}

\textbf{Subsystem}

Aquatic Observation System (AOS)

\begin{center}\rule{0.5\linewidth}{\linethickness}\end{center}

\textbf{Coverage}

Measured at all NEON aquatic sites (wadeable streams, lakes, and
non-wadeable streams).

\begin{center}\rule{0.5\linewidth}{\linethickness}\end{center}

\textbf{Description}

Dry weight of aquatic plant, bryophyte, and macroalgae from benthic
quadrats in lakes, non-wadeable streams, and wadeable streams

\begin{center}\rule{0.5\linewidth}{\linethickness}\end{center}

\textbf{Abstract}

This data product contains the quality-controlled, field sampling
metadata and associated taxonomic, and biomass data for aquatic plants,
bryophytes, and macroalgae. Benthic field samples are collected in
wadeable streams, rivers, and lakes, and processed at the domain support
facility. Clip harvest samples are collected once per year during the
mid-summer aquatic biological sampling bout, and additional
presence/absence data are collected in lakes and rivers during bouts 1
and 3 (similar to point transect data for streams). Grab samples are
collected from a known benthic area, separated by taxon in the domain
lab, identified, and processed for dry mass and ash-free dry mass. For
additional details, see
\href{http://data.neonscience.org/api/v0/documents/NEON.DOC.003039vB}{NEON.DOC.003039}:
AOS Protocol and Procedure: Aquatic Plant, Bryophyte, Lichen and
Macroalgae Sampling and
\href{http://data.neonscience.org/api/v0/documents/NEON.DOC.001152vA}{NEON.DOC.001152}:
NEON Aquatic Sampling Strategy.

\begin{center}\rule{0.5\linewidth}{\linethickness}\end{center}

\textbf{Usage Notes}

Queries for this data product will return data for apl\_clipHarvest,
apl\_biomass, and apl\_taxonomyProcessed during the date range
specified. The expanded package also returns raw taxonomy data in
apl\_taxonomyRaw. Information on morphospecies (apc\_morphospecies) over
all dates will also be returned in the expanded package. A given
apl\_clipHarvest.locationID within a namedLocation is expected to be
sampled once per collectDate (local time). A record from
apl\_clipHarvest may have zero or more related records in apl\_biomass,
depending on whether a physical sample (fieldID) was collected. An
record from apl\_biomass may have zero or more records in
apl\_taxonomy(raw or processed) depending on whether a sampleID was sent
for identification. Morphospecies may be resolved at any date. Some
morphospecies are never resolved. Duplicates may exist where protocol
and/or data entry abberations have occurred; users should check data
carefully for anomalies before analyzing data. Taxonomic IDs of species
of concern have been `fuzzed'; see data package readme files for more
information. \newpage
.

\section{DP1.20072.001}\label{dp1.20072.001}

Aquatic plant, bryophyte, lichen, and macroalgae point counts in
wadeable streams

\begin{center}\rule{0.5\linewidth}{\linethickness}\end{center}

\textbf{Subsystem}

Aquatic Observation System (AOS)

\begin{center}\rule{0.5\linewidth}{\linethickness}\end{center}

\textbf{Coverage}

Measured at all NEON wadeable stream sites.

\begin{center}\rule{0.5\linewidth}{\linethickness}\end{center}

\textbf{Description}

Point counts of aquatic plants, bryophytes, lichens, and macroalgae from
transects in wadeable streams

\begin{center}\rule{0.5\linewidth}{\linethickness}\end{center}

\textbf{Abstract}

This data product contains the quality-controlled, field sampling
metadata and associated taxonomic, and biomass data for aquatic plants,
bryophytes, and macroalgae. Field data are collected at 10 permanent
transects in wadeable streams three times per year. For additional
details, see
\href{http://data.neonscience.org/api/v0/documents/NEON.DOC.003039vB}{NEON.DOC.003039}:
AOS Protocol and Procedure: Aquatic Plant, Bryophyte, Lichen and
Macroalgae Sampling and
\href{http://data.neonscience.org/api/v0/documents/NEON.DOC.001152vA}{NEON.DOC.001152}:
NEON Aquatic Sampling Strategy.

\begin{center}\rule{0.5\linewidth}{\linethickness}\end{center}

\textbf{Usage Notes}

Queries for this data product will return all data for
apc\_pointTransect, apc\_perTaxon, apc\_taxonomyProcessed subset to the
date range specified. The expanded package also returns raw taxonomy
data in apc\_taxonomyRaw. Information on morphospecies
(apc\_morphospecies), and the collection (apc\_voucher) and
identification (apc\_voucherTaxonomyRaw, apc\_voucherTaxonomyProcessed)
of voucher specimens over all dates will also be returned in the
expanded package. A given apc\_pointTransect.pointNumber within a
namedLocation (transect location at a single site)is expected to be
sampled once per collectDate (local time). A record from
apc\_pointTransect may have zero or more related records in
apc\_perTaxon, depending on whether zero or more plants were detected at
a given point. An record from apc\_perTaxon may have zero or more
records in apc\_taxonomy(raw or processed) depending on whether a
physical sampleID was collected. Voucher samples are collected
opportunistically, and morphospecies may be resolved at any date. Some
morphospecies are never resolved. Duplicates may exist where protocol
and/or data entry abberations have occurred; users should check data
carefully for anomalies before analyzing data. Taxonomic IDs of species
of concern have been `fuzzed'; see data package readme files for more
information. \newpage
.

\section{DP1.20086.001 Benthic microbe community
composition}\label{dp1.20086.001-benthic-microbe-community-composition}

\begin{center}\rule{0.5\linewidth}{\linethickness}\end{center}

\textbf{Subsystem}

Aquatic Observation System (AOS)

\begin{center}\rule{0.5\linewidth}{\linethickness}\end{center}

\textbf{Coverage}

These data are collected at all NEON wadeable stream sites.

\begin{center}\rule{0.5\linewidth}{\linethickness}\end{center}

\textbf{Description}

Counts and relative abundances of archaeal, bacterial, and fungal taxa
observed in benthic microbial communities

\begin{center}\rule{0.5\linewidth}{\linethickness}\end{center}

\textbf{Abstract}

This data product contains the quality-controlled laboratory data and
metadata for NEON bacterial, archaeal, and fungal community composition
data derived from benthic microbial sampling in wadeable streams. Taxon
tables are derived from the 16S and ITS marker gene sequencing data
product, NEON.DP1.20280. Taxonomic data are generated from
quality-filtered sequence data using standard bioinformatics software.
For additional details about sampling methods and design, see
\href{http://data.neonscience.org/api/v0/documents/NEON.DOC.003044vB}{NEON.DOC.003044}:
AOS Protocol and Procedure: Aquatic Microbial Sampling; and
\href{http://data.neonscience.org/api/v0/documents/NEON.DOC.001152vA}{NEON.DOC.001152}:
NEON Aquatic Sampling Strategy.

\begin{center}\rule{0.5\linewidth}{\linethickness}\end{center}

\textbf{Usage Notes}

Queries for the basic download data product will return data from
amb\_fieldParent, mcc\_benthicTaxonTableMetadata\_16S and
mcc\_benthicTaxonTableMetadata\_ITS for all dates within the specified
date range. In addition, the expanded download also includes per-sample
taxon tables for all dates within the specified date range and data from
all dates for the mcc\_taxonTableLabSummary. A given dnaSampleID for
mcc\_benthicTaxonTableMetadata\_16S(ITS) is expected to generate one
record for each completeTaxonomy, and the number of records per sample
should equal the number of unique values for completeTaxonomy. Duplicate
samples and/or missing data may exist where protocol and/or data entry
aberrations have occurred; users should check data carefully for
anomalies before joining tables. \newpage
.

\section{DP1.20092.001 Chemical properties of
groundwater}\label{dp1.20092.001-chemical-properties-of-groundwater}

\begin{center}\rule{0.5\linewidth}{\linethickness}\end{center}

\textbf{Subsystem}

Aquatic Observation System (AOS)

\begin{center}\rule{0.5\linewidth}{\linethickness}\end{center}

\textbf{Coverage}

These data are collected in the fall and spring at all NEON aquatic
sites except for MCRA, CUPE, and TECR where there are no groundwater
wells.

\begin{center}\rule{0.5\linewidth}{\linethickness}\end{center}

\textbf{Description}

Grab samples of groundwater chemistry including general chemistry,
anions, cations, and nutrients.

\begin{center}\rule{0.5\linewidth}{\linethickness}\end{center}

\textbf{Abstract}

This data product contains the quality-controlled, native sampling
resolution data from NEON's groundwater chemistry sampling protocol.
Subsamples are analyzed at NEON domain headquarters for alkalinity and
acid neutralizing capacity (ANC); other subsamples are sent to external
facilities for a broad suite of analytes, including dissolved and total
nutrients and carbon, cations and anions, and general chemistry. For
additional details on protocol, see the AOS Protocol and Procedure:
Water Chemistry Sampling in Surface Waters and Groundwater
(NEON.DOC.002905).

\begin{center}\rule{0.5\linewidth}{\linethickness}\end{center}

\textbf{Usage Notes}

The protocol dictates that each siteID x stationID combination is
sampled at least once per event (one record expected per parentSampleID
in gwc\_fieldSuperParent). A record from gwc\_fieldSuperParent may have
zero or one child records in gwc\_fieldData, depending on whether a
water sample was collected. In the event that a water sample cannot be
taken, a record will still be created in gwc\_fieldSuperParent, and
gwc\_fieldSuperParent.samplingImpractical will be something other than
NULL, but there will be no corresponding record in gwc\_fieldData. Each
record from gwc\_fieldData is expected to have two child records in
gwc\_domainLabData (one each for ALK and ANC), and each record from
gwc\_fieldData is also expected to have one child record in
gwc\_externalLabData. However, duplicates and/or missing data may exist
where protocol and/or data entry abberations have occurred; Users should
check data carefully for anomalies before joining tables. \newpage
.

\section{DP1.20093.001 Chemical properties of surface
water}\label{dp1.20093.001-chemical-properties-of-surface-water}

\begin{center}\rule{0.5\linewidth}{\linethickness}\end{center}

\textbf{Subsystem}

Aquatic Observation System (AOS)

\begin{center}\rule{0.5\linewidth}{\linethickness}\end{center}

\textbf{Coverage}

Measured at all NEON aquatic sites.

\begin{center}\rule{0.5\linewidth}{\linethickness}\end{center}

\textbf{Description}

Grab samples of surface water chemistry including general chemistry,
anions, cations, and nutrients.

\begin{center}\rule{0.5\linewidth}{\linethickness}\end{center}

\textbf{Abstract}

This data product contains the quality-controlled, native sampling
resolution data from NEON's surface water chemistry sampling protocol.
Subsamples are analyzed at NEON domain headquarters for alkalinity and
acid neutralizing capacity (ANC); other subsamples are sent to external
facilities for a broad suite of analytes, including dissolved and total
nutrients and carbon, cations and anions, and general chemistry. For
additional details on NEON field and laboratory protocols, see the AOS
Protocol and Procedure: Water Chemistry Sampling in Surface Waters and
Groundwater (NEON.DOC.002905).

\begin{center}\rule{0.5\linewidth}{\linethickness}\end{center}

\textbf{Usage Notes}

The protocol dictates that each siteID x stationID combination is
sampled at least once per event (one record expected per parentSampleID
in swc\_fieldSuperParent). A record from swc\_fieldSuperParent may have
zero or one child records in swc\_fieldData, depending on whether a
water sample was collected. In the event that a water sample cannot be
taken, a record will still be created in swc\_fieldSuperParent, and
swc\_fieldSuperParent.samplingImpractical will be something other than
NULL, but there will be no corresponding record in swc\_fieldData. Each
record from swc\_fieldData is expected to have two child records in
swc\_domainLabData (one each for ALK and ANC), and each record from
swc\_fieldData is also expected to have one child record in
swc\_externalLabData. However, duplicates and/or missing data may exist
where protocol and/or data entry abberations have occurred; Users should
check data carefully for anomalies before joining tables. \newpage
.

\section{DP1.20097.001 Dissolved gases in surface
water}\label{dp1.20097.001-dissolved-gases-in-surface-water}

\begin{center}\rule{0.5\linewidth}{\linethickness}\end{center}

\textbf{Subsystem}

Aquatic Observation System (AOS)

\begin{center}\rule{0.5\linewidth}{\linethickness}\end{center}

\textbf{Coverage}

Measured at all NEON aquatic sites (wadeable streams, lakes, and
non-wadeable streams).

\begin{center}\rule{0.5\linewidth}{\linethickness}\end{center}

\textbf{Description}

Grab samples of surface water dissolved gases including carbon dioxide,
methane, and nitrous oxide

\begin{center}\rule{0.5\linewidth}{\linethickness}\end{center}

\textbf{Abstract}

This data product contains the quality-controlled, native sampling
resolution data from NEON's surface water dissolved gas sampling
protocol. Water samples are equilibrated with air in the field. Samples
of reference air (pre-equilibration) and equilibrated air
(post-equilibration) are sent to external facilities for analysis to
determine carbon dioxide, methane, and nitrous oxide concentrations in
the gas samples. Data users should refer to the user guide for dissolved
gases in surface water
(\href{http://data.neonscience.org/api/v0/documents/NEON_dissolvGasInWater_userGuide_vA}{NEON\_dissolvGasInWater\_UserGuide})
for suggestions on how to calculate dissolved concentrations of carbon
dioxide, methane, and nitrous oxide in the surface waters from which
samples were collected using Henry's Law and mass balance equations. For
additional details on NEON field and laboratory protocols, see the AOS
Protocol and Procedure: Surface Water Dissolved Gas Sampling
\href{http://data.neonscience.org/api/v0/documents/NEON.DOC.001199vK}{NEON.DOC.001199}.

\begin{center}\rule{0.5\linewidth}{\linethickness}\end{center}

\textbf{Usage Notes}

Queries for this data product will return data from the date range
specified for sdg\_fieldSuperParent, sdg\_fieldData, sdg\_fieldDataAir,
sdg\_fieldDataProc, and sdg\_externalLabData. The
sdg\_externalLabSummaryData from all dates and from relevant external
laboratories will be returned with the expanded package. The protocol
dictates that one water sample is collected per station per site per
date. At streams and rivers only one station is sampled, while at lakes
multiple stations may be sampled depending on the stratification or
other environmental conditions. A record from sdg\_fieldSuperParent may
have zero or one child records in sdg\_fieldData, depending on whether a
water sample was sucessfully collected. In the event that a water sample
cannot be taken, a record will still be created in
sdg\_fieldSuperParent, and sdg\_fieldSuperParent. samplingImpractical
will be something other than NULL, but there will be no corresponding
record in sdg\_fieldData. Each record from sdg\_fieldData is expected to
have one child record in sdg\_fieldDataProc, which is a mixture of a
water sample created in sdg\_fieldData and reference air sample created
in sdg\_fieldDataAir. Each record from sdg\_fieldDataProc and
sdg\_fieldDataAir is expected to have one child record in
sdg\_externalLabData. Duplicates may exist where protocol and/or data
entry abberations have occurred; users should check data carefully for
anomalies before joining tables. \newpage
.

\section{DP1.20100.001 Elevation of
groundwater}\label{dp1.20100.001-elevation-of-groundwater}

\begin{center}\rule{0.5\linewidth}{\linethickness}\end{center}

\textbf{Subsystem}

Aquatic Instrument System (AIS)

\begin{center}\rule{0.5\linewidth}{\linethickness}\end{center}

\textbf{Coverage}

These data are collected in the fall and spring at all NEON aquatic
sites except for MCRA, CUPE, and TECR where there are no groundwater
wells.

\begin{center}\rule{0.5\linewidth}{\linethickness}\end{center}

\textbf{Description}

Sensor based measurement of groundwater elevation calculated from
pressure transducer readings in each well

\begin{center}\rule{0.5\linewidth}{\linethickness}\end{center}

\textbf{Abstract}

There are important linkages and feedbacks between groundwater and
surface water in streams, rivers, and lakes. NEON measurements of
groundwater elevation at high resolution temporal changes informs these
linkages. Three to eight wells are available per aquatic site. In this
way, magnitude and direction of groundwater flow can be calculated.
Wells are located both near the water body and further from the water
body to enable investigation of hyporheic groundwater flow paths. This
data product contains continuous, quality-controlled, groundwater depth
converted to elevation above mean sea level. Measurements are captured
every five minutes and reported as 5-minute instantaneous measurements
and 30-minute averages.

\begin{center}\rule{0.5\linewidth}{\linethickness}\end{center}

\textbf{Usage Notes}

If the groundwater table drops below the sensor, the sensor elevation is
reported as groundwater elevation in the 5-minute data files and NA is
reported for the 30 minute data. In the sensor position file, the
reference elevation is the ground elevation and the z-offset is the
sensor position in relation to the ground elevation. a bug in the sensor
position code only publishes the current sensor position which may
differ from its position at the time the data was recorded. This will be
fixed in the future.

NEON groundwater well pressure transducers initially hung on cables that
were susceptible to slipping from the well's upper reference point. At
many sites these slippages resulted in deviations from the as-built
sensor positions and thus produced incorrect groundwater elevations
after transitioning the raw pressure data. Early NEON groundwater
elevation data prior to reconfirmation of the sensor position is flagged
for sites where this issue was identified; however, some data may be
salvageable as many of the slippage events can be clearly identified in
the data and they frequently appear to align with field staff service
visits. \newpage
.

\section{DP1.20105.001 Fish sequences DNA
barcode}\label{dp1.20105.001-fish-sequences-dna-barcode}

\begin{center}\rule{0.5\linewidth}{\linethickness}\end{center}

\textbf{Subsystem}

Aquatic Observation System (AOS)

\begin{center}\rule{0.5\linewidth}{\linethickness}\end{center}

\textbf{Coverage}

These data are collected at all NEON aquatic sites (wadeable streams,
lakes, and non-wadeable streams).

\begin{center}\rule{0.5\linewidth}{\linethickness}\end{center}

\textbf{Description}

CO1 DNA sequences from select fish in lakes and wadeable streams

\begin{center}\rule{0.5\linewidth}{\linethickness}\end{center}

\textbf{Abstract}

This data product contains the quality-controlled laboratory metadata
and QA results for NEON's cytochrome oxidase I (COI) barcoding of fish
sequences. Fin clips are taken from a subset of collected fish for DNA
analysis. The DNA barcoding procedure involves the removal of tissue,
extracting and sequencing DNA from the tissue, and matching that
sequence data to sequences from previously identified voucher specimens.
DNA analysis serves a number of purposes, including verification of
taxonomy of specimens that do not receive expert identification,
clarification of the taxonomy of rare or cryptic species, and
characterization of diversity using molecular markers. For additional
details on fish collection, see protocol
\href{http://data.neonscience.org/api/v0/documents/NEON.DOC.001295vD}{NEON.DOC.001295}:
AOS Protocol and Procedure: Fish Sampling in Wadeable Streams and
\href{http://data.neonscience.org/api/v0/documents/NEON.DOC.NEON.DOC.001296vD}{NEON.DOC.001296}:
AOS Protocol and Procedure: Fish Sampling in Lakes . Queries for this
data product will return metadata tables formatted for submission to the
Barcode of Life Database. These queries will also provide links to the
actual sequence data, which are publicly available on the Barcode of
Life Datasystem (BOLD, \url{http://www.barcodinglife.com/}). The
sequence data can be obtained by following the links from the NEON data
portal, or by directly querying NEON data sets on the BOLD server. From
the NEON portal, the link ``BOLD Project: Fish sequences DNA barcode''
redirects to a page on the BOLD public data portal for the queried data.
This is a dynamic link and will automatically update based on the user
query.

\begin{center}\rule{0.5\linewidth}{\linethickness}\end{center}

\textbf{Usage Notes}

Taxonomic IDs of species of concern have been `fuzzed'; see data package
readme files for more information. \newpage
.

\section{DP1.20107.001 Fish electrofishing, gill netting, and fyke
netting
counts}\label{dp1.20107.001-fish-electrofishing-gill-netting-and-fyke-netting-counts}

\begin{center}\rule{0.5\linewidth}{\linethickness}\end{center}

\textbf{Subsystem}

Aquatic Observation System (AOS)

\begin{center}\rule{0.5\linewidth}{\linethickness}\end{center}

\textbf{Coverage}

These data are collected in the fall and spring at all NEON wadeable
stream sites except for Como Creek (COMO, not sampled), Martha Creek
(MART, sampled Fall only) and McRae Creek (MCRA, sampled Fall only) and
at all lake sites except for Suggs Lake (SUGG) and Barco Lake (BARC).

\begin{center}\rule{0.5\linewidth}{\linethickness}\end{center}

\textbf{Description}

Counts of fish from electrofishing surveys in wadeable streams, or
electrofishing, gill netting, and/or fyke netting surveys in lakes.
Includes fish standard length and individual mass

\begin{center}\rule{0.5\linewidth}{\linethickness}\end{center}

\textbf{Abstract}

This data product contains the quality-controlled, native sampling
resolution data from NEON's fish sampling. Fish are sampled using a
combination of electrofishing, gill-nets and mini-fyke nets. Field
technicians identify fish to the lowest practical taxonomic level and
then weigh and measure a subset of captured individuals before
releasing. For additional details see protocols
\href{http://data.neonscience.org/api/v0/documents/NEON.DOC.001295vD}{DOC.001295}:
AOS Protocol and Procedure: Fish Sampling in Wadeable Streams and
\href{http://data.neonscience.org/api/v0/documents/NEON.DOC.001296vD}{DOC.001296}:
AOS Protocol and Procedure: Fish Sampling in Lakes and science design
\href{http://data.neonscience.org/api/v0/documents/NEON.DOC.001152vA}{NEON.DOC.001152}:
NEON Aquatic Sampling Strategy.

\begin{center}\rule{0.5\linewidth}{\linethickness}\end{center}

\textbf{Usage Notes}

Queries for this data product will return data subset to data collected
during the date range specified for the tables fsh\_fieldData,
fsh\_perPass, fsh\_perFish and fsh\_bulkSampling and data from all dates
for fsh\_morphospecies. A record from fsh\_fieldData may have zero (if
sampling is impractical; e.g.~the location is dry, ice-covered, etc) or
up to 5 child records in fsh\_perPass, depending on whether the reach is
being sampled using multiple electrofishing passes, and/or multiple
sampler types. Each record from fsh\_perPass may have zero (if
\textbf{targetTaxaPresent} = `No') or more child records per taxonID in
fsh\_perFish and 0 or more child records in fsh\_bulkCount, depending on
the taxonomic diversity and abundance at the site. Duplicates and/or
missing data may exist where protocol and/or data entry aberrations have
occurred; users should check data carefully for anomalies before joining
tables. Taxonomic IDs of species of concern have been `fuzzed'; see data
package readme files for more information. \newpage
.

\section{DP1.20120.001 Macroinvertebrate
collection}\label{dp1.20120.001-macroinvertebrate-collection}

\begin{center}\rule{0.5\linewidth}{\linethickness}\end{center}

\textbf{Subsystem}

Aquatic Observation System (AOS)

\begin{center}\rule{0.5\linewidth}{\linethickness}\end{center}

\textbf{Coverage}

These data are collected at all NEON aquatic sites (wadeable streams,
lakes, and non-wadeable streams).

\begin{center}\rule{0.5\linewidth}{\linethickness}\end{center}

\textbf{Description}

Collection of benthic macroinvertebrates using multiple sampling methods
in lakes, non-wadeable streams, and wadeable streams

\begin{center}\rule{0.5\linewidth}{\linethickness}\end{center}

\textbf{Abstract}

This data product contains the quality-controlled, native sampling
resolution data from NEON's aquatic Macroinvertebrate collection and
field metadata, as well as associated taxonomic, morphometric, and count
analyses data provided by a contracted lab. Benthic field samples are
collected in wadeable streams, rivers, and lakes, three times per year
during the growing season using the type of sampler most suitable to the
habitat types present at the site. Samples are preserved in ethanol in
the field and shipped to a contracting lab for analysis. For additional
details, see
\href{http://data.neonscience.org/api/v0/documents/NEON.DOC.003046vB}{NEON.DOC.003046}
AOS Protocol and Procedure: Aquatic Macroinvertebrate Sampling and
\href{http://data.neonscience.org/api/v0/documents/NEON.DOC.001152vA}{NEON.DOC.001152}:
NEON Aquatic Sampling Strategy.

\begin{center}\rule{0.5\linewidth}{\linethickness}\end{center}

\textbf{Usage Notes}

Queries for this data product will return all data for inv\_fieldData,
inv\_perSample, and inv\_taxonomyProcessed during the date range
specified. If sampling is not impractical, each record for in
inv\_fieldData will have one corresponding record in inv\_perSample, and
may have multiple corresponding records in inv\_taxonomyRaw and
inv\_taxonomyProcessed, one record for each scientificName and sizeClass
combination. A record from inv\_fieldData may have multiple or no
records in inv\_perVial, as that table represents individuals removed
from the final archived sample and placed in the external lab's in-house
reference collection, records in this table are opportunistic. The
expanded package also returns raw taxonomic data from the external
taxonomist in inv\_taxonomyRaw and information on the contents of the
vial sent to the archive facility in inv\_perVial. Duplicates may exist
where protocol and/or data entry aberrations have occurred; users should
check data carefully for anomalies before analyzing data. Taxonomic IDs
of species of concern have been `fuzzed'; see data package readme files
for more information. \newpage
.

\section{DP1.20126.001 Macroinvertebrate DNA
barcode}\label{dp1.20126.001-macroinvertebrate-dna-barcode}

\begin{center}\rule{0.5\linewidth}{\linethickness}\end{center}

\textbf{Subsystem}

Aquatic Observation System (AOS)

\begin{center}\rule{0.5\linewidth}{\linethickness}\end{center}

\textbf{Coverage}

Measured at all NEON aquatic sites (wadeable streams, lakes, and
non-wadeable streams).

\begin{center}\rule{0.5\linewidth}{\linethickness}\end{center}

\textbf{Description}

CO1 DNA sequences of the aquatic macroinvertebrate community

\begin{center}\rule{0.5\linewidth}{\linethickness}\end{center}

\textbf{Abstract}

This data product contains the quality-controlled, native sampling
resolution data from NEON's aquatic Macroinvertebrate DNA barcode
sampling protocol, as well as associated metadata provided by a
contracted lab. Benthic field samples are collected in wadeable streams,
rivers, and lakes, three times per year during the growing season using
the type of sampler most suitable to the habitat types present at the
site. Samples are preserved in ethanol in the field and shipped to a
contracting lab for analysis. For additional details, see
\href{http://data.neonscience.org/api/v0/documents/NEON.DOC.003046vC}{AOS
Protocol and Procedure: Aquatic Macroinvertebrate Sampling
(NEON.DOC.003046)} and
\href{http://data.neonscience.org/api/v0/documents/NEON.DOC.001152vA}{NEON
Aquatic Sampling Strategy (NEON.DOC.001152)}.

Queries for this data product return a downloadable data package with
laboratory methods and DNA extraction, PCR amplification, and sequencing
metadata for samples from the queried sites and date range. The actual
sequence data are publicly available and may be queried on the
\href{http://metagenomics.anl.gov/}{Metagenomics Rapid Annotation using
Subsystem Technology (MG-RAST)} server. There may be lags between
publication of metadata on the NEON data portal and availability of
sequence data on the public sequence repository. Sequence data may also
be obtained by querying NEON data sets at the
\href{https://www.ncbi.nlm.nih.gov/sra}{NCBI Sequence Read Archive (NCBI
SRA)} and the \href{https://www.ebi.ac.uk/}{European Bioinformatics
Institute (EMBL-EBI)}.

\begin{center}\rule{0.5\linewidth}{\linethickness}\end{center}

\textbf{Usage Notes}

Queries for this data product will return all data for zoo\_fieldData,
zoo\_dnaExtraction, zoo\_pcrAmplification, and zoo\_markerGeneSequencing
during the date range specified. For each successful collection of a
genetic sample in zoo\_fieldData, a unique sampleID is created, with one
sampleID per location per collect date (day of year, local time).
zoo\_fieldData.sampleIDs are shipped to an external facility where they
are subsampled into a portion for high-throughput sequencing analysis
(zoo\_dnaExtraction.geneticSampleID=zoo\_fieldData.sampleID), and the
remainder sent to archive. The protocol specifies that each
zoo\_dnaExtraction.geneticSampleID yields one
zoo\_dnaExtraction.dnaSampleID, except where multiple extractions are
necessary. Each dnaSampleID from zoo\_dnaExtraction should yield at
least one record in each of zoo\_pcrAmplification and
zoo\_markerGeneSequencing (one record expected per dnaSampleID/replicate
combination). Duplicates may exist where protocol and/or data entry
aberrations have occurred; users should check data carefully for
anomalies before analyzing data. \newpage
.

\section{DP1.20138.001 Surface water microbe cell
count}\label{dp1.20138.001-surface-water-microbe-cell-count}

\begin{center}\rule{0.5\linewidth}{\linethickness}\end{center}

\textbf{Subsystem}

Aquatic Observation System (AOS)

\begin{center}\rule{0.5\linewidth}{\linethickness}\end{center}

\textbf{Coverage}

Measured at all NEON aquatic sites (wadeable streams, lakes, and
non-wadeable streams).

\begin{center}\rule{0.5\linewidth}{\linethickness}\end{center}

\textbf{Description}

Cell counts from surface water microbial collection in lakes, wadeable
streams, and non-wadeable streams

\begin{center}\rule{0.5\linewidth}{\linethickness}\end{center}

\textbf{Abstract}

This data product contains the quality-controlled, native sampling
resolution data from NEON's Surface water microbe cell count sample
collection. Field samples are collected using a sterilized grab sampler
in the water column of wadeable streams, rivers, and lakes in
conjunction with standard recurrent surface water chemistry samples.
Cell count field samples are collected 12 times per year in streams and
6 times per year in lakes and rivers, and are collected year-round
unless ice cover is too thick to allow sampling. Samples are preserved
in the field and shipped to a contractacting lab for analysis. For
additional details, see
\href{http://data.neonscience.org/api/v0/documents/NEON.DOC.003044vB}{NEON.DOC.003044}:
AOS Protocol and Procedure: Aquatic Microbial Sampling and
\href{http://data.neonscience.org/api/v0/documents/NEON.DOC.001152vA}{NEON.DOC.001152}:
NEON Aquatic Sampling Strategy.

\begin{center}\rule{0.5\linewidth}{\linethickness}\end{center}

\textbf{Usage Notes}

Queries for this data product will return all data for
amc\_fieldSuperParent, amc\_fieldCellCounts, and amc\_cellCounts
collected during the date range specified. The expanded package also
returns quality information in amc\_cellCountLabSummary. The protocol
dictates that microbial cell count samples are collected with surface
water samples which share metadata in the fieldSuperParent table. At
lake and river sites, sampling depths are derived from data in
dep\_profileHeader (see Depth profile data product). Duplicates may
exist where protocol and/or data entry abberations have occurred; users
should check data carefully for anomalies before analyzing data.
\newpage
.

\section{DP1.20141.001 Surface water microbe community
composition}\label{dp1.20141.001-surface-water-microbe-community-composition}

\begin{center}\rule{0.5\linewidth}{\linethickness}\end{center}

\textbf{Subsystem}

Aquatic Observation System (AOS)

\begin{center}\rule{0.5\linewidth}{\linethickness}\end{center}

\textbf{Coverage}

These data are collected at all NEON aquatic sites (wadeable streams,
lakes, and non-wadeable streams).

\begin{center}\rule{0.5\linewidth}{\linethickness}\end{center}

\textbf{Description}

Counts and relative abundances of archaeal, bacterial, and fungal taxa
observed in surface water microbial communities in lakes, non-wadeable
streams, and wadeable streams

\begin{center}\rule{0.5\linewidth}{\linethickness}\end{center}

\textbf{Abstract}

This data product contains the quality-controlled laboratory data and
metadata for NEON bacterial, archaeal, and fungal community composition
data derived from surface water microbial sampling in lakes, and
wadeable and non-wadeable streams. Taxon tables are derived from the 16S
and ITS marker gene sequencing data product, NEON.DP1.20282. Taxonomic
data are generated from quality-filtered sequence data using standard
bioinformatics software. For additional details about sampling methods
and design, see
\href{http://data.neonscience.org/api/v0/documents/NEON.DOC.003044vB}{NEON.DOC.003044}:
AOS Protocol and Procedure: Aquatic Microbial Sampling; and
\href{http://data.neonscience.org/api/v0/documents/NEON.DOC.001152vA}{NEON.DOC.001152}:
NEON Aquatic Sampling Strategy.

\begin{center}\rule{0.5\linewidth}{\linethickness}\end{center}

\textbf{Usage Notes}

Queries for the basic download data product will return data from
amc\_fieldSuperParent, amc\_fieldGenetic, mcc\_swTaxonTableMetadata\_16S
and mcc\_swTaxonTableMetadata\_ITS for all dates within the specified
date range. In addition, the expanded download also includes per-sample
taxon tables for all dates within the specified date range and data from
all dates for the mcc\_taxonTableLabSummary. A given dnaSampleID for
mcc\_swTaxonTableMetadata\_16S(ITS) is expected to generate one record
for each completeTaxonomy, and the number of records per sample should
equal the number of unique values for completeTaxonomy. Duplicate
samples and/or missing data may exist where protocol and/or data entry
aberrations have occurred; users should check data carefully for
anomalies before joining tables. \newpage
.

\section{DP1.20163.001 Periphyton, seston, and phytoplankton chemical
properties}\label{dp1.20163.001-periphyton-seston-and-phytoplankton-chemical-properties}

\begin{center}\rule{0.5\linewidth}{\linethickness}\end{center}

\textbf{Subsystem}

Aquatic Observation System (AOS)

\begin{center}\rule{0.5\linewidth}{\linethickness}\end{center}

\textbf{Coverage}

Measured at all NEON aquatic sites (wadeable streams, lakes, and
non-wadeable streams).

\begin{center}\rule{0.5\linewidth}{\linethickness}\end{center}

\textbf{Description}

C, N, P, isotopes, chlorophyll a, and pheophytin of periphyton, seston,
and phytoplankton from benthic and water column samples in lakes,
non-wadeable streams, and wadeable streams

\begin{center}\rule{0.5\linewidth}{\linethickness}\end{center}

\textbf{Abstract}

This data product contains the quality-controlled, native sampling
resolution data from NEON's aquatic periphyton, seston, and
phytoplankton chemical analyses provided by a contracted lab. Benthic
and water column field samples are collected in wadeable streams,
rivers, and lakes, three times per year during the growing season using
the type of sampler most suitable to the habitat and substratum types
present at the site. Samples are processed at the domain support
facility and separated into aliquots and filtered onto glass-fiber
filters for chlorophyll, pheophytin, carbon, nitrogen, phosphorus, and
carbon, nictrogen, and sulfur isotopes for analysis at an external
facility. For additional details, see
\href{http://data.neonscience.org/api/v0/documents/NEON.DOC.003045vB}{NEON.DOC.003045}
AOS Protocol and Procedure: Periphyton, Seston, and Phytoplankton
Sampling and
\href{http://data.neonscience.org/api/v0/documents/NEON.DOC.001152vA}{NEON.DOC.001152}:
NEON Aquatic Sampling Strategy.

\begin{center}\rule{0.5\linewidth}{\linethickness}\end{center}

\textbf{Usage Notes}

Queries for this data product will return all data for alg\_fieldData,
alg\_domainLabChemistry, and alg\_algaeExternalLabDataPerSample during
the date range specified. Each record in alg\_fieldData and
alg\_algaeExternalLabDataPerSample may have several child records in
alg\_domainLabChemistry and alg\_algaeExternalLabDataPerSample. One
unique child record is created for each sampleID - analyte -
filterNumber combination. The expanded package also returns summary data
for each analytical method from the contractor in
asi\_externalLabPOMSummaryData, with one record per date range, analyte,
instrument, and method. Duplicates may exist where protocol and/or data
entry abberations have occurred; users should check data carefully for
anomalies before analyzing data. \newpage
.

\section{DP1.20166.001 Periphyton, seston, and phytoplankton
collection}\label{dp1.20166.001-periphyton-seston-and-phytoplankton-collection}

\begin{center}\rule{0.5\linewidth}{\linethickness}\end{center}

\textbf{Subsystem}

Aquatic Observation System (AOS)

\begin{center}\rule{0.5\linewidth}{\linethickness}\end{center}

\textbf{Coverage}

These data are collected at all NEON aquatic sites (wadeable streams,
lakes, and non-wadeable streams).

\begin{center}\rule{0.5\linewidth}{\linethickness}\end{center}

\textbf{Description}

Collection and biomass of periphyton, seston, and phytoplankton using
multiple benthic and water column sampling methods in lakes,
non-wadeable streams, and wadeable streams

\begin{center}\rule{0.5\linewidth}{\linethickness}\end{center}

\textbf{Abstract}

This data product contains the quality-controlled, native sampling
resolution data from NEON's aquatic periphyton, seston, and
phytoplankton collection and field metadata, as well as associated
taxonomic, morphometric, and count analyses data provided by a
contracted lab. Benthic and water column field samples are collected in
wadeable streams, rivers, and lakes three times per year during the
growing season. Samples are processed at the domain support facility and
separated into aliquots for taxonomic analysis (preserved in
glutaraldehyde or Lugol's iodine) for shipment to an external facility,
or filtered onto glass-fiber filters for biomass (ash-free dry mass).
For additional details, see
\href{http://data.neonscience.org/api/v0/documents/NEON.DOC.003045vB}{NEON.DOC.003045}
AOS Protocol and Procedure: Periphyton, Seston, and Phytoplankton
Sampling and
\href{http://data.neonscience.org/api/v0/documents/NEON.DOC.001152vA}{NEON.DOC.001152}:
NEON Aquatic Sampling Strategy.

\begin{center}\rule{0.5\linewidth}{\linethickness}\end{center}

\textbf{Usage Notes}

Queries for this data product will return all data for alg\_fieldData,
alg\_biomass, and alg\_taxonomyProcessed during the date range
specified. Each record in alg\_fieldData may have zero to 20 child
records in alg\_biomass (initial subsampling information, including
filter volumes, AFDM measurement, and taxonomy preservation),
alg\_taxonomyRaw (raw external lab taxonomic data), and
alg\_taxonomyProcessed (processed taxonomic data). Every record in
alg\_biomass, alg\_taxonomyRaw, and alg\_taxonomyProcessed should have a
corresponding record in alg\_fieldData describing field collection
conditions, location, and metadata during sample collection. In
alg\_biomass, there should be one unique record for each sampleID -
analysisType - filterNumber combination. alg\_taxonomyRaw and
alg\_taxonomyProcessed will have multiple records per sampleID,
organized by scientificName and algalParameter. Duplicates and/or
missing data may exist where protocol and/or data entry aberrations have
occurred; users should check data carefully for anomalies before joining
tables. Taxonomic IDs of species of concern have been `fuzzed'; see data
package readme files for more information. \newpage
.

\section{DP1.20190.001 Reaeration field and lab
collection}\label{dp1.20190.001-reaeration-field-and-lab-collection}

\begin{center}\rule{0.5\linewidth}{\linethickness}\end{center}

\textbf{Subsystem}

Aquatic Observation System (AOS)

\begin{center}\rule{0.5\linewidth}{\linethickness}\end{center}

\textbf{Coverage}

This data product is measured at NEON wadeable stream sites.

\begin{center}\rule{0.5\linewidth}{\linethickness}\end{center}

\textbf{Description}

Field and external laboratory data from the salt-tracer and gas
injection field reaeration measurements, including stream widths, inert
gas concentrations, gas loss rate calculations, and travel time
calculations.

\begin{center}\rule{0.5\linewidth}{\linethickness}\end{center}

\textbf{Abstract}

This data product contains the quality-controlled, native sampling
resolution data from NEON's wadeable stream reaeration sampling
protocol. Grab samples of stream water at NEON aquatic sites are
collected in streams at 4 sampling locations downstream of a continuous
injection of an inert gas (SF6) and conservative tracer (NaCl or NaBr).
Background samples are collected prior to tracer injection and are
analyzed for background salt tracer concentrations. Plateau samples are
collected once the tracer concentration reaches a constant concentration
(as measured by conductivity) and 5 replicate samples from each station
are analyzed for both salt and gas tracer concentrations. Data users
should refer to the user guide for reaeration and salt-based discharge
(\href{http://data.neonscience.org/api/v0/documents/NEON_ReaerSaltBasedQ_userGuide_vA}{NEON\_ReaerSaltBasedQ\_userGuide\_vA})
for suggestions on how to calculate reaeration rates from the published
data packages. For additional details on NEON field and laboratory
protocols, see the AOS Protocol and Procedure: Reaeration in Streams
\href{http://data.neonscience.org/api/v0/documents/NEON.DOC.000693vH}{NEON.DOC.000693}.

\begin{center}\rule{0.5\linewidth}{\linethickness}\end{center}

\textbf{Usage Notes}

Queries for this data product will return data from the date range
specified for rea\_fieldData, rea\_backgroundFieldSaltData,
rea\_backgroundFieldCondData, rea\_plateauMeasurementFieldData,
rea\_plateauSampleFieldData, rea\_widthFieldData,
rea\_externalLabDataGas, rea\_externalLabDataSalt and
rea\_conductivityFieldData. The rea\_externalLabSummaryData from all
dates and from relevant external laboratories will be returned with the
expanded package. The protocol dictates that the tracer injection will
take place at each siteID per event (one record expected per siteID and
collectedDate combination in rea\_fieldData). A record from
rea\_fieldData will usually have 4 child records in
rea\_backgroundFieldSaltData and rea\_plateauMeasurementFieldData (one
fore each sampling station), two child records in
rea\_backgroundFieldCondData (one for the upstream, station \#1 and
downstream, station \#4 stations where conductivity loggers are
deployed), many child records in rea\_conductivityFieldData (logger
conductivity data is collected and a record is created every 10 seconds
during the duration of the injection experiment). 30 ecords are created
in rea\_widthFieldData (one for each wetted width measurement) for each
injection experiment. Each record from rea\_plateauMeasurementFieldData
is expected to have 5 child records in rea\_plateauSampleFieldData (one
fore each replicate collected at the station). Each record from
rea\_backgroundFieldSaltData (background samples),
rea\_plateauMeasurementFieldData (plateau samples), and rea\_fieldData
(injectate sample) is expected to have one child record in
rea\_externalLabDataSalt with the salt tracer concentration. Each record
from rea\_plateauMeasurementFieldData is expected to have one child
record in rea\_externalLabDataGas with the gas tracer concentration.
However, duplicates and/or missing data may exist where protocol and/or
data entry aberrations have occurred; users should check data carefully
for anomalies before joining tables. \newpage
.

\section{DP1.20191.001 Riparian vegetation \%
cover}\label{dp1.20191.001-riparian-vegetation-cover}

\begin{center}\rule{0.5\linewidth}{\linethickness}\end{center}

\textbf{Subsystem}

Aquatic Observation System (AOS)

\begin{center}\rule{0.5\linewidth}{\linethickness}\end{center}

\textbf{Coverage}

Measured at all NEON aquatic wadeable stream sites.

\begin{center}\rule{0.5\linewidth}{\linethickness}\end{center}

\textbf{Description}

Assessment of riparian vegetation percent cover in wadeable streams

\begin{center}\rule{0.5\linewidth}{\linethickness}\end{center}

\textbf{Abstract}

This data product contains the quality-controlled, native sampling
resolution data from the stream canopy cover component of NEON's
riparian habitat assessment protocol. Using a modified convex
densiometer, 1/3 (17 points) of the view field is used to measure the
riparian canopy cover at wadeable stream sites. For additional details,
see protocol
\href{http://data.neonscience.org/api/v0/documents/NEON.DOC.003826vB}{NEON.DOC.
003826}: AOS Protocol and Procedure: Riparian Habitat Assessment and
science design
\href{http://data.neonscience.org/api/v0/documents/NEON.DOC.001152vA}{NEON.DOC.001152}:
Aquatic Sampling Design

\begin{center}\rule{0.5\linewidth}{\linethickness}\end{center}

\textbf{Usage Notes}

Queries for this data product will return all data for
rip\_percentComposition that was collected during the date range
specified. The protocol dictates that at streams each riparian transect
has 6 observations per year (one from each bank looking perpendicular to
the stream flow, and 4 from the center of the stream, looking upstream,
downstream and towards each bank). Duplicates may exist where protocol
and/or data entry aberrations have occurred; users should check data
carefully for anomalies before analyzing data. \newpage
.

\section{DP1.20193.001 Salt-based stream
discharge}\label{dp1.20193.001-salt-based-stream-discharge}

\begin{center}\rule{0.5\linewidth}{\linethickness}\end{center}

\textbf{Subsystem}

Aquatic Observation System (AOS)

\begin{center}\rule{0.5\linewidth}{\linethickness}\end{center}

\textbf{Coverage}

This data product is measured at NEON aquatic wadeable stream sites.

\begin{center}\rule{0.5\linewidth}{\linethickness}\end{center}

\textbf{Description}

Discharge measured using a constant-rate addition salt tracer during
reaeration measurements

\begin{center}\rule{0.5\linewidth}{\linethickness}\end{center}

\textbf{Abstract}

This data product contains the quality-controlled, native sampling
resolution data for NEON's Salt-based Discharge data product. The data
for this data product is collected as part of the wadeable stream
reaeration sampling protocol. Briefly, grab samples of stream water at
NEON aquatic sites are collected in streams at 4 sampling locations
downstream of a continuous injection of a conservative tracer (NaCl or
NaBr). Background samples are collected prior to tracer injection and
are analyzed for background salt tracer concentrations. Plateau samples
are collected once the tracer concentration reaches a constant
concentration (as measured by conductivity) and 5 replicate samples from
each station are analyzed for salt tracer concentrations. Data users
should refer to the user guide for reaeration and salt-based discharge
(\href{http://data.neonscience.org/api/v0/documents/NEON_ReaerSaltBasedQ_userGuide_vA}{NEON\_ReaerSaltBasedQ\_userGuide\_vA})
for suggestions on how to calculate discharge values from the published
data packages. For additional details on NEON field and laboratory
protocols, see the AOS Protocol and Procedure: Reaeration in Streams
\href{http://data.neonscience.org/api/v0/documents/NEON.DOC.000693vH}{NEON.DOC.000693}.

\begin{center}\rule{0.5\linewidth}{\linethickness}\end{center}

\textbf{Usage Notes}

Queries for this data product will return data from the date range
specified for sbd\_fieldData, sbd\_backgroundFieldSaltData,
sbd\_backgroundFieldCondData, sbd\_plateauMeasurementFieldData,
sbd\_plateauSampleFieldData, sbd\_externalLabDataGas,
sbd\_externalLabDataSalt and sbd\_conductivityFieldData. The
rea\_externalLabSummaryData from all dates and from relevant external
laboratories will be returned with the expanded package. All of the data
in the salt based discharge download package is also part of the
reaeration download package. However, the reaeration data package
contains tables and fields for the inert, volatile gas tracer injection
that are not included in the salt-based discharge download package. The
protocol dictates that the tracer injection will take place at each
siteID per event (one record expected per siteID and collectedDate
combination in sbd\_fieldData). A record from sbd\_fieldData will
usually have 4 child records in sbd\_backgroundFieldSaltData and
sbd\_plateauMeasurementFieldData (one fore each sampling station), two
child records in sbd\_backgroundFieldCondData (one for the upstream,
station \#1 and downstream, station \#4 stations where conductivity
loggers are deployed), many child records in sbd\_conductivityFieldData
(logger conductivity data is collected and a record is created every 10
seconds during the duration of the injection experiment). 30 ecords are
created in sbd\_widthFieldData (one for each wetted width measurement)
for each injection experiment. Each record from
sbd\_plateauMeasurementFieldData is expected to have 5 child records in
sbd\_plateauSampleFieldData (one fore each replicate collected at the
station). Each record from sbd\_backgroundFieldSaltData (background
samples), sbd\_plateauMeasurementFieldData (plateau samples), and
sbd\_fieldData (injectate sample) is expected to have one child record
in sbd\_externalLabDataSalt with the salt tracer concentration. However,
duplicates and/or missing data may exist where protocol and/or data
entry aberrations have occurred; users should check data carefully for
anomalies before joining tables. \newpage
.

\section{DP1.20194.001 Sediment chemical
properties}\label{dp1.20194.001-sediment-chemical-properties}

\begin{center}\rule{0.5\linewidth}{\linethickness}\end{center}

\textbf{Subsystem}

Aquatic Observation System (AOS)

\begin{center}\rule{0.5\linewidth}{\linethickness}\end{center}

\textbf{Coverage}

Measured at all NEON aquatic sites (wadeable streams, lakes, and
non-wadeable streams).

\begin{center}\rule{0.5\linewidth}{\linethickness}\end{center}

\textbf{Description}

Inorganic, organic, and organic contaminant analyses of wadeable stream,
non-wadeable stream, and lake bed sediments

\begin{center}\rule{0.5\linewidth}{\linethickness}\end{center}

\textbf{Abstract}

This data product contains the quality-controlled, native sampling
resolution data from NEON's aquatic sediment collection and field
metadata, as well as associated chemical data (Inorganic, organic, and
metal analyses) provided by a contracted lab. Sediment field samples are
collected in wadeable streams, rivers, and lakes two times per year
during the growing season. Samples are homogenized from several
depositional zones, and prepared for shipment to an external facility.
For additional details, see
\href{http://data.neonscience.org/api/v0/documents/NEON.DOC.001191vD}{NEON.DOC.001191:
AOS Protocol and Procedure: Sediment Chemistry Sampling in Lakes and
Non-Wadeable Streams}, or
\href{http://data.neonscience.org/api/v0/documents/NEON.DOC.001193vE}{NEON.DOC.001193:
AOS Protocol and Procedure: Sediment Chemistry Sampling in Wadeable
Streams} and
\href{http://data.neonscience.org/api/v0/documents/NEON.DOC.001152vA}{NEON.DOC.001152:
NEON Aquatic Sampling Strategy}.

\begin{center}\rule{0.5\linewidth}{\linethickness}\end{center}

\textbf{Usage Notes}

Queries for this data product will return all data for
asc\_fieldDataStation, asc\_fieldDataZone, asc\_fieldDataPoint,
asc\_externalLabData\_pub, and asc\_externalLabSummary\_pub during the
date range specified. There should be one record in
asc\_fieldDataStation per sedimentSampleID. For each record in
asc\_fieldDataStation that does not list samplingImpractical, there may
be 1 to 5 (or more if sediment deposition is scarce) child records in
asc\_fieldDataZone (field metadata at the zone level, including habitat
and sampler type used, and GPS locations of the depositional zones).
Each record in asc\_fieldDataZone may have 1 to 7 (or more if sediment
deposition is scarce) records in asc\_fieldDataPoint (field metadata at
the point level, including water quality parameters such as depth,
temperature, dissolved oxygen and conductivity). Each record in
asc\_fieldDataStation may have several records in asc\_externalLabData,
one per analyte. All field tables can be linked by sedimentSampleID. The
sampleID in asc\_externalLabData can be linked to one of the
corresponding child records in asc\_fieldDataStation:
inorganicSedimentSampleID, organicSedimentSampleID,
carbonSedimentSampleID or physicalSedimentSampleID. Duplicates and/or
missing data may exist where protocol and/or data entry aberrations have
occurred; users should check data carefully for anomalies before joining
tables. \newpage
.

\section{DP1.20197.001 Sediment physical
properties}\label{dp1.20197.001-sediment-physical-properties}

\begin{center}\rule{0.5\linewidth}{\linethickness}\end{center}

\textbf{Subsystem}

Aquatic Observation System (AOS)

\begin{center}\rule{0.5\linewidth}{\linethickness}\end{center}

\textbf{Coverage}

Measured at all NEON aquatic sites (wadeable streams, lakes, and
non-wadeable streams).

\begin{center}\rule{0.5\linewidth}{\linethickness}\end{center}

\textbf{Description}

Size analysis of wadeable stream, non-wadeable stream, and lake bed
sediments

\begin{center}\rule{0.5\linewidth}{\linethickness}\end{center}

\textbf{Abstract}

This data product contains the quality-controlled, native sampling
resolution data from NEON's aquatic sediment collection and field
metadata, as well as associated physical data (size analyses) provided
by a contracted lab. Sediment field samples are collected in wadeable
streams, rivers, and lakes two times per year during the growing season.
Samples are homogenized from several depositional zones, and prepared
for shipment to an external facility. For additional details, see
\href{http://data.neonscience.org/api/v0/documents/NEON.DOC.001191vD}{NEON.DOC.001191:
AOS Protocol and Procedure: Sediment Chemistry Sampling in Lakes and
Non-Wadeable Streams}, or
\href{http://data.neonscience.org/api/v0/documents/NEON.DOC.001193vE}{NEON.DOC.001193:
AOS Protocol and Procedure: Sediment Chemistry Sampling in Wadeable
Streams} and
\href{http://data.neonscience.org/api/v0/documents/NEON.DOC.001152vA}{NEON.DOC.001152:
NEON Aquatic Sampling Strategy}.

\begin{center}\rule{0.5\linewidth}{\linethickness}\end{center}

\textbf{Usage Notes}

Queries for this data product will return all data for
asc\_fieldDataStation, asc\_fieldDataZone, asc\_fieldDataPoint,
asc\_externalLabData\_pub, and asc\_externalLabSummary\_pub during the
date range specified. There should be one record in
asc\_fieldDataStation per sedimentSampleID. For each record in
asc\_fieldDataStation that does not list samplingImpractical, there may
be 1 to 5 (or more if sediment deposition is scarce) child records in
asc\_fieldDataZone (field metadata at the zone level, including habitat
and sampler type used, and GPS locations of the depositional zones).
Each record in asc\_fieldDataZone may have 1 to 7 (or more if sediment
deposition is scarce) records in asc\_fieldDataPoint (field metadata at
the point level, including water quality parameters such as depth,
temperature, dissolved oxygen and conductivity). Each record in
asc\_fieldDataStation may have several records in asc\_externalLabData,
one per analyte. All field tables can be linked by sedimentSampleID. The
sampleID in asc\_externalLabData can be linked to one of the
corresponding child records in asc\_fieldDataStation:
inorganicSedimentSampleID, organicSedimentSampleID,
carbonSedimentSampleID or physicalSedimentSampleID. Duplicates and/or
missing data may exist where protocol and/or data entry aberrations have
occurred; users should check data carefully for anomalies before joining
tables. \newpage
.

\section{DP1.20206.001 Stable isotope concentrations in surface
waters}\label{dp1.20206.001-stable-isotope-concentrations-in-surface-waters}

\begin{center}\rule{0.5\linewidth}{\linethickness}\end{center}

\textbf{Subsystem}

Aquatic Observation System (AOS)

\begin{center}\rule{0.5\linewidth}{\linethickness}\end{center}

\textbf{Coverage}

Measured at all NEON aquatic sites (wadeable streams, non-wadeable
rivers, and lakes).

\begin{center}\rule{0.5\linewidth}{\linethickness}\end{center}

\textbf{Description}

Grab samples for stable isotope chemistry including water and organic
matter, in lakes, non-wadeable streams, and wadeable streams

\begin{center}\rule{0.5\linewidth}{\linethickness}\end{center}

\textbf{Abstract}

This data product contains the quality-controlled, native sampling
resolution data from NEON's stable isotope concentrations in surface
water sampling protocol. Filters containing suspended particulate
organic matter (POM) are sent to external facilities for analysis to
determine 15N/14N and 13C/12C isotope ratios. Water samples are sent to
external facilities for analysis to determine 18O/12O and 2H/1H water
isotope ratios. For additional details on NEON field and laboratory
protocols, see the AOS Protocol and Procedure: Stable Isotope Sampling
in Surface and Ground Waters
(\href{(http://data.neonscience.org/api/v0/documents/NEON.DOC.001886vE)}{NEON.DOC.001886}.

\begin{center}\rule{0.5\linewidth}{\linethickness}\end{center}

\textbf{Usage Notes}

The protocol dictates that each siteID x stationID combination is
sampled at least once per event (one record expected per parentSampleID
in asi\_fieldSuperParent). A record from asi\_fieldSuperParent may have
zero or one child records in asi\_fieldData, depending on whether a
water sample was collected. In the event that a water sample cannot be
taken, a record will still be created in asi\_fieldSuperParent, and
asi\_fieldSuperParent.samplingImpractical will be something other than
NULL, but there will be no corresponding record in asi\_fieldData. Each
record from asi\_fieldData is expected to have a child record in
asi\_externalLabH2OIsotopes and asi\_POMExternalLabDataPerSample.
However, duplicates and/or missing data may exist where protocol and/or
data entry aberrations have occurred; users should check data carefully
for anomalies before joining tables. \newpage
.

\section{DP1.20217.001 Temperature of
groundwater}\label{dp1.20217.001-temperature-of-groundwater}

\begin{center}\rule{0.5\linewidth}{\linethickness}\end{center}

\textbf{Subsystem}

Aquatic Instrument System (AIS)

\begin{center}\rule{0.5\linewidth}{\linethickness}\end{center}

\textbf{Coverage}

These data are collected in the fall and spring at all NEON aquatic
sites except for MCRA, CUPE, and TECR where there are no groundwater
wells.

\begin{center}\rule{0.5\linewidth}{\linethickness}\end{center}

\textbf{Description}

Sensor based measurement of groundwater temperature in each well.

\begin{center}\rule{0.5\linewidth}{\linethickness}\end{center}

\textbf{Abstract}

Groundwater plays an important role in modulating temperature of surface
water, which is critical to habitat quality and ecosystem function. NEON
measures groundwater temperature at high temporal resolution. Three to
eight wells are available per aquatic site. From NEON groundwater
elevation measurements, the magnitude and direction of groundwater flow
can be calculated, which will help to inform the heat flux between
groundwater and surface water. This data product includes continuous
quality-controlled groundwater temperature captured every 5 minutes and
are reported as 5-minute instantaneous measurements and 30-minute
averages. \newpage
.

\section{DP1.20219.001 Zooplankton
collection}\label{dp1.20219.001-zooplankton-collection}

\begin{center}\rule{0.5\linewidth}{\linethickness}\end{center}

\textbf{Subsystem}

Aquatic Observation System (AOS)

\begin{center}\rule{0.5\linewidth}{\linethickness}\end{center}

\textbf{Coverage}

These data are collected at all NEON aquatic lake sites.

\begin{center}\rule{0.5\linewidth}{\linethickness}\end{center}

\textbf{Description}

Collection of zooplankton from water column samples in lakes

\begin{center}\rule{0.5\linewidth}{\linethickness}\end{center}

\textbf{Abstract}

This data product contains the quality-controlled, native sampling
resolution data and metadata from NEON's aquatic zooplankton collection
protocol, as well as associated taxonomic, morphometric, and count
analyses data provided by a contracted lab. Field samples are collected
in the water column of lakes using the most appropriate sampler
(vertical tow net or Schindler trap) for the depth of water, preserved
in ethanol in the field, and shipped to a contracting lab for analysis.
For additional details, see
\href{http://data.neonscience.org/api/v0/documents/NEON.DOC.001194vF}{NEON.DOC.001194}
AOS Protocol and Procedure: Zooplankton Sampling in Lakes and
\href{http://data.neonscience.org/api/v0/documents/NEON.DOC.001152vA}{NEON.DOC.001152}:
NEON Aquatic Sampling Strategy.

\begin{center}\rule{0.5\linewidth}{\linethickness}\end{center}

\textbf{Usage Notes}

Queries for this data product will return all data for zoo\_fieldData
and zoo\_taxonomyProcessed during the date range specified. If sampling
is not impractical, each record for in zoo\_fieldData may have multiple
corresponding records in zoo\_taxonomyRaw and zoo\_taxonomyProcessed,
one record for each scientificName per sampleID. A record from
zoo\_fieldData may have multiple or no records in zoo\_perVial, as that
table represents individuals removed from the final archived sample and
placed in the external lab's in-house reference collection, records in
this table are opportunistic. The expanded package also returns raw
taxonomic data from the external taxonomist in inv\_taxonomyRaw and
information on the contents of the vial sent to the archive facility in
inv\_perVial. Duplicates may exist where protocol and/or data entry
aberrations have occurred; users should check data carefully for
anomalies before analyzing data. Taxonomic IDs of species of concern
have been `fuzzed'; see data package readme files for more information.
\newpage
.

\section{DP1.20221.001 Zooplankton DNA
barcode}\label{dp1.20221.001-zooplankton-dna-barcode}

\begin{center}\rule{0.5\linewidth}{\linethickness}\end{center}

\textbf{Subsystem}

Aquatic Observation System (AOS)

\begin{center}\rule{0.5\linewidth}{\linethickness}\end{center}

\textbf{Coverage}

These data are collected at all NEON aquatic lake sites.

\begin{center}\rule{0.5\linewidth}{\linethickness}\end{center}

\textbf{Description}

CO1 DNA sequences of the zooplankton community

\begin{center}\rule{0.5\linewidth}{\linethickness}\end{center}

\textbf{Abstract}

This data product contains the quality-controlled, native sampling
resolution data and metadata from NEON's aquatic zooplankton DNA
sampling protocol, as well as associated metadata provided by a
contracted lab. Field samples are collected in the water column of lakes
using the most appropriate sampler (vertical tow net or Schindler trap)
for the depth of water at the same time and location as morphological
taxonomy samples, preserved in ethanol in the field, and shipped to a
contracting lab for processing and sequencing. For additional details,
see
\href{http://data.neonscience.org/api/v0/documents/NEON.DOC.001194vG}{AOS
Protocol and Procedure: Zooplankton Sampling in Lakes (NEON.DOC.001194)}
and
\href{http://data.neonscience.org/api/v0/documents/NEON.DOC.001152vA}{NEON
Aquatic Sampling Strategy (NEON.DOC.001152)}.

Queries for this data product return a downloadable data package with
laboratory methods and DNA extraction, PCR amplification, and sequencing
metadata for samples from the queried sites and date range. The actual
sequence data are publicly available and may be queried on the
\href{http://metagenomics.anl.gov/}{Metagenomics Rapid Annotation using
Subsystem Technology (MG-RAST)} server. There may be lags between
publication of metadata on the NEON data portal and availability of
sequence data on the public sequence repository. Sequence data may also
be obtained by querying NEON data sets at the
\href{https://www.ncbi.nlm.nih.gov/sra}{NCBI Sequence Read Archive (NCBI
SRA)} and the \href{https://www.ebi.ac.uk/}{European Bioinformatics
Institute (EMBL-EBI)}.

\begin{center}\rule{0.5\linewidth}{\linethickness}\end{center}

\textbf{Usage Notes}

Queries for this data product will return all data for zoo\_fieldData,
zoo\_dnaExtraction, zoo\_pcrAmplification, and zoo\_markerGeneSequencing
during the date range specified. For each record collected in
zoo\_fieldData, a unique geneticSampleID is created with one sample per
collectDate per location (day of year, local time). geneticSampleIDs are
shipped to an external facility where they are subsampled into a portion
for high-throughput sequencing analysis (dnaSampleID), and the remainder
sent to archive. The protocol specifies that each
zoo\_dnaExtraction.geneticSampleID yields one
zoo\_dnaExtraction.dnaSampleID, except where multiple extractions are
necessary. Each dnaSampleID from zoo\_dnaExtraction should yield a
record in each of zoo\_pcrAmplification and zoo\_markerGeneSequencing
per replicate (one record expected per dnaSampleID/replicate
combination). Duplicates may exist where protocol and/or data entry
aberrations have occurred; users should check data carefully for
anomalies before analyzing data \newpage
.

\section{DP1.20252.001 Secchi depth}\label{dp1.20252.001-secchi-depth}

\begin{center}\rule{0.5\linewidth}{\linethickness}\end{center}

\textbf{Subsystem}

Aquatic Observation System (AOS)

\begin{center}\rule{0.5\linewidth}{\linethickness}\end{center}

\textbf{Coverage}

Measured at all NEON lake and non-wadeable stream sites.

\begin{center}\rule{0.5\linewidth}{\linethickness}\end{center}

\textbf{Description}

Measurement of water column Secchi depth in non-wadeable streams and
lakes

\begin{center}\rule{0.5\linewidth}{\linethickness}\end{center}

\textbf{Abstract}

This data product contains the quality-controlled, native sampling
resolution data from NEON's Secchi depth data collection. Secchi
measurements indicate water clarity, and secchi depth is used to
determine the depth to which light penetrates. This value can also be
used to calculate the depth of the euphotic zone in a lake or river.
Secchi data are collected when collecting data for any standard
operating procedure that samples the water column in a lake or river by
lowering a Secchi disk through the water column, and recorded the
depth(s) to which it disappears from view. Secchi depth measurements are
collected only during ice-free periods, and are collected a minimum of 4
times per year, up to 12+ times per year. For additional details, see
\href{http://data.neonscience.org/api/v0/documents/NEON.DOC.002792vB}{NEON.DOC.002792}:
AOS Protocol and Procedure: Secchi Depth and Depth Profile Sampling in
Lakes and Non-Wadeable Streams and
\href{data.neonscience.org/api/v0/documents/NEON.DOC.001152vA}{NEON.DOC.001152}:
NEON Aquatic Sampling Strategy.

\begin{center}\rule{0.5\linewidth}{\linethickness}\end{center}

\textbf{Usage Notes}

Queries for this data product will return all data for dep\_secchi
collected during the date range specified. The protocol dictates that
secchi measurements are collected at lake and river (non-wadeable
stream) sites whenever another protocol that samples the water column is
implemented (i.e., surface water chemistry, surface water microbes,
phytoplankton, zooplankton) during ice-free periods, which results in a
minimum of 4 data points per year collected near the sensor buoy
location. Each record in dep\_secchi corresponds to one record in
dep\_profileHeader (see related data products). Duplicates may exist
where protocol and/or data entry aberrations have occurred; users should
check data carefully for anomalies before analyzing data. \newpage
.

\section{DP1.20254.001 Depth profile at specific
depths}\label{dp1.20254.001-depth-profile-at-specific-depths}

\begin{center}\rule{0.5\linewidth}{\linethickness}\end{center}

\textbf{Subsystem}

Aquatic Observation System (AOS)

\begin{center}\rule{0.5\linewidth}{\linethickness}\end{center}

\textbf{Coverage}

Measured at all NEON lake and non-wadeable stream sites.

\begin{center}\rule{0.5\linewidth}{\linethickness}\end{center}

\textbf{Description}

Measurements of water column temperature and depth profile in
non-wadeable streams and lakes

\begin{center}\rule{0.5\linewidth}{\linethickness}\end{center}

\textbf{Abstract}

This data product contains the quality-controlled, native sampling
resolution data from NEON's Depth profile at specific depths data
collection. Depth profile data are collected along with anyother
sampling in the water column. Depth profile data include water
temperature, conductivity, and dissolved oxygen data collected every 0.5
m through the water column using a handheld proble. Depth profile data
are only collected at the deepest location of the lake (buoy) or near
the non-wadeable stream sensor set. These data not only provide metadata
to accompany the sampling modules, but also inform sampling depths based
the thermocline (if present) for water chemistry and associated
analytes, surface water microbes, and phytoplankton sampling. Depth
profiles are collected year-round, including under ice at northern
sites, a minimum of 12 times per year. For additional details, see
{[}NEON.DOC.002792{]}: AOS Protocol and Procedure: Secchi Depth and
Depth Profile Sampling in Lakes and Non-Wadeable Streams and
{[}NEON.DOC.001152{]}: NEON Aquatic Sampling Strategy.

\begin{center}\rule{0.5\linewidth}{\linethickness}\end{center}

\textbf{Usage Notes}

Queries for this data product will return data from dep\_profileHeader
and dep\_profileData collected during the date range specified. Each
record in dep\_profileHeader corresponds to several records in
dep\_profileData; the number of profile measurements per event depends
on lake or river depth. Each record in dep\_profileHeader also
corresponds to one record in dep\_secchi (see related data products).
The protocol dictates that depth profile measurements are collected at
lake and river sites whenever another protocol that samples the water
column is implemented (i.e., surface water chemistry, surface water
microbes, phtyoplankton, zooplankton), and can be joined to those data
by site and date. Duplicates may exist where protocol and/or data entry
abberations have occurred; users should check data carefully for
anomalies before analyzing data. \newpage
.

\section{DP1.20261.001 Photosynthetically active radiation below water
surface}\label{dp1.20261.001-photosynthetically-active-radiation-below-water-surface}

\begin{center}\rule{0.5\linewidth}{\linethickness}\end{center}

\textbf{Subsystem}

Aquatic Instrument System (AIS)

\begin{center}\rule{0.5\linewidth}{\linethickness}\end{center}

\textbf{Coverage}

Inlet and outlet sensor sets are at all lake sites within NEON. Buoys
will be deployed at all~lake and large river sites within NEON.

\begin{center}\rule{0.5\linewidth}{\linethickness}\end{center}

\textbf{Description}

Photosynthetically Active Radiation (PAR) observations represent the
radiation flux at wavelengths between 400-700 nm, which constitute the
wavelengths that drive photosynthesis. This data product is available as
one- and thirty-minute averages. Observations are made at the aquatic
sensor sets at the lake inlet, lake outlet, and lake buoy.

\begin{center}\rule{0.5\linewidth}{\linethickness}\end{center}

\textbf{Abstract}

Photosynthetically active radiation below water is measured at 1 Hz at
lake inlet and outlet sensor sets and twice per minute on buoys at lake
and river sites. It is reported as 1-minute mean measurements and
30-minute mean values.

\begin{center}\rule{0.5\linewidth}{\linethickness}\end{center}

\textbf{Usage Notes}

Due to the nature of the floating platform, below water PAR sensors on
the buoy will not meet the following requirement at this time:
NEON.AIS.4.1334 ``All radiation sensors shall be mounted to remain level
to within ±1°''. \newpage
.

\section{DP1.20264.001 Temperature at specific depth in surface
water}\label{dp1.20264.001-temperature-at-specific-depth-in-surface-water}

\begin{center}\rule{0.5\linewidth}{\linethickness}\end{center}

\textbf{Subsystem}

Aquatic Instrument System (AIS)

\begin{center}\rule{0.5\linewidth}{\linethickness}\end{center}

\textbf{Coverage}

Buoys will be deployed at all~lake and large river sites within NEON.

\begin{center}\rule{0.5\linewidth}{\linethickness}\end{center}

\textbf{Description}

Sensor based measurements of water temperature in lake and river sites.
Temperature is measured at specific depths by a fixed-length
buoy-mounted array of temperature sensors.

\begin{center}\rule{0.5\linewidth}{\linethickness}\end{center}

\textbf{Abstract}

Temperature at specific depths on buoys is measured every minute and is
reported as 1-minute instantaneous measurements and 30-minute mean
values. A temperature chain with between 3 and 10 thermistors (depending
on lake depth) is affixed from the buoy at the water surface. The
shallowest thermistor is located at 5 cm below the water surface. Deeper
thermistors have their depth published with the data and vary for each
lake or river. \newpage
.

\section{DP1.20267.001 Gauge height}\label{dp1.20267.001-gauge-height}

\begin{center}\rule{0.5\linewidth}{\linethickness}\end{center}

\textbf{Subsystem}

Aquatic Instrument System (AIS)

\begin{center}\rule{0.5\linewidth}{\linethickness}\end{center}

\textbf{Coverage}

Gauge height is measured at all NEON aquatic sites.

\begin{center}\rule{0.5\linewidth}{\linethickness}\end{center}

\textbf{Description}

Gauge height, in meters, measured at lakes, wadeable streams and
non-wadeable streams.

\begin{center}\rule{0.5\linewidth}{\linethickness}\end{center}

\textbf{Abstract}

Gauge height is the height of the water surface above an established
altitude where the stage is zero. The zero level is arbitrary, but is
often close to the streambed. Gauge height at stream sites will be
related to measurements of stream discharge to formulate stage-discharge
rating curves at all NEON stream and river sites. Rating equations will
be applied to continuous surface water elevation data in order to
calculate continuous stream discharge at all NEON aquatic sites. For
additional details, see protocol
\href{http://data.neonscience.org/api/v0/documents/NEON.DOC.001085vD}{NEON.DOC.
001085}: AOS Protocol and Procedure: Stream Discharge,
\href{http://data.neonscience.org/api/v0/documents/NEON.DOC.001646vC}{NEON.DOC.001646}:
General AQU Field Metadata Sheet, and
\href{http://data.neonscience.org/api/v0/documents/NEON.DOC.001152vA}{NEON.DOC.001152}:
Aquatic Sampling Design.

\begin{center}\rule{0.5\linewidth}{\linethickness}\end{center}

\textbf{Usage Notes}

Queries for this data product will return data from gag\_fieldData
collected during the date range specified. Gauge measurements and field
metadata are collected at all aquatic sites whenever another protocol is
implemented; gauge data can be joined to other data by site, date, and
time. Duplicates may exist where protocol and/or data entry aberrations
have occurred; users should check data carefully for anomalies before
analyzing data. \newpage
.

\section{DP1.20271.001 Relative humidity above water
on-buoy}\label{dp1.20271.001-relative-humidity-above-water-on-buoy}

\begin{center}\rule{0.5\linewidth}{\linethickness}\end{center}

\textbf{Subsystem}

Aquatic Instrument System (AIS)

\begin{center}\rule{0.5\linewidth}{\linethickness}\end{center}

\textbf{Coverage}

Buoys will be deployed at all~lake and large river sites within NEON.

\begin{center}\rule{0.5\linewidth}{\linethickness}\end{center}

\textbf{Description}

Relative humidity, temperature, and dew or frost point temperature,
available as one- and thirty-minute averages of 1 Hz observations.
Observations are made by sensors located on the buoy in lakes and
rivers.

\begin{center}\rule{0.5\linewidth}{\linethickness}\end{center}

\textbf{Abstract}

Relative humidity on buoys is measured every minute and is reported as
1-minute instantaneous measurements and 30-minute mean values. Other
than the data collection frequency, this data product has the same data
streams and processing as relative humidity measured at aquatic met
stations and on the terrestrial instrument towers.

\begin{center}\rule{0.5\linewidth}{\linethickness}\end{center}

\textbf{Usage Notes}

The sensor error flag for the HMP155 on the buoy is not currently
active. The field will remain as a placeholder with no flags until the
data logger is capable of incorporating this feature. \newpage
.

\section{DP1.20275.001 Riparian composition and
structure}\label{dp1.20275.001-riparian-composition-and-structure}

\begin{center}\rule{0.5\linewidth}{\linethickness}\end{center}

\textbf{Subsystem}

Aquatic Observation System (AOS)

\begin{center}\rule{0.5\linewidth}{\linethickness}\end{center}

\textbf{Coverage}

Riparian composition and structure are measured at all NEON aquatic
sites.

\begin{center}\rule{0.5\linewidth}{\linethickness}\end{center}

\textbf{Description}

Assessment of riparian vegetation composition and physical structure in
lakes, non-wadeable streams, and wadeable streams

\begin{center}\rule{0.5\linewidth}{\linethickness}\end{center}

\textbf{Abstract}

This data product contains the quality-controlled, native sampling
resolution data from the composition and structure components of NEON's
riparian and habitat assessment protocol. This protocol provides a rapid
estimate of the riparian vegetation, human impacts, and bank
characteristics, which buffer the banks of NEON Aquatic lakes, rivers,
and streams. For additional details, see protocol
\href{http://data.neonscience.org/api/v0/documents/NEON.DOC.003826vB}{NEON.DOC.
003826}:AOS Protocol and Procedure: Riparian Habitat Assessment and
science design
\href{http://data.neonscience.org/api/v0/documents/NEON.DOC.001152vA}{NEON.DOC.001152}:
Aquatic Sampling Design.

\begin{center}\rule{0.5\linewidth}{\linethickness}\end{center}

\textbf{Usage Notes}

Queries for this data product will return all data for rip\_assessment
that was collected during the date range specified. The protocol
dictates that at lakes and rivers each riparian point has one
observation per year and at streams each riparian transect has 2
observations per year (one each at left and right banks). Duplicates may
exist where protocol and/or data entry aberrations have occurred; users
should check data carefully for anomalies before analyzing data.
Taxonomic IDs of species of concern have been `fuzzed'; see data package
readme files for more information. \newpage
.

\section{DP1.20276.001 Stable isotope concentrations in
groundwater}\label{dp1.20276.001-stable-isotope-concentrations-in-groundwater}

\begin{center}\rule{0.5\linewidth}{\linethickness}\end{center}

\textbf{Subsystem}

Aquatic Observation System (AOS)

\begin{center}\rule{0.5\linewidth}{\linethickness}\end{center}

\textbf{Coverage}

These data are collected in the fall and spring at all NEON aquatic
sites except for MCRA, CUPE, and TECR where there are no groundwater
wells.

\begin{center}\rule{0.5\linewidth}{\linethickness}\end{center}

\textbf{Description}

Grab samples for stable isotopes of water in groundwater

\begin{center}\rule{0.5\linewidth}{\linethickness}\end{center}

\textbf{Abstract}

This data product contains the quality-controlled, native sampling
resolution data from NEON's stable isotope concentrations in groundwater
sampling protocol. Water samples are sent to external facilities for
analysis to determine 18O/16O and 2H/1H water isotope ratios. For
additional details on NEON field and laboratory protocols, see the AOS
Protocol and Procedure: Stable Isotope Sampling in Surface and Ground
Waters
(\href{http://data.neonscience.org/api/v0/documents/NEON.DOC.001886vE}{NEON.DOC.001886}).

\begin{center}\rule{0.5\linewidth}{\linethickness}\end{center}

\textbf{Usage Notes}

The protocol dictates that each siteID x stationID combination is
sampled at least once per event (one record expected per parentSampleID
in gsi\_fieldSuperParent). A record from gsi\_fieldSuperParent may have
zero or one child records in gsi\_fieldData, depending on whether a
water sample was collected. In the event that a water sample cannot be
taken, a record will still be created in gsi\_fieldSuperParent, and
gsi\_fieldSuperParent.samplingImpractical will be something other than
NULL, but there will be no corresponding record in gsi\_fieldData. Each
record from gsi\_fieldData is expected to have a child record in
gsi\_externalLabH2OIsotopes. However, duplicates and/or missing data may
exist where protocol and/or data entry aberrations have occurred; users
should check data carefully for anomalies before joining tables.
\newpage
.

\section{DP1.20277.001 Benthic microbe group
abundances}\label{dp1.20277.001-benthic-microbe-group-abundances}

\begin{center}\rule{0.5\linewidth}{\linethickness}\end{center}

\textbf{Subsystem}

Aquatic Observation System (AOS)

\begin{center}\rule{0.5\linewidth}{\linethickness}\end{center}

\textbf{Coverage}

Measured at all NEON wadeable stream sites.

\begin{center}\rule{0.5\linewidth}{\linethickness}\end{center}

\textbf{Description}

Counts and relative abundances of marker genes from total archaea,
bacteria, and fungi observed by qPCR in benthic microbial communities

\begin{center}\rule{0.5\linewidth}{\linethickness}\end{center}

\textbf{Abstract}

This data product contains the quality-controlled laboratory data and
metadata for NEON's benthic bacterial, archaeal, and fungal group
abundances analysis, which are derived from benthic sampling in wadeable
streams. Benthic and water column field samples are collected in
wadeable streams, rivers, and lakes three times per year during the
growing season. For additional details, see protocol
\href{http://data.neonscience.org/api/v0/documents/NEON.DOC.003044vB}{NEON.DOC.003044}
AOS Protocol and Procedure: Aquatic Microbial Sampling and science
design
\href{http://data.neonscience.org/api/v0/documents/NEON.DOC.001152vA}{NEON.DOC.001152}
NEON Aquatic Sampling Strategy.

\begin{center}\rule{0.5\linewidth}{\linethickness}\end{center}

\textbf{Usage Notes}

Queries for this data product will return data from
mga\_benthicGroupAbundances for all dates within the specified date
range. The mga\_benthicBatchResults table will also be returned that
includes all batch-level records for the data product. The number of
records in mga\_benthicBatchResults should match the number of unique
batches of samples that have been analyzed. A given
mga\_benthicGroupAbundances.dnaSampleID is expected to occur one time.
Duplicate samples may exist where protocol and/or data entry aberrations
have occurred; users should check data carefully for anomalies before
joining tables. \newpage
.

\section{DP1.20278.001 Surface water microbe group
abundances}\label{dp1.20278.001-surface-water-microbe-group-abundances}

\begin{center}\rule{0.5\linewidth}{\linethickness}\end{center}

\textbf{Subsystem}

Aquatic Observation System (AOS)

\begin{center}\rule{0.5\linewidth}{\linethickness}\end{center}

\textbf{Coverage}

Measured at all NEON aquatic sites (wadeable streams, lakes, and
non-wadeable streams).

\begin{center}\rule{0.5\linewidth}{\linethickness}\end{center}

\textbf{Description}

Counts and relative abundances of marker genes from total archaea,
bacteria, and fungi observed by qPCR in surface water microbial
communities

\begin{center}\rule{0.5\linewidth}{\linethickness}\end{center}

\textbf{Abstract}

This data product contains the quality-controlled laboratory data and
metadata for NEON's surface water bacterial, archaeal and fungal group
abundances analysis, which are derived from surface water microbial
sampling. Surface water grab samples are filtered on 0.22 um Sterivex
capsule filters, capped and flash-frozen in the field. For additional
details, see protocol
\href{http://data.neonscience.org/api/v0/documents/NEON.DOC.003044vB}{NEON.DOC.003044}
AOS Protocol and Procedure: Aquatic Microbial Sampling and science
design
\href{http://data.neonscience.org/api/v0/documents/NEON.DOC.001152vA}{NEON.DOC.001152}
NEON Aquatic Sampling Strategy.

\begin{center}\rule{0.5\linewidth}{\linethickness}\end{center}

\textbf{Usage Notes}

ltr\_chemistrysubsampling) will be subset to data collected during the
date range specified. \newpage
.

\section{DP1.20279.001 Benthic microbe metagenome
sequences}\label{dp1.20279.001-benthic-microbe-metagenome-sequences}

\begin{center}\rule{0.5\linewidth}{\linethickness}\end{center}

\textbf{Subsystem}

Aquatic Observation System (AOS)

\begin{center}\rule{0.5\linewidth}{\linethickness}\end{center}

\textbf{Coverage}

These data are collected at all NEON wadeable stream sites.

\begin{center}\rule{0.5\linewidth}{\linethickness}\end{center}

\textbf{Description}

Metagenomic sequence data from benthic samples

\begin{center}\rule{0.5\linewidth}{\linethickness}\end{center}

\textbf{Abstract}

This data product contains the primary field and quality-controlled
laboratory metadata and QA results for NEON's shotgun metagenomic
sequences derived from benthic microbial sampling in wadeable streams.
Benthic samples are collected concurrently with stream periphyton
samples. Cobble scrubs are are filtered on 0.22 um Sterivex capsule
filters, capped and flash-frozen in the field. Grab samples of sediment
(silt, sand) or plant material/epiphyton are collected when appropriate
and flash frozen in the field. For additional details, see protocol
{[}NEON.DOC.003044vB{]}
(\url{http://data.neonscience.org/api/v0/documents/NEON.DOC.003044vB}):
AOS Protocol and Procedure for Aquatic Microbial Sampling. Queries for
this data product will return metadata tables that include field
observations and measurements, laboratory methods, and results from DNA
extraction, sample preparation, and sequencing for samples from the
specified sites and within the specified date range. The actual sequence
data are publicly available and may be queried on the
\href{http://metagenomics.anl.gov/}{Metagenomics Rapid Annotation using
Subsystem Technology (MG-RAST)} server. There may be lags between
publication of metadata on the NEON data portal and availability of
sequence data on the public sequence repository.

\begin{center}\rule{0.5\linewidth}{\linethickness}\end{center}

\textbf{Usage Notes}

Queries for this data product will return data from amb\_fieldParent,
mms\_benthicMetagenomeDnaExtraction and mms\_benthicMetagenomeSequencing
for all dates within the specified date range. The
mms\_benthicMetagenomeDnaExtraction data table is generic for all
microbial genetic data products: non-target samples may be included and
can be filtered using the field ``sequenceAnalysisType'' (filter to
values of ``metagenomes''). Each record in amb\_fieldParent may have one
or more child records in mms\_benthicMetagenomeDnaExtraction, and there
should be one child record in mms\_benthicMetagenomeSequencing for each
record in mms\_benthicMetagenomeDnaExtraction. A given
mms\_benthicMetagenomeDnaExtraction.dnaSampleID is expected to be
sampled one time per collectDate (local time). Duplicate samples may
exist where protocol and/or data entry aberrations have occurred; users
should check data carefully for anomalies before joining tables.
\newpage
.

\section{DP1.20280.001 Benthic microbe marker gene
sequences}\label{dp1.20280.001-benthic-microbe-marker-gene-sequences}

\begin{center}\rule{0.5\linewidth}{\linethickness}\end{center}

\textbf{Subsystem}

Aquatic Observation System (AOS)

\begin{center}\rule{0.5\linewidth}{\linethickness}\end{center}

\textbf{Coverage}

These data are collected at all NEON wadeable stream sites.

\begin{center}\rule{0.5\linewidth}{\linethickness}\end{center}

\textbf{Description}

DNA sequence data from ribosomal RNA marker genes from benthic samples

\begin{center}\rule{0.5\linewidth}{\linethickness}\end{center}

\textbf{Abstract}

This data product contains the quality-controlled laboratory metadata
and 16S and ITS marker gene sequences derived from NEON's benthic
microbial sampling in wadeable streams. For details about the methods
and design, see
\href{http://data.neonscience.org/api/v0/documents/NEON.DOC.003044vB}{AOS
Protocol and Procedure: Aquatic Microbial Sampling (NEON.DOC.003044)}
and
\href{http://data.neonscience.org/api/v0/documents/NEON.DOC.001152vA}{NEON
Aquatic Sampling Strategy (NEON.DOC.001152)}.

Queries for this data product return a downloadable data package with
laboratory methods and DNA extraction, PCR amplification, and sequencing
metadata for samples from the queried sites and date range. The actual
sequence data are publicly available and may be queried on the
\href{http://metagenomics.anl.gov/}{Metagenomics Rapid Annotation using
Subsystem Technology (MG-RAST)} server. There may be lags between
publication of metadata on the NEON data portal and availability of
sequence data on the public sequence repository. Sequence data may also
be obtained by querying NEON data sets at the
\href{https://www.ncbi.nlm.nih.gov/sra}{NCBI Sequence Read Archive (NCBI
SRA)} and the \href{https://www.ebi.ac.uk/}{European Bioinformatics
Institute (EMBL-EBI)}.

\begin{center}\rule{0.5\linewidth}{\linethickness}\end{center}

\textbf{Usage Notes}

Queries for this data product will return data from amb\_fieldParent,
mmg\_benthicDnaExtraction, mmg\_benthicPcrAmplification\_16S (and ITS)
and mmg\_benthicMarkerGeneSequencing\_16S (and ITS) for all months
within the specified date range. A given
mmg\_benthicMarkerGeneSequencing\_16S (or ITS). dnaSampleID is expected
to generate one record for each targetTaxonGroup. Duplicate samples
and/or missing data may exist where protocol and/or data entry
aberrations have occurred; users should check data carefully for
anomalies before joining tables. \newpage
.

\section{DP1.20281.001 Surface water microbe metagenome
sequences}\label{dp1.20281.001-surface-water-microbe-metagenome-sequences}

\begin{center}\rule{0.5\linewidth}{\linethickness}\end{center}

\textbf{Subsystem}

Aquatic Observation System (AOS)

\begin{center}\rule{0.5\linewidth}{\linethickness}\end{center}

\textbf{Coverage}

These data are collected at all NEON aquatic sites (wadeable streams,
lakes, and non-wadeable streams).

\begin{center}\rule{0.5\linewidth}{\linethickness}\end{center}

\textbf{Description}

Metagenomic sequence data from surface water samples

\begin{center}\rule{0.5\linewidth}{\linethickness}\end{center}

\textbf{Abstract}

This data product contains the quality-controlled laboratory metadata
and QA results for NEON's shotgun metagenomic sequences derived from
surface water microbial sampling. Surface water grab samples are
filtered on 0.22 um Sterivex capsule filters, capped and flash-frozen in
the field. For additional details, see protocol {[}NEON.DOC.003044vB{]}
(\url{http://data.neonscience.org/api/v0/documents/NEON.DOC.003044vB}):
AOS Protocol and Procedure for Aquatic Microbial Sampling. Queries for
this data product will return metadata tables that include field
observations and measurements, laboratory methods, and results from DNA
extraction, sample preparation, and sequencing for samples from the
specified sites and within the specified date range. The actual sequence
data are publicly available and may be queried on the
\href{http://metagenomics.anl.gov/}{Metagenomics Rapid Annotation using
Subsystem Technology (MG-RAST)} server. There may be lags between
publication of metadata on the NEON data portal and availability of
sequence data on the public sequence repository.

\begin{center}\rule{0.5\linewidth}{\linethickness}\end{center}

\textbf{Usage Notes}

Queries for the basic download data product will return data from
amc\_fieldSuperParent, amc\_fieldGenetic, mms\_swMetagenomeDnaExtraction
and mms\_swMetagenomeSequencing for all dates within the specified date
range. The mms\_swMetagenomeDnaExtraction data table is generic for all
microbial genetic data products: non-target samples may be included and
can be filtered using the field ``sequenceAnalysisType'' (filter to
values of ``metagenomes''). Each record in amc\_fieldSuperParent may
have one or more child records in amc\_fieldGenetic and
mms\_swMetagenomeDnaExtraction, and there should be one child record in
mms\_swcMetagenomeSequencing for each record in
mms\_swMetagenomeDnaExtraction. A given
mms\_swMetagenomeDnaExtraction.dnaSampleID is expected to be sampled one
time per collectDate (local time). Duplicate samples may exist where
protocol and/or data entry aberrations have occurred; users should check
data carefully for anomalies before joining tables. \newpage
.

\section{DP1.20282.001 Surface water microbe marker gene
sequences}\label{dp1.20282.001-surface-water-microbe-marker-gene-sequences}

\begin{center}\rule{0.5\linewidth}{\linethickness}\end{center}

\textbf{Subsystem}

Aquatic Observation System (AOS)

\begin{center}\rule{0.5\linewidth}{\linethickness}\end{center}

\textbf{Coverage}

These data are collected at all NEON lake and non-wadeable stream sites.

\begin{center}\rule{0.5\linewidth}{\linethickness}\end{center}

\textbf{Description}

DNA sequence data from ribosomal RNA marker genes from surface water
samples

\begin{center}\rule{0.5\linewidth}{\linethickness}\end{center}

\textbf{Abstract}

This data product contains the quality-controlled laboratory metadata
and 16S and ITS marker gene sequences derived from NEON's surface water
microbial sampling. For details about the methods and design, see
\href{http://data.neonscience.org/api/v0/documents/NEON.DOC.003044vB}{AOS
Protocol and Procedure: Aquatic Microbial Sampling (NEON.DOC.003044)}
and
\href{http://data.neonscience.org/api/v0/documents/NEON.DOC.001152vA}{NEON
Aquatic Sampling Strategy (NEON.DOC.001152)}.

Queries for this data product return a downloadable data package with
laboratory methods and DNA extraction, PCR amplification, and sequencing
metadata for samples from the queried sites and date range. The actual
sequence data are publicly available and may be queried on the
\href{http://metagenomics.anl.gov/}{Metagenomics Rapid Annotation using
Subsystem Technology (MG-RAST)} server. There may be lags between
publication of metadata on the NEON data portal and availability of
sequence data on the public sequence repository. Sequence data may also
be obtained by querying NEON data sets at the
\href{https://www.ncbi.nlm.nih.gov/sra}{NCBI Sequence Read Archive (NCBI
SRA)} and the \href{https://www.ebi.ac.uk/}{European Bioinformatics
Institute (EMBL-EBI)}.

\begin{center}\rule{0.5\linewidth}{\linethickness}\end{center}

\textbf{Usage Notes}

Queries for this data product will return data from
amc\_fieldSuperParent, amc\_fieldGenetic, mmg\_swDnaExtraction,
mmg\_swPcrAmplification\_16S (and ITS) and
mmg\_swMarkerGeneSequencing\_16S (and ITS) for all months within the
specified date range. A given mmg\_swMarkerGeneSequencing\_16S (or ITS).
dnaSampleID is expected to generate one record for each
targetTaxonGroup. Duplicate samples and/or missing data may exist where
protocol and/or data entry aberrations have occurred; users should check
data carefully for anomalies before joining tables. \newpage
.

\section{DP1.20288.001 Water quality}\label{dp1.20288.001-water-quality}

\begin{center}\rule{0.5\linewidth}{\linethickness}\end{center}

\textbf{Subsystem}

Aquatic Instrument System (AIS)

\begin{center}\rule{0.5\linewidth}{\linethickness}\end{center}

\textbf{Coverage}

S1 (upstream) and S2 (downstream) sensor sets are at all wadeable stream
sites within NEON. Buoys are deployed at all~lake and large river sites
within NEON.

\begin{center}\rule{0.5\linewidth}{\linethickness}\end{center}

\textbf{Description}

In situ sensor-based specific conductivity, concentration of chlorophyll
a, dissolved oxygen content, fDOM concentration, pH, and turbidity,
available as one-, five-, and thirty-minute averages in surface water of
lakes, wadeable streams, and non-wadeable streams.

\begin{center}\rule{0.5\linewidth}{\linethickness}\end{center}

\textbf{Abstract}

Water quality is measured once per minute at stream sensor sets and once
per 5 minutes on buoys at lake and river sites. It is reported as 1- or
5-minute instantaneous measurements. \newpage
.

\section{DP1.30001.001 LiDAR slant range
waveform}\label{dp1.30001.001-lidar-slant-range-waveform}

\begin{center}\rule{0.5\linewidth}{\linethickness}\end{center}

\textbf{Subsystem}

Airborne Observation Platform (AOP)

\begin{center}\rule{0.5\linewidth}{\linethickness}\end{center}

\textbf{Coverage}

NEON AOP data are planned for yearly collects at all NEON sites at 90\%
of maximum greenness or greater. Coverage is planned to include at least
95\% of NIS Tower Airshed area as well as at least 80\% of a minimum
10km x 10 km box around that. All acquisitions are subject to change due
to weather conditions as well as program planning changes.

\begin{center}\rule{0.5\linewidth}{\linethickness}\end{center}

\textbf{Description}

Outgoing pulse and slant range return waveform signals with geolocation
information provided, but no spatial resampling. Data are provided by
flightline in a binary format designed by NEON.

\begin{center}\rule{0.5\linewidth}{\linethickness}\end{center}

\textbf{Abstract}

The Level 1 Slant Range Waveform Lidar data product provides a
geolocated waveform for each laser pulse in a binary output format. The
X and Y coordinates are reported in the output horizontal datum and
projection and the Z values are reported in absolute elevation in the
output vertical datum. The waveform product saves the continuous
received signal versus time (digitized into 1 nsec time bins. The
waveform shapes might provide important information about scattering
properties, especially in the case of vegetation. Each AOP flight line
is saved as an individual zip file, which includes a set of binary files
plus the quality check (QC) first return LAS file. A nominal 10 km long
flight line flown at a speed of approximately 100 knots will take about
200 seconds to collect. At a pulse repetition frequency (PRF) value of
100 kHz, the resulting product will contain approximately 20 million
laser pulses. The return waveforms are saved as a binary data file with
250 columns (the 1 nsec time bins) by the number of rows equaling the
number of laser pulses. A nominal waveform .zip file will be
approximately 50 GB and contains several files. Waveform lidar data have
many uses: 3D visualization; generation of surface models such as
bare-Earth digital elevation models (DEM) also referred to as digital
terrain models (DTM), digital surface models (DSM), and canopy height
models (CHM); analysis of vegetation structure, leaf area index, and
biomass; analysis of canopy light penetration and attenuation; and
watershed analysis.

*Note: Data are being migrated to the data portal. If you don't find the
data you are looking for (e.g., from specific sites or years), please
request data
\href{http://www.neonscience.org/request-airborne-data}{here}. \newpage
.

\section{DP1.30003.001 Discrete return LiDAR point
cloud}\label{dp1.30003.001-discrete-return-lidar-point-cloud}

\begin{center}\rule{0.5\linewidth}{\linethickness}\end{center}

\textbf{Subsystem}

Airborne Observation Platform (AOP)

\begin{center}\rule{0.5\linewidth}{\linethickness}\end{center}

\textbf{Coverage}

NEON AOP data are planned for yearly collects at all NEON sites at 90\%
of maximum greenness or greater. Coverage is planned to include at least
95\% of NIS Tower Airshed area as well as at least 80\% of a minimum
10km x 10 km box around that. All acquisitions are subject to change due
to weather conditions as well as program planning changes.

\begin{center}\rule{0.5\linewidth}{\linethickness}\end{center}

\textbf{Description}

Unclassified three-dimensional point cloud stored in LAS format.
Classifications follow ASPRS definition. All point coordinates are
provided in meters. Data provided by flightline.

\begin{center}\rule{0.5\linewidth}{\linethickness}\end{center}

\textbf{Abstract}

The NEON AOP Discrete Return Light Detection and Ranging (LiDAR) Point
Cloud is an American Society for Photogrammetry and Remote Sensing
(ASPRS) LASer format data product in UTM map projection. It provides the
X, Y, and Z coordinates for each laser return point. AOP discrete LiDAR
is collected at approximately 4 points per square meter, and each point
can have up to 5 returns. L1 LiDAR point clouds are distributed in their
original flight lines with one flight line per file. \newpage
.

\section{DP1.30006.001}\label{dp1.30006.001}

Spectrometer orthorectified surface directional reflectance - flightline

\begin{center}\rule{0.5\linewidth}{\linethickness}\end{center}

\textbf{Subsystem}

Airborne Observation Platform (AOP)

\begin{center}\rule{0.5\linewidth}{\linethickness}\end{center}

\textbf{Coverage}

NEON AOP data are planned for yearly collects at all NEON sites at 90\%
of maximum greenness or greater. Coverage is planned to include at least
95\% of NIS Tower Airshed area as well as at least 80\% of a minimum
10km x 10 km box around that. All acquisitions are subject to change due
to weather conditions as well as program planning changes.

\begin{center}\rule{0.5\linewidth}{\linethickness}\end{center}

\textbf{Description}

Surface reflectance (0-1 unitless, scaled by 10,000) computed from the
NEON Imaging Spectrometer using ATCOR4r is orthorectified and output
onto a fixed, uniform spatial grid using nearest-neighbor resampling.
Fixed spatial grid is based on the native spatial resolution which is
driven by the aircraft altitude; data are provided by flightline.

\begin{center}\rule{0.5\linewidth}{\linethickness}\end{center}

\textbf{Abstract}

The NEON AOP surface directional reflectance data product is an
orthorectified (UTM projection) hyperspectral raster product. It is
distributed in an open HDF5 format including all 426 bands from the NEON
Imaging Spectrometer. It is a calibrated and atmospherically corrected
product distributed as scaled reflectance. It includes many QA and
ancillary rasters used as inputs to ATCOR for atmospheric correction as
well as outputs from ATCOR for diagnostic purposes. L1 reflectance is
distributed by original flight line with one HDF5 file per flight line
including the reflectance data and all metadata and ancillary data.

*Note: Data are being migrated to the data portal. If you don't find the
data you are looking for (e.g., from specific sites or years), please
request data
\href{http://www.neonscience.org/request-airborne-data}{here}. \newpage
.

\section{DP1.30008.001}\label{dp1.30008.001}

Spectrometer orthrorectified at-sensor radiance - flightline

\begin{center}\rule{0.5\linewidth}{\linethickness}\end{center}

\textbf{Subsystem}

Airborne Observation Platform (AOP)

\begin{center}\rule{0.5\linewidth}{\linethickness}\end{center}

\textbf{Coverage}

NEON AOP data are planned for yearly collects at all NEON sites at 90\%
of maximum greenness or greater. Coverage is planned to include at least
95\% of NIS Tower Airshed area as well as at least 80\% of a minimum
10km x 10 km box around that. All acquisitions are subject to change due
to weather conditions as well as program planning changes.

\begin{center}\rule{0.5\linewidth}{\linethickness}\end{center}

\textbf{Description}

Calibrated radiance in units of uW/cm\^{}2-sr-nm as measured by the NEON
Imaging Spectrometer is orthorectified and output onto a fixed, uniform
spatial grid using nearest-neighbor resampling. Fixed spatial grid is
based on the native spatial resolution which is driven by the aircraft
altitude; data are provided by flightline.

\begin{center}\rule{0.5\linewidth}{\linethickness}\end{center}

\textbf{Abstract}

The NEON AOP at-sensor radiance data product is a calibrated,
orthorectified (UTM projection) hyperspectral raster product. It is
distributed in an open HDF5 format including all 426 bands from the NEON
Imaging Spectrometer. It includes many QA and ancillary rasters required
for atmospheric correction. L1 radiance is distributed by original
flight line with one HDF5 file per flight line including the radiance
data and all metadata and ancillary data. \newpage
.

\section{DP1.30010.001 High-resolution orthorectified camera
imagery}\label{dp1.30010.001-high-resolution-orthorectified-camera-imagery}

\begin{center}\rule{0.5\linewidth}{\linethickness}\end{center}

\textbf{Subsystem}

Airborne Observation Platform (AOP)

\begin{center}\rule{0.5\linewidth}{\linethickness}\end{center}

\textbf{Coverage}

NEON AOP data are planned for yearly collects at all NEON sites at 90\%
of maximum greenness or greater. Coverage is planned to include at least
95\% of NIS Tower Airshed area as well as at least 80\% of a minimum
10km x 10 km box around that. All acquisitions are subject to change due
to weather conditions as well as program planning changes.

\begin{center}\rule{0.5\linewidth}{\linethickness}\end{center}

\textbf{Description}

White balanced 8 bit RGB images orthorectified and output onto a fixed,
uniform spatial grid using nearest neighbor resampling to a 10 cm
spatial resolution.

\begin{center}\rule{0.5\linewidth}{\linethickness}\end{center}

\textbf{Abstract}

The digital camera is part of a suite of instruments on the NEON
Airborne Observation Platform (AOP) that also includes a full-waveform,
small-footprint LiDAR system and the NEON Imaging Spectrometer. In the
orthorectification process, the digital imagery is remapped to the same
geographic projection as the LiDAR and imaging spectrometer data that is
acquired simultaneously. The resulting images will share the same map
projection grid space as the orthorectified spectrometer and LiDAR
imagery. Since the digital camera imagery is acquired at higher spatial
resolution than the imaging spectrometer data, it can aid in identifying
features in the spectrometer images including manmade features (e.g.,
roads, fence lines, and buildings) that are indicative of land-use
change. Level 1 RGB camera images are distributed as one camera frame
per GeoTIFF file.

*Note: Data are being migrated to the data portal. If you don't find the
data you are looking for (e.g., from specific sites or years), please
request data
\href{http://www.neonscience.org/request-airborne-data}{here}. \newpage
.

\section{DP1.30012.001 Field spectral
data}\label{dp1.30012.001-field-spectral-data}

\begin{center}\rule{0.5\linewidth}{\linethickness}\end{center}

\textbf{Subsystem}

Airborne Observation Platform (AOP)

\begin{center}\rule{0.5\linewidth}{\linethickness}\end{center}

\textbf{Coverage}

All NEON sites.

\begin{center}\rule{0.5\linewidth}{\linethickness}\end{center}

\textbf{Description}

Reflectance of collected sample(s) or transects using an ASD Field
Spectrometer; level of effort product, provided when collected.

\begin{center}\rule{0.5\linewidth}{\linethickness}\end{center}

\textbf{Abstract}

NEON AOP Field Spectral Data are collected using an ASD field
spectrometer for calibrations tarps and representative ground and
landcover materials while the airborne data are being simultaneously
collected. The Field Spectral Data are collected and distributed as a
Level Of Effort activity and as such are not collected for every flight
or site. Typical collection rates are 2 to 3 sites per year. \newpage
.

\section{DP2.00004.001 Temporally interpolated biological
temperature}\label{dp2.00004.001-temporally-interpolated-biological-temperature}

\begin{center}\rule{0.5\linewidth}{\linethickness}\end{center}

\textbf{Subsystem}

Terrestrial Instrument System (TIS)

\begin{center}\rule{0.5\linewidth}{\linethickness}\end{center}

\textbf{Coverage}

NA

\begin{center}\rule{0.5\linewidth}{\linethickness}\end{center}

\textbf{Description}

Temporally interpolated (i.e., to gap fill missing data) biological
temperature (i.e.~surface temperature) measured via IR temperature
sensors located in the soil array and at multiple heights on the tower
infrastructure.

\begin{center}\rule{0.5\linewidth}{\linethickness}\end{center}

\textbf{Abstract}

NA \newpage
.

\section{DP2.00005.001}\label{dp2.00005.001}

Temporally interpolated photosynthetically active radiation

\begin{center}\rule{0.5\linewidth}{\linethickness}\end{center}

\textbf{Subsystem}

Terrestrial Instrument System (TIS)

\begin{center}\rule{0.5\linewidth}{\linethickness}\end{center}

\textbf{Coverage}

NA

\begin{center}\rule{0.5\linewidth}{\linethickness}\end{center}

\textbf{Description}

Temporally interpolated (i.e., to gap fill missing data)
Photosynthetically Active Radiation (PAR). Observations are made by
sensors located at multiple heights on the tower infrastructure and a
single sensor located on the aquatic meteorology station.

\begin{center}\rule{0.5\linewidth}{\linethickness}\end{center}

\textbf{Abstract}

NA \newpage
.

\section{DP2.00006.001 Temporally interpolated soil
temperature}\label{dp2.00006.001-temporally-interpolated-soil-temperature}

\begin{center}\rule{0.5\linewidth}{\linethickness}\end{center}

\textbf{Subsystem}

Terrestrial Instrument System (TIS)

\begin{center}\rule{0.5\linewidth}{\linethickness}\end{center}

\textbf{Coverage}

NA

\begin{center}\rule{0.5\linewidth}{\linethickness}\end{center}

\textbf{Description}

Temporally interpolated (i.e., to gap fill missing data) soil
temperature at various depth below the soil surface from 2 cm up to 200
cm at non-permafrost sites (up to 300 cm at Alaskan sites). Data are
from all five Instrumented Soil Plots per site and presented as 1-minute
and 30-minute averages.

\begin{center}\rule{0.5\linewidth}{\linethickness}\end{center}

\textbf{Abstract}

NA \newpage
.

\section{DP2.00008.001 CO2 concentration rate of
change}\label{dp2.00008.001-co2-concentration-rate-of-change}

\begin{center}\rule{0.5\linewidth}{\linethickness}\end{center}

\textbf{Subsystem}

Terrestrial Instrument System (TIS)

\begin{center}\rule{0.5\linewidth}{\linethickness}\end{center}

\textbf{Coverage}

These data are measured at all terrestrial sites. Sensors are located
inside the instrument hut near the bottom of the tower. The air samples
from different measurement heights are pumped through gas tubing to
sensor for analysis.

\begin{center}\rule{0.5\linewidth}{\linethickness}\end{center}

\textbf{Description}

Time rate of change of CO2 concentration (storage component only) over
30 minutes at each measurement level along the vertical tower profile.
Gap-filling is not applicable. This data product is bundled into
DP4.00200, Bundled data products - eddy covariance, and is not available
as a stand-alone download.

\begin{center}\rule{0.5\linewidth}{\linethickness}\end{center}

\textbf{Abstract}

This data product is the temporally interpolated CO2 data (time rate of
change for CO2) at the 30 minute time scale at different measurement
levels on the tower. The data are delivered with the Bundled data
products - eddy covariance data product (DP4.00200.001).

\begin{center}\rule{0.5\linewidth}{\linethickness}\end{center}

\textbf{Usage Notes}

During subsequent nominal operations, we plan to produce and publish the
data products in three phases, to accommodate a variety of use cases:
the initial near-real-time transition, a science reviewed quality
transition, and the epoch yearly transition. The initial near-real-time
transition is scheduled to process daily files at a 5-day delay after
data collection to accommodate a 9-day centered planar-fit window. If
the data has not been received from the field it will attempt to process
daily for 30\,days, and if not all data is available after this window a
force execution is performed populating a HDF5 file with metadata and
filling data with NaN's. The monthly file will be produced after all
daily files are available, no later than 30 days after the last daily
file was initially attempted to be processed. After the initial
transition, the NEON science team has a one month window to manually
flag data that were identified as suspect through field-based problem
tracking and resolution tickets or through additional manual data
quality analysis. Then, the science-reviewed transition will occur, and
the data will be republished to the data portal. The last transition
type is part of the yearly epoch versioning, which provides a fully
quality assured and quality controlled version of the data using the
latest full release of the processing code. This transition is scheduled
to occur 18 months after the initial data collection. \newpage
.

\section{DP2.00009.001 H2O concentration rate of
change}\label{dp2.00009.001-h2o-concentration-rate-of-change}

\begin{center}\rule{0.5\linewidth}{\linethickness}\end{center}

\textbf{Subsystem}

Terrestrial Instrument System (TIS)

\begin{center}\rule{0.5\linewidth}{\linethickness}\end{center}

\textbf{Coverage}

These data are measured at all terrestrial sites. Sensors are located
inside the instrument hut near the bottom of the tower. The air samples
from different measurement heights are pumped through gas tubing to
sensor for analysis.

\begin{center}\rule{0.5\linewidth}{\linethickness}\end{center}

\textbf{Description}

Time rate of change of H2O concentration (storage component only) over
30 minutes at each measurement level along the vertical tower profile.
Gap-filling is not applicable. This data product is bundled into
DP4.00200, Bundled data products - eddy covariance, and is not available
as a stand-alone download.

\begin{center}\rule{0.5\linewidth}{\linethickness}\end{center}

\textbf{Abstract}

This data product is the temporally interpolated H2O data (time rate of
change for H2O) at the 30 minute time scale at different measurement
levels on the tower. The data are delivered with the Bundled data
products - eddy covariance data product (DP4.00200.001).

\begin{center}\rule{0.5\linewidth}{\linethickness}\end{center}

\textbf{Usage Notes}

During subsequent nominal operations, we plan to produce and publish the
data products in three phases, to accommodate a variety of use cases:
the initial near-real-time transition, a science reviewed quality
transition, and the epoch yearly transition. The initial near-real-time
transition is scheduled to process daily files at a 5-day delay after
data collection to accommodate a 9-day centered planar-fit window. If
the data has not been received from the field it will attempt to process
daily for 30\,days, and if not all data is available after this window a
force execution is performed populating a HDF5 file with metadata and
filling data with NaN's. The monthly file will be produced after all
daily files are available, no later than 30 days after the last daily
file was initially attempted to be processed. After the initial
transition, the NEON science team has a one month window to manually
flag data that were identified as suspect through field-based problem
tracking and resolution tickets or through additional manual data
quality analysis. Then, the science-reviewed transition will occur, and
the data will be republished to the data portal. The last transition
type is part of the yearly epoch versioning, which provides a fully
quality assured and quality controlled version of the data using the
latest full release of the processing code. This transition is scheduled
to occur 18 months after the initial data collection. \newpage
.

\section{DP2.00016.001 Temporally interpolated
PAR-line}\label{dp2.00016.001-temporally-interpolated-par-line}

\begin{center}\rule{0.5\linewidth}{\linethickness}\end{center}

\textbf{Subsystem}

Terrestrial Instrument System (TIS)

\begin{center}\rule{0.5\linewidth}{\linethickness}\end{center}

\textbf{Coverage}

NA

\begin{center}\rule{0.5\linewidth}{\linethickness}\end{center}

\textbf{Description}

Temporally interpolated (i.e., to gap fill missing data)
Photosynthetically Active Radiation (PAR). Observations are made by
sensors at the soil surface covering a one meter length.

\begin{center}\rule{0.5\linewidth}{\linethickness}\end{center}

\textbf{Abstract}

NA \newpage
.

\section{DP2.00020.001}\label{dp2.00020.001}

Temporally interpolated shortwave and longwave radiation (net
radiometer)

\begin{center}\rule{0.5\linewidth}{\linethickness}\end{center}

\textbf{Subsystem}

Terrestrial Instrument System (TIS)

\begin{center}\rule{0.5\linewidth}{\linethickness}\end{center}

\textbf{Coverage}

NA

\begin{center}\rule{0.5\linewidth}{\linethickness}\end{center}

\textbf{Description}

Temporally interpolated (i.e., to gap fill missing data) net radiation
that is composed of incoming and outgoing shortwave and longwave
radiation. These data products are available as one- and thirty-minute
averages from sensors located on the TIS tower and located on the
aquatic meteorology station.

\begin{center}\rule{0.5\linewidth}{\linethickness}\end{center}

\textbf{Abstract}

NA \newpage
.

\section{DP2.00023.001 Temporally interpolated triple aspirated tower
temperature}\label{dp2.00023.001-temporally-interpolated-triple-aspirated-tower-temperature}

\begin{center}\rule{0.5\linewidth}{\linethickness}\end{center}

\textbf{Subsystem}

Terrestrial Instrument System (TIS)

\begin{center}\rule{0.5\linewidth}{\linethickness}\end{center}

\textbf{Coverage}

NA

\begin{center}\rule{0.5\linewidth}{\linethickness}\end{center}

\textbf{Description}

Temporally interpolated (i.e.~to gap fill missing data) air temperature,
derived from triplicate 1 Hz temperature observations. Observations are
made by sensors located at the top of the tower infrastructure.
Temperature observations are made by three platinum resistance
thermometers, which are housed together in a fan aspirated shield to
reduce radiative biases.

\begin{center}\rule{0.5\linewidth}{\linethickness}\end{center}

\textbf{Abstract}

NA \newpage
.

\section{DP2.00024.001 Temperature rate of
change}\label{dp2.00024.001-temperature-rate-of-change}

\begin{center}\rule{0.5\linewidth}{\linethickness}\end{center}

\textbf{Subsystem}

Terrestrial Instrument System (TIS)

\begin{center}\rule{0.5\linewidth}{\linethickness}\end{center}

\textbf{Coverage}

These data are collected at all NEON terrestrial sites.

\begin{center}\rule{0.5\linewidth}{\linethickness}\end{center}

\textbf{Description}

Time rate of change of temperature (storage component only) over 30
minutes at each measurement level along the vertical tower profile.
Gap-filling is not applicable. This data product is bundled into
DP4.00200, Bundled data products - eddy covariance, and is not available
as a stand-alone download.

\begin{center}\rule{0.5\linewidth}{\linethickness}\end{center}

\textbf{Abstract}

This data product is the temporally interpolated temperature data (time
rate of change for temperature) at the 30 minute time scale at different
measurement levels on the tower. The data are delivered with the Bundled
data products - eddy covariance data product (DP4.00200.001).

\begin{center}\rule{0.5\linewidth}{\linethickness}\end{center}

\textbf{Usage Notes}

During subsequent nominal operations, we plan to produce and publish the
data products in three phases, to accommodate a variety of use cases:
the initial near-real-time transition, a science reviewed quality
transition, and the epoch yearly transition. The initial near-real-time
transition is scheduled to process daily files at a 5-day delay after
data collection to accommodate a 9-day centered planar-fit window. If
the data has not been received from the field it will attempt to process
daily for 30\,days, and if not all data is available after this window a
force execution is performed populating a HDF5 file with metadata and
filling data with NaN's. The monthly file will be produced after all
daily files are available, no later than 30 days after the last daily
file was initially attempted to be processed. After the initial
transition, the NEON science team has a one month window to manually
flag data that were identified as suspect through field-based problem
tracking and resolution tickets or through additional manual data
quality analysis. Then, the science-reviewed transition will occur, and
the data will be republished to the data portal. The last transition
type is part of the yearly epoch versioning, which provides a fully
quality assured and quality controlled version of the data using the
latest full release of the processing code. This transition is scheduled
to occur 18 months after the initial data collection. \newpage
.

\section{DP2.30011.001 Albedo - spectrometer -
flightline}\label{dp2.30011.001-albedo---spectrometer---flightline}

\begin{center}\rule{0.5\linewidth}{\linethickness}\end{center}

\textbf{Subsystem}

Airborne Observation Platform (AOP)

\begin{center}\rule{0.5\linewidth}{\linethickness}\end{center}

\textbf{Coverage}

NEON AOP data are planned for yearly collects at all NEON sites at 90\%
of maximum greenness or greater. Coverage is planned to include at least
95\% of NIS Tower Airshed area as well as at least 80\% of a minimum
10km x 10 km box around that. All acquisitions are subject to change due
to weather conditions as well as program planning changes.

\begin{center}\rule{0.5\linewidth}{\linethickness}\end{center}

\textbf{Description}

Total amount of solar radiation in the 0.4 to 2.5 micron band reflected
by the Earth surface into an upward hemisphere divided by the total
amount incident from this hemisphere; data are provided by flightline

\begin{center}\rule{0.5\linewidth}{\linethickness}\end{center}

\textbf{Abstract}

Albedo, the ratio of a surface's reflected energy to its incident
energy, is an important measurement for characterizing earth system
energy balance. Light and dark surfaces correspond to high and low
albedo, respectively. An opaque surface's difference in energy reflected
as compared to the energy incident on it is absorbed by the surface,
increasing its temperature. (Sabins, Jr., 1978). Albedo values depend on
wavelength, illumination sources and geometry, sensor viewing geometry,
reflectance as a function of angle and wavelength, as well as the
scattering, absorbing, and re-radiating effects of the atmosphere. These
factors are modeled/accounted for to best approximate a bi-hemispherical
reflectance as would be measured in a laboratory setting. To this end,
the wavelength-integrated surface reflectance, weighted with the global
flux on the ground, is produced as the best practically achievable
albedo measurement. (Richter \& Schlapfer, 2017) L2 Albedo is
distributed by flight line.

*Note: Data are being migrated to the data portal. If you don't find the
data you are looking for (e.g., from specific sites or years), please
request data
\href{http://www.neonscience.org/request-airborne-data}{here}. \newpage
.

\section{DP2.30012.001 LAI - spectrometer -
flightline}\label{dp2.30012.001-lai---spectrometer---flightline}

\begin{center}\rule{0.5\linewidth}{\linethickness}\end{center}

\textbf{Subsystem}

Airborne Observation Platform (AOP)

\begin{center}\rule{0.5\linewidth}{\linethickness}\end{center}

\textbf{Coverage}

NEON AOP data are planned for yearly collects at all NEON sites at 90\%
of maximum greenness or greater. Coverage is planned to include at least
95\% of NIS Tower Airshed area as well as at least 80\% of a minimum
10km x 10 km box around that. All acquisitions are subject to change due
to weather conditions as well as program planning changes.

\begin{center}\rule{0.5\linewidth}{\linethickness}\end{center}

\textbf{Description}

The ratio of upper leaf surface area to ground area (for broadleaf
canopies), or projected conifer needle surface area to ground area (for
coniferous plants) for a given unit area; measured by an industry
standard (ATCOR) algorithm based on Soil Adjusted Vegetation Index
(SAVI) as input; data are provided by flightline.

\begin{center}\rule{0.5\linewidth}{\linethickness}\end{center}

\textbf{Abstract}

The leaf area index (LAI) is a derived spectral product from remotely
sensed data that is used as a proxy for describing leaf area across
areas larger than can be measured by more direct ground-based
measurements such as hemispherical photography. It is often used as an
input layer for productivity, landscape, and climate models. The Level 2
LAI product is distributed by flight line in GeoTIFF format. \newpage
.

\section{DP2.30014.001 fPAR - spectrometer -
flightline}\label{dp2.30014.001-fpar---spectrometer---flightline}

\begin{center}\rule{0.5\linewidth}{\linethickness}\end{center}

\textbf{Subsystem}

Airborne Observation Platform (AOP)

\begin{center}\rule{0.5\linewidth}{\linethickness}\end{center}

\textbf{Coverage}

NEON AOP data are planned for yearly collects at all NEON sites at 90\%
of maximum greenness or greater. Coverage is planned to include at least
95\% of NIS Tower Airshed area as well as at least 80\% of a minimum
10km x 10 km box around that. All acquisitions are subject to change due
to weather conditions as well as program planning changes.

\begin{center}\rule{0.5\linewidth}{\linethickness}\end{center}

\textbf{Description}

The fraction of incident photosynthetically active radiation (400-700
nm) absorbed by the green elements of a vegetation canopy; calculated by
an industry standard (ATCOR) algorithm based on Soil Adjusted Vegetation
Index (SAVI) as input; data are provided by flightline at equivalent
resolution to spectrometer orthorectified surface directional
reflectance, carbon and nutrient cycling due to its relationship with
vegetative productivity.

\begin{center}\rule{0.5\linewidth}{\linethickness}\end{center}

\textbf{Abstract}

The fraction of photosynthetically active radiation (fPAR) describes the
relative quantity of incident solar radiation of relevant
photosynthetically active wavelengths (0.4-0.7 microns) absorbed by
vegetative material. The fPAR is an important biophysical variable used
in the simulation of water, carbon and nutrient cycling due to its
relationship with vegetative productivity. Theoretically, if a plant is
able to intercept and absorb relevant photosynthetically active
wavelengths, this will result in a higher state of productivity, gas
exchange and transpiration. The application of this theory is critical
in assessments of productivity change through time, and simulation of
climate models to predict ecosystems response to climate variability.
The level 2 version of the fPAR product is distributed by flight line.
\newpage
.

\section{DP2.30016.001 Total biomass map - spectrometer -
flightline}\label{dp2.30016.001-total-biomass-map---spectrometer---flightline}

\begin{center}\rule{0.5\linewidth}{\linethickness}\end{center}

\textbf{Subsystem}

Airborne Observation Platform (AOP)

\begin{center}\rule{0.5\linewidth}{\linethickness}\end{center}

\textbf{Coverage}

NEON AOP data are planned for yearly collects at all NEON sites at 90\%
of maximum greenness or greater. Coverage is planned to include at least
95\% of NIS Tower Airshed area as well as at least 80\% of a minimum
10km x 10 km box around that. All acquisitions are subject to change due
to weather conditions as well as program planning changes.

\begin{center}\rule{0.5\linewidth}{\linethickness}\end{center}

\textbf{Description}

Mass of all above ground organic matter per unit area at particular
time; estimate of biomass derived from correlation with NDVI and LAI
parameters; data are provided by flightline data are provided by
flightline at equivalent resolution to spectrometer orthorectified
surface directional reflectance.

\begin{center}\rule{0.5\linewidth}{\linethickness}\end{center}

\textbf{Abstract}

The NEON AOP Total Biomass data product is an orthorectified (UTM
projection) raster product derived from NEON AOP Imaging Spectrometer
(NIS) reflectance data. Biomass is an important layer in models and
measurements involving climate, landscape ecology, and the carbon cycle.
Remotely sensed estimates of biomass are important links between ground
based biomass measurements and models operating at landscape, regional,
or global scales. The biomass product is distributed in GeoTIFF format
with each file containing the biomass raster for a single flight line.
\newpage
.

\section{DP2.30018.001 Canopy nitrogen -
flightline}\label{dp2.30018.001-canopy-nitrogen---flightline}

\begin{center}\rule{0.5\linewidth}{\linethickness}\end{center}

\textbf{Subsystem}

Airborne Observation Platform (AOP)

\begin{center}\rule{0.5\linewidth}{\linethickness}\end{center}

\textbf{Coverage}

NEON AOP data are planned for yearly collects at all NEON sites at 90\%
of maximum greenness or greater. Coverage is planned to include at least
95\% of NIS Tower Airshed area as well as at least 80\% of a minimum
10km x 10 km box around that. All acquisitions are subject to change due
to weather conditions as well as program planning changes.

\begin{center}\rule{0.5\linewidth}{\linethickness}\end{center}

\textbf{Description}

Normalized difference nitrogen index from remotely sensed data; data are
provided by flightline

\begin{center}\rule{0.5\linewidth}{\linethickness}\end{center}

\textbf{Abstract}

Canopy Nitrogen, or Normalized Difference Nitrogen Index (NDNI),
estimates the relative amounts of nitrogen in vegetation land cover. The
index uses reflectance at 1510 nm (determined largely by nitrogen
concentration in plants and foliar biomass) and at 1680 nm (sensitive to
biomass but not to nitrogen absorption). NDNI is a relatively new
spectral index in remote sensing. L2 NDNI is distributed in the original
North/South flight lines.

*Note: Data are being migrated to the data portal. If you don't find the
data you are looking for (e.g., from specific sites or years), please
request data
\href{http://www.neonscience.org/request-airborne-data}{here}. \newpage
.

\section{DP2.30019.001 Canopy water content -
flightline}\label{dp2.30019.001-canopy-water-content---flightline}

\begin{center}\rule{0.5\linewidth}{\linethickness}\end{center}

\textbf{Subsystem}

Airborne Observation Platform (AOP)

\begin{center}\rule{0.5\linewidth}{\linethickness}\end{center}

\textbf{Coverage}

NEON AOP data are planned for yearly collects at all NEON sites at 90\%
of maximum greenness or greater. Coverage is planned to include at least
95\% of NIS Tower Airshed area as well as at least 80\% of a minimum
10km x 10 km box around that. All acquisitions are subject to change due
to weather conditions as well as program planning changes.

\begin{center}\rule{0.5\linewidth}{\linethickness}\end{center}

\textbf{Description}

Normalized index of canopy water content; data are provided by
flightline

\begin{center}\rule{0.5\linewidth}{\linethickness}\end{center}

\textbf{Abstract}

The Canopy Water Content data products are a family of 5 spectral
indices: MSI, NDII, NDWI, NMDI, and WBI. These indices use regions
vegetation reflectance spectra known to be indicators of leaf water
content, relative canopy water content, changes in canopy water content,
soil and canopy water content, and changes in canopy water status,
respectively. L2 Canopy Water Content is distributed in the original
North/South flight lines and is packaged as a zip file containing one
GeoTIFF for each index.

*Note: Data are being migrated to the data portal. If you don't find the
data you are looking for (e.g., from specific sites or years), please
request data
\href{http://www.neonscience.org/request-airborne-data}{here}. \newpage
.

\section{DP2.30020.001 Canopy xanthophyll cycle -
flightline}\label{dp2.30020.001-canopy-xanthophyll-cycle---flightline}

\begin{center}\rule{0.5\linewidth}{\linethickness}\end{center}

\textbf{Subsystem}

Airborne Observation Platform (AOP)

\begin{center}\rule{0.5\linewidth}{\linethickness}\end{center}

\textbf{Coverage}

NEON AOP data are planned for yearly collects at all NEON sites at 90\%
of maximum greenness or greater. Coverage is planned to include at least
95\% of NIS Tower Airshed area as well as at least 80\% of a minimum
10km x 10 km box around that. All acquisitions are subject to change due
to weather conditions as well as program planning changes.

\begin{center}\rule{0.5\linewidth}{\linethickness}\end{center}

\textbf{Description}

Normalized index of xanthophyll concentration; data are provided by
flightline

\begin{center}\rule{0.5\linewidth}{\linethickness}\end{center}

\textbf{Abstract}

Canopy Xanthophyll, or Photochemical Reflectance Index (PRI), is a
reflectance ratio index that is sensitive to changes in carotenoid
pigments, particularly xanthophyll pigments, in live foliage (Gamon,
Penuelas, \& Field, 1992). Carotenoid pigments are proxies for
photosynthetic light use efficiency, or the rate of carbon dioxide
uptake by foliage per unit energy absorbed. PRI is used in studies of
vegetation productivity and stress. Applications include vegetation
health in evergreen shrublands, forests, and agricultural crops prior to
senescence. L2 Canopy Xanthophyll is distributed in the original
North/South flight lines.

*Note: Data are being migrated to the data portal. If you don't find the
data you are looking for (e.g., from specific sites or years), please
request data
\href{http://www.neonscience.org/request-airborne-data}{here}. \newpage
.

\section{DP2.30022.001 Canopy lignin -
flightline}\label{dp2.30022.001-canopy-lignin---flightline}

\begin{center}\rule{0.5\linewidth}{\linethickness}\end{center}

\textbf{Subsystem}

Airborne Observation Platform (AOP)

\begin{center}\rule{0.5\linewidth}{\linethickness}\end{center}

\textbf{Coverage}

NEON AOP data are planned for yearly collects at all NEON sites at 90\%
of maximum greenness or greater. Coverage is planned to include at least
95\% of NIS Tower Airshed area as well as at least 80\% of a minimum
10km x 10 km box around that. All acquisitions are subject to change due
to weather conditions as well as program planning changes.

\begin{center}\rule{0.5\linewidth}{\linethickness}\end{center}

\textbf{Description}

Normalized index of canopy lignin concentration; data are provided by
flightline

\begin{center}\rule{0.5\linewidth}{\linethickness}\end{center}

\textbf{Abstract}

Lignin, or Normalized Difference Lignin Index (NDLI), estimates the
relative amounts of lignin contained in vegetation canopies. Leaf lignin
concentration and canopy foliage biomass are the determining factors for
vegetation reflectance spectra at 1754 nm. NDLI uses leaf lignin
concentration and canopy foliar biomass, as combined in the 1750 nm
range, as a means for predicting total canopy lignin content. NDLI is
most frequently used for ecosystem analysis and detection of surface
plant litter. (Serrano, Penuelas, \& Ustin, 2002) L2 Canopy Lignin is
distributed in the original North/South flight lines.

*Note: Data are being migrated to the data portal. If you don't find the
data you are looking for (e.g., from specific sites or years), please
request data
\href{http://www.neonscience.org/request-airborne-data}{here}. \newpage
.

\section{DP2.30026.001 Vegetation indices - spectrometer -
flightline}\label{dp2.30026.001-vegetation-indices---spectrometer---flightline}

\begin{center}\rule{0.5\linewidth}{\linethickness}\end{center}

\textbf{Subsystem}

Airborne Observation Platform (AOP)

\begin{center}\rule{0.5\linewidth}{\linethickness}\end{center}

\textbf{Coverage}

NEON AOP data are planned for yearly collects at all NEON sites at 90\%
of maximum greenness or greater. Coverage is planned to include at least
95\% of NIS Tower Airshed area as well as at least 80\% of a minimum
10km x 10 km box around that. All acquisitions are subject to change due
to weather conditions as well as program planning changes.

\begin{center}\rule{0.5\linewidth}{\linethickness}\end{center}

\textbf{Description}

NDVI - Normalized ratio of NIR and IR bands; characterizes the ``red
edge'' in vegetation spectra. SAVI - Normalized ratio of 850 nm and 650
nm bands with gain and offset factors to minimize soil contribution in
result; primary input to LAI product. EVI - Normalized ratio of NIR and
IR bans (red edge characterization); includes Blue channel for better
aerosol characterization. Data are provided by flightline; additional
indices will be assessed and added to this product

\begin{center}\rule{0.5\linewidth}{\linethickness}\end{center}

\textbf{Abstract}

The Vegetation Indices data product is a family of 4 spectral indices:
NDVI, EVI, ARVI, and SAVI. These indices use regions of vegetation
reflectance spectra known to be indicators of vegetation health,
vegetation health in high LAI areas, vegetation health in lush and/or
humid regions, and vegetation health in mixed soil and vegetation
landcover areas, respectively. The indices are derived from NEON
Airborne Observation Platform (AOP) Imaging Spectrometer (NIS) data
collected in North-South oriented flight lines to reduce BRDF effects.
These data are processed to orthorectified directional surface
reflectance and then processed to the indices. L2 Vegetation Indices are
distributed in the original North/South flight lines and are packaged as
a zip file containg one GeoTIFF for each index. The Level 2 vegetation
indices are distributed in their original North-South flight lines.
\newpage
.

\section{DP3.00008.001 Temperature rate of change
profile}\label{dp3.00008.001-temperature-rate-of-change-profile}

\begin{center}\rule{0.5\linewidth}{\linethickness}\end{center}

\textbf{Subsystem}

Terrestrial Instrument System (TIS)

\begin{center}\rule{0.5\linewidth}{\linethickness}\end{center}

\textbf{Coverage}

These data are collected at all NEON terrestrial sites.

\begin{center}\rule{0.5\linewidth}{\linethickness}\end{center}

\textbf{Description}

Time rate of change of temperature (storage component only) over 30 min,
spatially interpolated along the vertical tower profile. This data
product is bundled into DP4.00200, Bundled data products - eddy
covariance, and is not available as a stand-alone download.

\begin{center}\rule{0.5\linewidth}{\linethickness}\end{center}

\textbf{Abstract}

This data product contains spatially interpolated temperature data at a
0.1 m vertical interval based on the 30 minute time rate of change for
temperature at different measurement levels on the tower. The data are
delivered with the Bundled data products - eddy covariance data product
(DP4.00200.001).

\begin{center}\rule{0.5\linewidth}{\linethickness}\end{center}

\textbf{Usage Notes}

During subsequent nominal operations, we plan to produce and publish the
data products in three phases, to accommodate a variety of use cases:
the initial near-real-time transition, a science reviewed quality
transition, and the epoch yearly transition. The initial near-real-time
transition is scheduled to process daily files at a 5-day delay after
data collection to accommodate a 9-day centered planar-fit window. If
the data has not been received from the field it will attempt to process
daily for 30\,days, and if not all data is available after this window a
force execution is performed populating a HDF5 file with metadata and
filling data with NaN's. The monthly file will be produced after all
daily files are available, no later than 30 days after the last daily
file was initially attempted to be processed. After the initial
transition, the NEON science team has a one month window to manually
flag data that were identified as suspect through field-based problem
tracking and resolution tickets or through additional manual data
quality analysis. Then, the science-reviewed transition will occur, and
the data will be republished to the data portal. The last transition
type is part of the yearly epoch versioning, which provides a fully
quality assured and quality controlled version of the data using the
latest full release of the processing code. This transition is scheduled
to occur 18 months after the initial data collection. \newpage
.

\section{DP3.00009.001 CO2 concentration rate of change
profile}\label{dp3.00009.001-co2-concentration-rate-of-change-profile}

\begin{center}\rule{0.5\linewidth}{\linethickness}\end{center}

\textbf{Subsystem}

Terrestrial Instrument System (TIS)

\begin{center}\rule{0.5\linewidth}{\linethickness}\end{center}

\textbf{Coverage}

These data are measured at all terrestrial sites. Sensors are located
inside the instrument hut near the bottom of the tower. The air samples
from different measurement heights are pumped through gas tubing to
sensor for analysis.

\begin{center}\rule{0.5\linewidth}{\linethickness}\end{center}

\textbf{Description}

Time rate of change of CO2 concentration (storage component only) over
30 min, spatially interpolated along the vertical tower profile. This
data product is bundled into DP4.00200, Bundled data products - eddy
covariance, and is not available as a stand-alone download.

\begin{center}\rule{0.5\linewidth}{\linethickness}\end{center}

\textbf{Abstract}

This data product contains spatially interpolated CO2 data at a 0.1 m
vertical interval based on the 30 minute time rate of change for CO2
molar fraction at different measurement levels on the tower. The data
are delivered with the Bundled data products - eddy covariance data
product (DP4.00200.001).

\begin{center}\rule{0.5\linewidth}{\linethickness}\end{center}

\textbf{Usage Notes}

During subsequent nominal operations, we plan to produce and publish the
data products in three phases, to accommodate a variety of use cases:
the initial near-real-time transition, a science reviewed quality
transition, and the epoch yearly transition. The initial near-real-time
transition is scheduled to process daily files at a 5-day delay after
data collection to accommodate a 9-day centered planar-fit window. If
the data has not been received from the field it will attempt to process
daily for 30\,days, and if not all data is available after this window a
force execution is performed populating a HDF5 file with metadata and
filling data with NaN's. The monthly file will be produced after all
daily files are available, no later than 30 days after the last daily
file was initially attempted to be processed. After the initial
transition, the NEON science team has a one month window to manually
flag data that were identified as suspect through field-based problem
tracking and resolution tickets or through additional manual data
quality analysis. Then, the science-reviewed transition will occur, and
the data will be republished to the data portal. The last transition
type is part of the yearly epoch versioning, which provides a fully
quality assured and quality controlled version of the data using the
latest full release of the processing code. This transition is scheduled
to occur 18 months after the initial data collection. \newpage
.

\section{DP3.00010.001 H2O concentration rate of change
profile}\label{dp3.00010.001-h2o-concentration-rate-of-change-profile}

\begin{center}\rule{0.5\linewidth}{\linethickness}\end{center}

\textbf{Subsystem}

Terrestrial Instrument System (TIS)

\begin{center}\rule{0.5\linewidth}{\linethickness}\end{center}

\textbf{Coverage}

Please see the Bundled Eddy Covariance (DP4.00200.001) data product for
more information.

\begin{center}\rule{0.5\linewidth}{\linethickness}\end{center}

\textbf{Description}

Time rate of change of H2O concentration (storage component only) over
30 min, spatially interpolated along the vertical tower profile. This
data product is bundled into DP4.00200, Bundled data products - eddy
covariance, and is not available as a stand-alone download.

\begin{center}\rule{0.5\linewidth}{\linethickness}\end{center}

\textbf{Abstract}

This data product contains spatially interpolated H2O data at a 0.1 m
vertical interval based on the 30 minute time rate of change for H2O
molar fraction at different measurement levels on the tower. The data
are delivered with the Bundled Eddy Covariance (DP4.00200.001) data
product.

\begin{center}\rule{0.5\linewidth}{\linethickness}\end{center}

\textbf{Usage Notes}

During subsequent nominal operations, we plan to produce and publish the
data products in three phases, to accommodate a variety of use cases:
the initial near-real-time transition, a science reviewed quality
transition, and the epoch yearly transition. The initial near-real-time
transition is scheduled to process daily files at a 5-day delay after
data collection to accommodate a 9-day centered planar-fit window. If
the data has not been received from the field it will attempt to process
daily for 30\,days, and if not all data is available after this window a
force execution is performed populating a HDF5 file with metadata and
filling data with NaN's. The monthly file will be produced after all
daily files are available, no later than 30 days after the last daily
file was initially attempted to be processed. After the initial
transition, the NEON science team has a one month window to manually
flag data that were identified as suspect through field-based problem
tracking and resolution tickets or through additional manual data
quality analysis. Then, the science-reviewed transition will occur, and
the data will be republished to the data portal. The last transition
type is part of the yearly epoch versioning, which provides a fully
quality assured and quality controlled version of the data using the
latest full release of the processing code. This transition is scheduled
to occur 18 months after the initial data collection. \newpage
.

\section{DP3.30006.001}\label{dp3.30006.001}

Spectrometer orthorectified surface directional reflectance - mosaic

\begin{center}\rule{0.5\linewidth}{\linethickness}\end{center}

\textbf{Subsystem}

Airborne Observation Platform (AOP)

\begin{center}\rule{0.5\linewidth}{\linethickness}\end{center}

\textbf{Coverage}

NEON AOP data are planned for yearly collects at all NEON sites at 90\%
of maximum greenness or greater. Coverage is planned to include at least
95\% of NIS Tower Airshed area as well as at least 80\% of a minimum
10km x 10 km box around that. All acquisitions are subject to change due
to weather conditions as well as program planning changes.

\begin{center}\rule{0.5\linewidth}{\linethickness}\end{center}

\textbf{Description}

Orthorectified Surface reflectance (0-1 unitless, scaled by 10,000)
computed from the NEON Imaging Spectrometer (NIS) per pixel; data are
orthorectified and output onto a fixed, uniform spatial grid using
nearest-neighbor resampling (tbr). Level 1 flight lines over a given
site are mosaicked into single product; spatial resolution is 1m.

\begin{center}\rule{0.5\linewidth}{\linethickness}\end{center}

\textbf{Abstract}

The NEON AOP surface directional reflectance data product is an
orthorectified (UTM projection) hyperspectral raster product. It is
distributed in an open HDF5 format including all 426 bands from the NEON
Imaging Spectrometer. It is a calibrated and atmospherically corrected
product distributed as scaled reflectance. It includes many QA and
ancillary rasters used as inputs to ATCOR for atmospheric correction as
well as outputs from ATCOR for diagnostic purposes. L3 reflectance is
distributed in 1 km by 1 km tiles with one HDF5 file per tile including
the reflectance data and all metadata and ancillary data. The mosaic is
created using the most-nadir pixels from the flight lines covering the
tile. \newpage
.

\section{DP3.30010.001 High-resolution orthorectified camera imagery
mosaic}\label{dp3.30010.001-high-resolution-orthorectified-camera-imagery-mosaic}

\begin{center}\rule{0.5\linewidth}{\linethickness}\end{center}

\textbf{Subsystem}

Airborne Observation Platform (AOP)

\begin{center}\rule{0.5\linewidth}{\linethickness}\end{center}

\textbf{Coverage}

NEON AOP data are planned for yearly collects at all NEON sites at 90\%
of maximum greenness or greater. Coverage is planned to include at least
95\% of NIS Tower Airshed area as well as at least 80\% of a minimum
10km x 10 km box around that. All acquisitions are subject to change due
to weather conditions as well as program planning changes.

\begin{center}\rule{0.5\linewidth}{\linethickness}\end{center}

\textbf{Description}

Level 1 high-resolution orthorectified camera images are mosaiced and
tiled into 1 km by 1 km data sets. Mosiac is output onto a fixed,
uniform spatial grid using nearest-neighbor resampling; spatial
resolution is maintained at 0.1 m.

\begin{center}\rule{0.5\linewidth}{\linethickness}\end{center}

\textbf{Abstract}

The digital camera is part of a suite of instruments on the NEON
Airborne Observation Platform (AOP) that also includes a full-waveform,
small-footprint LiDAR system and the NEON Imaging Spectrometer. In the
orthorectification process, the digital imagery is remapped to the same
geographic projection as the LiDAR and imaging spectrometer data that is
acquired simultaneously. The resulting images will share the same map
projection grid space as the orthorectified spectrometer and LiDAR
imagery. Since the digital camera imagery is acquired at higher spatial
resolution than the imaging spectrometer data, it can aid in identifying
features in the spectrometer images including manmade features (e.g.,
roads, fence lines, and buildings) that are indicative of land-use
change. Level 3 RGB camera images are distributed as mosaics of
individual camera images in 1 km by 1 km tiles in GeoTIFF format.
\newpage
.

\section{DP3.30011.001 Albedo - spectrometer -
mosiac}\label{dp3.30011.001-albedo---spectrometer---mosiac}

\begin{center}\rule{0.5\linewidth}{\linethickness}\end{center}

\textbf{Subsystem}

Airborne Observation Platform (AOP)

\begin{center}\rule{0.5\linewidth}{\linethickness}\end{center}

\textbf{Coverage}

NEON AOP data are planned for yearly collects at all NEON sites at 90\%
of maximum greenness or greater. Coverage is planned to include at least
95\% of NIS Tower Airshed area as well as at least 80\% of a minimum
10km x 10 km box around that. All acquisitions are subject to change due
to weather conditions as well as program planning changes.

\begin{center}\rule{0.5\linewidth}{\linethickness}\end{center}

\textbf{Description}

Total amount of solar radiation in the 0.4 to 2.5 micron band reflected
by the Earth's surface into an upward hemisphere divided by the total
amount incident from this hemisphere

\begin{center}\rule{0.5\linewidth}{\linethickness}\end{center}

\textbf{Abstract}

Albedo, the ratio of a surface's reflected energy to its incident
energy, is an important measurement for characterizing earth system
energy balance. Light and dark surfaces correspond to high and low
albedo, respectively. An opaque surface's difference in energy reflected
as compared to the energy incident on it is absorbed by the surface,
increasing its temperature. (Sabins, Jr., 1978). Albedo values depend on
wavelength, illumination sources and geometry, sensor viewing geometry,
reflectance as a function of angle and wavelength, as well as the
scattering, absorbing, and re-radiating effects of the atmosphere. These
factors are modeled/accounted for to best approximate a bi-hemispherical
reflectance as would be measured in a laboratory setting. To this end,
the wavelength-integrated surface reflectance, weighted with the global
flux on the ground, is produced as the best practically achievable
albedo measurement. (Richter \& Schlapfer, 2017) L3 Albedo is
distributed in 1km tiles created by taking the most-nadir pixels from
the clearest flightlines acquired for each pixel. \newpage
.

\section{DP3.30012.001 LAI - spectrometer -
mosaic}\label{dp3.30012.001-lai---spectrometer---mosaic}

\begin{center}\rule{0.5\linewidth}{\linethickness}\end{center}

\textbf{Subsystem}

Airborne Observation Platform (AOP)

\begin{center}\rule{0.5\linewidth}{\linethickness}\end{center}

\textbf{Coverage}

NEON AOP data are planned for yearly collects at all NEON sites at 90\%
of maximum greenness or greater. Coverage is planned to include at least
95\% of NIS Tower Airshed area as well as at least 80\% of a minimum
10km x 10 km box around that. All acquisitions are subject to change due
to weather conditions as well as program planning changes.

\begin{center}\rule{0.5\linewidth}{\linethickness}\end{center}

\textbf{Description}

The ratio of upper leaf surface area to ground area (for broadleaf
canopies), or projected conifer needle surface area to ground area (for
coniferous plants) for a given unit area; Level 2 products derived from
individual flight lines over a given site are mosaiced into single
product; spatial resolution is 1m.

\begin{center}\rule{0.5\linewidth}{\linethickness}\end{center}

\textbf{Abstract}

The leaf area index (LAI) is a derived spectral product from remotely
sensed data that is used as a proxy for describing leaf area across
areas larger than can be measured by more direct ground-based
measurements such as hemispherical photography. It is often used as an
input layer for productivity, landscape, and climate models. The Level 3
LAI product is distributed in 1 km by 1 km mosaic tiles using the
most-nadir pixels from the original flight lines and is in GeoTIFF
format. \newpage
.

\section{DP3.30014.001 fPAR - spectrometer -
mosaic}\label{dp3.30014.001-fpar---spectrometer---mosaic}

\begin{center}\rule{0.5\linewidth}{\linethickness}\end{center}

\textbf{Subsystem}

Airborne Observation Platform (AOP)

\begin{center}\rule{0.5\linewidth}{\linethickness}\end{center}

\textbf{Coverage}

NEON AOP data are planned for yearly collects at all NEON sites at 90\%
of maximum greenness or greater. Coverage is planned to include at least
95\% of NIS Tower Airshed area as well as at least 80\% of a minimum
10km x 10 km box around that. All acquisitions are subject to change due
to weather conditions as well as program planning changes.

\begin{center}\rule{0.5\linewidth}{\linethickness}\end{center}

\textbf{Description}

The fraction of incident photosynthetically active radiation (400-700
nm) absorbed by the green elements of a vegetation canopy; mosaiced from
the fPAR level 2 product onto a spatially uniform grid at 1 m spatial
resolution and provided as 1 km by 1 km tiles.

\begin{center}\rule{0.5\linewidth}{\linethickness}\end{center}

\textbf{Abstract}

The fraction of photosynthetically active radiation (fPAR) describes the
relative quantity of incident solar radiation of relevant
photosynthetically active wavelengths (0.4-0.7 microns) absorbed by
vegetative material. The fPAR is an important biophysical variable used
in the simulation of water, carbon and nutrient cycling due to its
relationship with vegetative productivity. Theoretically, if a plant is
able to intercept and absorb relevant photosynthetically active
wavelengths, this will result in a higher state of productivity, gas
exchange and transpiration. The application of this theory is critical
in assessments of productivity change through time, and simulation of
climate models to predict ecosystems response to climate variability.
The level 3 version of the fPAR product is distributed in 1 km by 1 km
tiles which are mosaics of the most-nadir pixels of the flight lines
collected. \newpage
.

\section{DP3.30015.001 Ecosystem
structure}\label{dp3.30015.001-ecosystem-structure}

\begin{center}\rule{0.5\linewidth}{\linethickness}\end{center}

\textbf{Subsystem}

Airborne Observation Platform (AOP)

\begin{center}\rule{0.5\linewidth}{\linethickness}\end{center}

\textbf{Coverage}

NEON AOP data are planned for yearly collects at all NEON sites at 90\%
of maximum greenness or greater. Coverage is planned to include at least
95\% of NIS Tower Airshed area as well as at least 80\% of a minimum
10km x 10 km box around that. All acquisitions are subject to change due
to weather conditions as well as program planning changes.

\begin{center}\rule{0.5\linewidth}{\linethickness}\end{center}

\textbf{Description}

Height of the top of canopy above bare earth; data are mosaicked over
AOP footprint; mosaicked onto a spatially uniform grid at 1 m spatial
resolution in 1 km by 1 km tiles.

\begin{center}\rule{0.5\linewidth}{\linethickness}\end{center}

\textbf{Abstract}

Forests store and sequester a considerable proportion of the terrestrial
global carbon budget. Forest canopy metrics are directly measurable with
LiDAR sensors because laser pulses will be reflected from the uppermost
canopy layers and remaining energy will penetrate to, and reflect from,
under-story and the ground surface. The near simultaneous direct
measurement of ground and canopy elevation allows the canopy height to
be estimated through differencing. The CHM is generated by creating a
continuous surface of canopy height estimates across the entire spatial
domain of the LiDAR survey. The CHM is derived directly from the LiDAR
point cloud. The LiDAR point cloud is produced from LiDAR return signals
from both surface features and the true-ground as LiDAR pulses will be
reflected from the uppermost layers of the canopy, as well as the
underlying ground surface. To produce the CHM, the point cloud is
separated into classes representing the ground and vegetation returns.
The ground classified points allow calculation of a height normalized
point cloud to provide a relative estimate of vegetation elevation. A
surface is then generated using the height normalized vegetation points
to produce the CHM.

*Note: Data are being migrated to the data portal. If you don't find the
data you are looking for (e.g., from specific sites or years), please
request data
\href{http://www.neonscience.org/request-airborne-data}{here}. \newpage
.

\section{DP3.30016.001 Total biomass map - spectrometer -
mosaic}\label{dp3.30016.001-total-biomass-map---spectrometer---mosaic}

\begin{center}\rule{0.5\linewidth}{\linethickness}\end{center}

\textbf{Subsystem}

Airborne Observation Platform (AOP)

\begin{center}\rule{0.5\linewidth}{\linethickness}\end{center}

\textbf{Coverage}

NEON AOP data are planned for yearly collects at all NEON sites at 90\%
of maximum greenness or greater. Coverage is planned to include at least
95\% of NIS Tower Airshed area as well as at least 80\% of a minimum
10km x 10 km box around that. All acquisitions are subject to change due
to weather conditions as well as program planning changes.

\begin{center}\rule{0.5\linewidth}{\linethickness}\end{center}

\textbf{Description}

Mass of all aboveground organic matter per unit area at particular time;
estimate of biomass derived from correlation with NDVI and LAI
parameters; mosaicked from the total biomass level 2 product onto a
spatially uniform grid at 1 m spatial resolution and provided as 1 km by
1 km tiles.

\begin{center}\rule{0.5\linewidth}{\linethickness}\end{center}

\textbf{Abstract}

The NEON AOP Total Biomass data product is an orthorectified (UTM
projection) raster product derived from NEON AOP Imaging Spectrometer
(NIS) reflectance data. Biomass is an important layer in models and
measurements involving climate, landscape ecology, and the carbon cycle.
Remotely sensed estimates of biomass are important links between ground
based biomass measurements and models operating at landscape, regional,
or global scales. The Level 3 mosaic biomass product is distributed in 1
km by 1 km tiles in GeoTIFF format with each file containing the biomass
raster for a single tile. The mosaic is created using the most-nadir
pixel values from the single flight line biomass products intersecting
the tile. \newpage
.

\section{DP3.30018.001 Canopy nitrogen -
mosaic}\label{dp3.30018.001-canopy-nitrogen---mosaic}

\begin{center}\rule{0.5\linewidth}{\linethickness}\end{center}

\textbf{Subsystem}

Airborne Observation Platform (AOP)

\begin{center}\rule{0.5\linewidth}{\linethickness}\end{center}

\textbf{Coverage}

NEON AOP data are planned for yearly collects at all NEON sites at 90\%
of maximum greenness or greater. Coverage is planned to include at least
95\% of NIS Tower Airshed area as well as at least 80\% of a minimum
10km x 10 km box around that. All acquisitions are subject to change due
to weather conditions as well as program planning changes.

\begin{center}\rule{0.5\linewidth}{\linethickness}\end{center}

\textbf{Description}

Measure of canopy nitrogen concentration in remotely sensed data. Level
2 products derived from individual flight lines over a given site are
mosaicked into single product; spatial resolution is 1m.

\begin{center}\rule{0.5\linewidth}{\linethickness}\end{center}

\textbf{Abstract}

Canopy Nitrogen, or Normalized Difference Nitrogen Index (NDNI),
estimates the relative amounts of nitrogen in vegetation land cover. The
index uses reflectance at 1510 nm (determined largely by nitrogen
concentration in plants and foliar biomass) and at 1680 nm (sensitive to
biomass but not to nitrogen absorption). NDNI is a relatively new
spectral index in remote sensing. L3 NDNI is distributed in 1 km square
tiles with 1 m pixels whose values are taken from the most-nadir pixel
from the original flight line collections. \newpage
.

\section{DP3.30019.001 Canopy water content -
mosaic}\label{dp3.30019.001-canopy-water-content---mosaic}

\begin{center}\rule{0.5\linewidth}{\linethickness}\end{center}

\textbf{Subsystem}

Airborne Observation Platform (AOP)

\begin{center}\rule{0.5\linewidth}{\linethickness}\end{center}

\textbf{Coverage}

NEON AOP data are planned for yearly collects at all NEON sites at 90\%
of maximum greenness or greater. Coverage is planned to include at least
95\% of NIS Tower Airshed area as well as at least 80\% of a minimum
10km x 10 km box around that. All acquisitions are subject to change due
to weather conditions as well as program planning changes.

\begin{center}\rule{0.5\linewidth}{\linethickness}\end{center}

\textbf{Description}

Normalized index of canopy water content. Level 2 products derived from
individual flight lines over a given site are mosaicked into single
product; spatial resolution is 1m.

\begin{center}\rule{0.5\linewidth}{\linethickness}\end{center}

\textbf{Abstract}

The Canopy Water Content data products are a family of 5 spectral
indices: MSI, NDII, NDWI, NMDI, and WBI. These indices use regions
vegetation reflectance spectra known to be indicators of leaf water
content, relative canopy water content, changes in canopy water content,
soil and canopy water content, and changes in canopy water status,
respectively. L2 Canopy Water Content is distributed in 1 km square
tiles with 1 m pixels whose values are taken from the most-nadir pixel
from the original flight line collections. Each tile is packaged as a
zip file containing one GeoTIFF for each index. \newpage
.

\section{DP3.30020.001 Canopy xanthophyll cycle -
mosaic}\label{dp3.30020.001-canopy-xanthophyll-cycle---mosaic}

\begin{center}\rule{0.5\linewidth}{\linethickness}\end{center}

\textbf{Subsystem}

Airborne Observation Platform (AOP)

\begin{center}\rule{0.5\linewidth}{\linethickness}\end{center}

\textbf{Coverage}

NEON AOP data are planned for yearly collects at all NEON sites at 90\%
of maximum greenness or greater. Coverage is planned to include at least
95\% of NIS Tower Airshed area as well as at least 80\% of a minimum
10km x 10 km box around that. All acquisitions are subject to change due
to weather conditions as well as program planning changes.

\begin{center}\rule{0.5\linewidth}{\linethickness}\end{center}

\textbf{Description}

Normalized index of xanthophyll concentration. Level 2 products derived
from individual flight lines over a given site are mosaicked into single
product; spatial resolution is 1m.

\begin{center}\rule{0.5\linewidth}{\linethickness}\end{center}

\textbf{Abstract}

Canopy Xanthophyll, or Photochemical Reflectance Index (PRI), is a
reflectance ratio index that is sensitive to changes in carotenoid
pigments, particularly xanthophyll pigments, in live foliage (Gamon,
Penuelas, \& Field, 1992). Carotenoid pigments are proxies for
photosynthetic light use efficiency, or the rate of carbon dioxide
uptake by foliage per unit energy absorbed. PRI is used in studies of
vegetation productivity and stress. Applications include vegetation
health in evergreen shrublands, forests, and agricultural crops prior to
senescence. L3 Canopy Xanthophyll is distributed in 1 km square tiles
with 1 m pixels whose values are taken from the most-nadir pixel from
the original flight line collections. \newpage
.

\section{DP3.30022.001 Canopy lignin -
mosaic}\label{dp3.30022.001-canopy-lignin---mosaic}

\begin{center}\rule{0.5\linewidth}{\linethickness}\end{center}

\textbf{Subsystem}

Airborne Observation Platform (AOP)

\begin{center}\rule{0.5\linewidth}{\linethickness}\end{center}

\textbf{Coverage}

NEON AOP data are planned for yearly collects at all NEON sites at 90\%
of maximum greenness or greater. Coverage is planned to include at least
95\% of NIS Tower Airshed area as well as at least 80\% of a minimum
10km x 10 km box around that. All acquisitions are subject to change due
to weather conditions as well as program planning changes.

\begin{center}\rule{0.5\linewidth}{\linethickness}\end{center}

\textbf{Description}

Normalized index of canopy lignin concentration. Level 2 products
derived from individual flight lines over a given site are mosaiced into
single product; spatial resolution is 1m.

\begin{center}\rule{0.5\linewidth}{\linethickness}\end{center}

\textbf{Abstract}

Canopy Lignin, or Normalized Difference Lignin Index (NDLI), estimates
the relative amounts of lignin contained in vegetation canopies. Leaf
lignin concentration and canopy foliage biomass are the determining
factors for vegetation reflectance spectra at 1754 nm. NDLI uses leaf
lignin concentration and canopy foliar biomass, as combined in the 1750
nm range, as a means for predicting total canopy lignin content. NDLI is
most frequently used for ecosystem analysis and detection of surface
plant litter. (Serrano, Penuelas, \& Ustin, 2002) L3 Canopy Lignin is
distributed in 1 km square tiles with 1 m pixels whose values are taken
from the most-nadir pixel from the original flight line collections.
\newpage
.

\section{DP3.30024.001 Elevation -
LiDAR}\label{dp3.30024.001-elevation---lidar}

\begin{center}\rule{0.5\linewidth}{\linethickness}\end{center}

\textbf{Subsystem}

Airborne Observation Platform (AOP)

\begin{center}\rule{0.5\linewidth}{\linethickness}\end{center}

\textbf{Coverage}

NEON AOP data are planned for yearly collects at all NEON sites at 90\%
of maximum greenness or greater. Coverage is planned to include at least
95\% of NIS Tower Airshed area as well as at least 80\% of a minimum
10km x 10 km box around that. All acquisitions are subject to change due
to weather conditions as well as program planning changes.

\begin{center}\rule{0.5\linewidth}{\linethickness}\end{center}

\textbf{Description}

Bare earth elevation given in meters above mean sea level (topographic
information with vegetation and man-made structures removed) and
mosaicked onto a spatially uniform grid at 1 m spatial resolution in 1
km by 1 km tiles. Surface features given in meters above mean sea level
(topographic information with vegetation and man-made structures
removed) and mosaicked onto a spatially uniform grid at 1 m spatial
resolution in 1 km by 1 km tiles.

\begin{center}\rule{0.5\linewidth}{\linethickness}\end{center}

\textbf{Abstract}

The elevation product, in the form of a DTM, provides information on
terrain structure, and is an important data layer in spatially driven
models of landscape processes, and these models allow for spatially
explicit predictability of phenomena internal and external to the
landscape. Currently, LIDAR sensors provide the most efficient means for
collecting an accurate and dense sample of the terrain among competing
remote sensing or positioning systems. For example, high-resolution
digital stereo photogrammetry can compete in terms of point density in
open terrain, but suffers from sparse sampling beneath tree canopy. The
DSM provides two important functions as complimentary information to the
optical sensors on the AOP. The first function is strictly as a tool in
the geolocation processing of the hyperspectral sensor and the RGB
digital camera. The DSM also provides information on the structure of
surface features, including derived vegetation structure, which can be
used as a proxy to estimate important ecological quantities of interest.

*Note: Data are being migrated to the data portal. If you don't find the
data you are looking for (e.g., from specific sites or years), please
request data
\href{http://www.neonscience.org/request-airborne-data}{here}. \newpage
.

\section{DP3.30025.001 Slope and Aspect -
LiDAR}\label{dp3.30025.001-slope-and-aspect---lidar}

\begin{center}\rule{0.5\linewidth}{\linethickness}\end{center}

\textbf{Subsystem}

Airborne Observation Platform (AOP)

\begin{center}\rule{0.5\linewidth}{\linethickness}\end{center}

\textbf{Coverage}

NEON AOP data are planned for yearly collects at all NEON sites at 90\%
of maximum greenness or greater. Coverage is planned to include at least
95\% of NIS Tower Airshed area as well as at least 80\% of a minimum
10km x 10 km box around that. All acquisitions are subject to change due
to weather conditions as well as program planning changes.

\begin{center}\rule{0.5\linewidth}{\linethickness}\end{center}

\textbf{Description}

Slope is a ratio of rise over run (height over distance) of the bare
earth elevation product given in degrees; aspect is the direction of the
steepest slope of the bare earth elevation product (e.g., north, east,
south, west) given in degrees clockwise from grid north; both mosaicked
onto a spatially uniform grid at 1 m spatial resolution in 1 km by 1 km
tiles.

\begin{center}\rule{0.5\linewidth}{\linethickness}\end{center}

\textbf{Abstract}

The NEON AOP LiDAR Slope and Aspect product includes a slope map and
aspect map, both in raster GeoTIFF format. Slope and aspect maps are
derived from the DTM, which includes only elevations which relate to the
physical terrain or ``bare earth'' surface model. Raster maps for the
slope and aspect are reported with horizontal reference to the ITRF00
datum and projected to the Universal Transverse Mercator (UTM) mapping
frame. Slope is determined as the angle between a plane tangential to
the local terrain surface and a plane tangential to the local Geoid12A
surface, reported in degrees. Aspect is the direction of the steepest
slope, given in degrees referenced to grid north. The slope and aspect
rasters are divided into a set of 1 km by 1 km tiles, which have corners
spatially referenced to an even kilometer. \newpage
.

\section{DP3.30026.001 Vegetation indices - spectrometer -
mosaic}\label{dp3.30026.001-vegetation-indices---spectrometer---mosaic}

\begin{center}\rule{0.5\linewidth}{\linethickness}\end{center}

\textbf{Subsystem}

Airborne Observation Platform (AOP)

\begin{center}\rule{0.5\linewidth}{\linethickness}\end{center}

\textbf{Coverage}

NEON AOP data are planned for yearly collects at all NEON sites at 90\%
of maximum greenness or greater. Coverage is planned to include at least
95\% of NIS Tower Airshed area as well as at least 80\% of a minimum
10km x 10 km box around that. All acquisitions are subject to change due
to weather conditions as well as program planning changes.

\begin{center}\rule{0.5\linewidth}{\linethickness}\end{center}

\textbf{Description}

NDVI - Normalized ratio of NIR and IR bands; characterizes the ``red
edge'' in vegetation spectra. SAVI - Normalized ratio of 850 nm and 650
nm bands with gain and offset factors to minimize soil contribution in
result; primary input to LAI product. EVI - Normalized ratio of NIR and
IR bans (red edge characterization); includes Blue channel for better
aerosol characterization. Level 2 products derived from individual
flight lines over a given site are mosaiced into single product; spatial
resolution is 1m.

\begin{center}\rule{0.5\linewidth}{\linethickness}\end{center}

\textbf{Abstract}

The Vegetation Indices data product is a family of 4 spectral indices:
NDVI, EVI, ARVI, and SAVI. These indices use regions of vegetation
reflectance spectra known to be indicators of vegetation health,
vegetation health in high LAI areas, vegetation health in lush and/or
humid regions, and vegetation health in mixed soil and vegetation
landcover areas, respectively. L3 Vegetation Indices are distributed in
1 km square tiles with 1 m pixels whose values are taken from the
most-nadir pixel from the original flight line collections. \newpage
.

\section{DP4.00001.001 Summary weather
statistics}\label{dp4.00001.001-summary-weather-statistics}

\begin{center}\rule{0.5\linewidth}{\linethickness}\end{center}

\textbf{Subsystem}

Terrestrial Instrument System (TIS)

\begin{center}\rule{0.5\linewidth}{\linethickness}\end{center}

\textbf{Coverage}

Summary weather statistics are generated for each Core terrestrial site
in all 20 of NEON's domains.

\begin{center}\rule{0.5\linewidth}{\linethickness}\end{center}

\textbf{Description}

Present summary statistics for biometeorological variables for NEON
weather stations at core TIS sites. Statistics will include means,
standard deviations, maxima, and minima for periods of days, months, and
years. Engineering-grade product only.

\begin{center}\rule{0.5\linewidth}{\linethickness}\end{center}

\textbf{Abstract}

The data products used for computing summary weather statistics
represent fundamental meteorological parameters and are commonly
monitored by many meteorological networks (e.g., USCRN, SCAN, etc.).
Summaries of these meteorologic parameters are useful for understanding
trends and changes in weather patterns. \newpage
.

\section{DP4.00002.001 Sensible heat
flux}\label{dp4.00002.001-sensible-heat-flux}

\begin{center}\rule{0.5\linewidth}{\linethickness}\end{center}

\textbf{Subsystem}

Terrestrial Instrument System (TIS)

\begin{center}\rule{0.5\linewidth}{\linethickness}\end{center}

\textbf{Coverage}

Data are collected at all terrestrial sites, along the tower profile
from the ground to the tower top above the canopy, in order to study the
ecosystem exchange of scalars (CO2, H2O, etc.) and energy between the
atmosphere and the ecosystem of interest.

\begin{center}\rule{0.5\linewidth}{\linethickness}\end{center}

\textbf{Description}

Sensible heat flux is estimated based on the eddy covariance technique
using a sonic anemometer to measure vertical winds and air temperature
and tower profile measurements of air temperature. This data product is
bundled into DP4.00200, Bundled data products - eddy covariance, and is
not available as a stand-alone download.

\begin{center}\rule{0.5\linewidth}{\linethickness}\end{center}

\textbf{Abstract}

Sensible heat flux is estimated based on the eddy-covariance technique
using high frequency sonic anemometer to measurements of vertical wind
velocity and air temperature to calculate turbulent flux and tower
profile measurements of air temperature to calculate storage flux. This
data product contains the measurement data and associated metadata in
HDF5 format. The key sub-data products include storage flux, turbulent
flux, and net surface-atmosphere exchange (NSAE) which is defined as the
sum of storage flux and turbulent flux, on a 30 min basis. The data are
delivered with the Bundled data products - eddy covariance data product
(DP4.00200.001).

\begin{center}\rule{0.5\linewidth}{\linethickness}\end{center}

\textbf{Usage Notes}

During subsequent nominal operations, we plan to produce and publish the
data products in three phases, to accommodate a variety of use cases:
the initial near-real-time transition, a science reviewed quality
transition, and the epoch yearly transition. The initial near-real-time
transition is scheduled to process daily files at a 5-day delay after
data collection to accommodate a 9-day centered planar-fit window. If
the data has not been received from the field it will attempt to process
daily for 30\,days, and if not all data is available after this window a
force execution is performed populating a HDF5 file with metadata and
filling data with NaN's. The monthly file will be produced after all
daily files are available, no later than 30 days after the last daily
file was initially attempted to be processed. After the initial
transition, the NEON science team has a one month window to manually
flag data that were identified as suspect through field-based problem
tracking and resolution tickets or through additional manual data
quality analysis. Then, the science-reviewed transition will occur, and
the data will be republished to the data portal. The last transition
type is part of the yearly epoch versioning, which provides a fully
quality assured and quality controlled version of the data using the
latest full release of the processing code. This transition is scheduled
to occur 18 months after the initial data collection. \newpage
.

\section{DP4.00007.001 Momentum flux}\label{dp4.00007.001-momentum-flux}

\begin{center}\rule{0.5\linewidth}{\linethickness}\end{center}

\textbf{Subsystem}

Terrestrial Instrument System (TIS)

\begin{center}\rule{0.5\linewidth}{\linethickness}\end{center}

\textbf{Coverage}

Data are collected at all terrestrial sites at the tower top above the
canopy, in order to study the exchange of momentum and development of
turbulence between the atmosphere and the ecosystem of interest.

\begin{center}\rule{0.5\linewidth}{\linethickness}\end{center}

\textbf{Description}

Momentum flux is estimated based on the eddy covariance technique using
a sonic anemometer to measure vertical and horizontal winds. This data
product is bundled into DP4.00200, Bundled data products - eddy
covariance, and is not available as a stand-alone download.

\begin{center}\rule{0.5\linewidth}{\linethickness}\end{center}

\textbf{Abstract}

Momentum flux is estimated based on the eddy-covariance technique using
a sonic anemometer to measure vertical and horizontal wind velocities.
This data product contains the measurement data and associated metadata
in HDF5 format. The key sub-data product include only the turbulent flux
on a 30 min basis. The data are delivered with the Bundled data products
- eddy covariance data product (DP4.00200.001).

\begin{center}\rule{0.5\linewidth}{\linethickness}\end{center}

\textbf{Usage Notes}

During subsequent nominal operations, we plan to produce and publish the
data products in three phases, to accommodate a variety of use cases:
the initial near-real-time transition, a science reviewed quality
transition, and the epoch yearly transition. The initial near-real-time
transition is scheduled to process daily files at a 5-day delay after
data collection to accommodate a 9-day centered planar-fit window. If
the data has not been received from the field it will attempt to process
daily for 30\,days, and if not all data is available after this window a
force execution is performed populating a HDF5 file with metadata and
filling data with NaN's. The monthly file will be produced after all
daily files are available, no later than 30 days after the last daily
file was initially attempted to be processed. After the initial
transition, the NEON science team has a one month window to manually
flag data that were identified as suspect through field-based problem
tracking and resolution tickets or through additional manual data
quality analysis. Then, the science-reviewed transition will occur, and
the data will be republished to the data portal. The last transition
type is part of the yearly epoch versioning, which provides a fully
quality assured and quality controlled version of the data using the
latest full release of the processing code. This transition is scheduled
to occur 18 months after the initial data collection. \newpage
.

\section{DP4.00067.001 Carbon dioxide
flux}\label{dp4.00067.001-carbon-dioxide-flux}

\begin{center}\rule{0.5\linewidth}{\linethickness}\end{center}

\textbf{Subsystem}

Terrestrial Instrument System (TIS)

\begin{center}\rule{0.5\linewidth}{\linethickness}\end{center}

\textbf{Coverage}

Data are collected at all terrestrial sites, along the tower profile
from the ground to the tower top above the canopy, in order to study the
ecosystem exchange of scalars (CO2, H2O, etc.) and energy between the
atmosphere and the ecosystem of interest.

\begin{center}\rule{0.5\linewidth}{\linethickness}\end{center}

\textbf{Description}

Carbon dioxide flux of CO2 is estimated based on the eddy covariance
technique from sonic anemometer measurements of vertical winds and an
IRGA measurement of CO2 concentration and tower profile measurements of
CO2 concentration. This data product is bundled into DP4.00200, Bundled
data products - eddy covariance, and is not available as a stand-alone
download.

\begin{center}\rule{0.5\linewidth}{\linethickness}\end{center}

\textbf{Abstract}

Carbon dioxide flux is estimated based on the eddy-covariance technique
using high frequency sonic anemometer measurements of vertical winds
velocity and a infrared gas analyzer (IRGA) measurements of CO2
concentration to calculate turbulent flux, and tower profile
measurements of CO2 concentration to calculate storage flux. This data
product contains the measurement data and associated metadata in HDF5
format. The key sub-data products include storage flux, turbulent flux,
and net surface-atmosphere exchange (NSAE) which is defined as the sum
of storage flux and turbulent flux, on a 30 min basis. The data are
delivered with the Bundled Eddy Covariance (DP4.00200.001) data product.

\begin{center}\rule{0.5\linewidth}{\linethickness}\end{center}

\textbf{Usage Notes}

During subsequent nominal operations, we plan to produce and publish the
data products in three phases, to accommodate a variety of use cases:
the initial near-real-time transition, a science reviewed quality
transition, and the epoch yearly transition. The initial near-real-time
transition is scheduled to process daily files at a 5-day delay after
data collection to accommodate a 9-day centered planar-fit window. If
the data has not been received from the field it will attempt to process
daily for 30\,days, and if not all data is available after this window a
force execution is performed populating a HDF5 file with metadata and
filling data with NaN's. The monthly file will be produced after all
daily files are available, no later than 30 days after the last daily
file was initially attempted to be processed. After the initial
transition, the NEON science team has a one month window to manually
flag data that were identified as suspect through field-based problem
tracking and resolution tickets or through additional manual data
quality analysis. Then, the science-reviewed transition will occur, and
the data will be republished to the data portal. The last transition
type is part of the yearly epoch versioning, which provides a fully
quality assured and quality controlled version of the data using the
latest full release of the processing code. This transition is scheduled
to occur 18 months after the initial data collection. \newpage
.

\section{DP4.00130.001 Stream
discharge}\label{dp4.00130.001-stream-discharge}

\begin{center}\rule{0.5\linewidth}{\linethickness}\end{center}

\textbf{Subsystem}

Aquatic Instrument System (AIS)

\begin{center}\rule{0.5\linewidth}{\linethickness}\end{center}

\textbf{Coverage}

This data product is measured at NEON aquatic wadeable stream and river
sites.

\begin{center}\rule{0.5\linewidth}{\linethickness}\end{center}

\textbf{Description}

Continuous measurements of stream discharge calculated from a
stage-discharge rating curve and sensor-based measurements of water
surface elevation.

\begin{center}\rule{0.5\linewidth}{\linethickness}\end{center}

\textbf{Abstract}

This data product describes the volume of water flowing through a stream
or river cross-section during a given period of time. For each NEON
stream or river site, site-specific stage-discharge rating curve
equations are derived from point observations of gauge height and
discharge. Continuous sensor measurements of surface water pressure are
used to derive water column height. The rating curve equations are
applied to water column height to derive continuous stream discharge.

\begin{center}\rule{0.5\linewidth}{\linethickness}\end{center}

\textbf{Usage Notes}

Queries for this product will return data for all dates within the
specified date range. Continuous stream discharge data are provided at
one-minute intervals. Preliminary data for continuous discharge are
calculated using the rating curve from the preceding water year (October
1 - September 31), while the annual versions are re-calculated following
the end of the water year in which the pressure data was recorded, using
the most recent rating curve. The data package includes unique
identifiers for the curve fit to stage-discharge rating curves,
uncertainty values at the 95\% confidence intervals, and data quality
flags associated with individual discharge values. \newpage
.

\section{DP4.00131.001 Stream morphology
map}\label{dp4.00131.001-stream-morphology-map}

\begin{center}\rule{0.5\linewidth}{\linethickness}\end{center}

\textbf{Subsystem}

Aquatic Observation System (AOS)

\begin{center}\rule{0.5\linewidth}{\linethickness}\end{center}

\textbf{Coverage}

Geomorphology surveys are conducted at NEON aquatic wadeable stream
sites.

\begin{center}\rule{0.5\linewidth}{\linethickness}\end{center}

\textbf{Description}

Map showing the morphology of streams. These maps denote topography of
the stream basin as well as location of the thalweg, coarse woody
debris, gravel/sand bars, and other features of interest.

\begin{center}\rule{0.5\linewidth}{\linethickness}\end{center}

\textbf{Abstract}

The wadeable stream morphology data product provides raw survey data,
maps, shapefiles, and metric tables that quantify stream channel
geomorphology and bed composition and delineates biological habitats
within the aquatic reach boundaries (approximately 1,000 meters in
stream length) of wadeable streams at NEON aquatic sites. Raw survey
data is collected with high-resolution total station survey equipment at
each NEON wadeable stream site. Survey maps and channel metrics are
produced and calculated using raw survey data (Level 0) that are
geo-referenced to a global coordinate system (Level 4). Geomorphology
surveys are conducted at each site once every five years or immediately
following a storm event deemed to have significantly altered stream
morphology within the aquatic reach. Geomorphology surveys conducted
immediately after a stochastic event will assess event magnitude by
quantifying changes in channel geometry, bed composition, and biological
habitat. For further details see
\href{http://data.neonscience.org/api/v0/documents/NEON.DOC.003162vB}{NEON.DOC.003162vB}
AOS Protocol and Procedure: Wadeable Stream Morphology.

\begin{center}\rule{0.5\linewidth}{\linethickness}\end{center}

\textbf{Usage Notes}

Queries for this data product will return all survey data collected from
the site and date range specified. Data provided in each package file
are specific to the same invididual survey. File availability may vary
by site and specific survey. For example, if sensors were not present at
the site during the survey the sensor.shp, S1S2habitatIDS.csv, and
S1S2Length.csv files will not be available as that data was not captured
during the survey. Each data file corresponds with specific
characteristics of the aquatic reach and many are inter-related. For
example, the thalwegLongProfile.csv file details thalweg elevation
throughout the reach and the thalwegByHabitatID.csv file details how
different habitat unit types comprise the long profile. The number of
records in each file will vary by the complexity of the site being
surveyed. The geo\_resultsFile will have one record per site per year of
collection, and will include URLs to download a .zip folder containing
maps and calculated statistics and a separate .zip file to download raw
survey data. Clicking on a URL will start a direct download of this .zip
file from a cloud storage loaction. Using ArcGIS software, data
contained in the .csv files can be spatially represented by associated
shapefiles. Data packaging naming convention will reflect the domain,
site, and end survey date (the last day the survey took place), the data
product, and whether the package contains L4 or L0 data. Example:
``NEON\_D04\_GUIL\_20170825\_MORPHOLOGY\_L4.zip''. \newpage
.

\section{DP4.00132.001 Bathymetric and morphological
maps}\label{dp4.00132.001-bathymetric-and-morphological-maps}

\begin{center}\rule{0.5\linewidth}{\linethickness}\end{center}

\textbf{Subsystem}

Aquatic Observation System (AOS)

\begin{center}\rule{0.5\linewidth}{\linethickness}\end{center}

\textbf{Coverage}

Measured at all NEON lakes and non-wadeable streams (rivers).

\begin{center}\rule{0.5\linewidth}{\linethickness}\end{center}

\textbf{Description}

Bathymetry of lake bottoms and non-wadeable streams for detecting
environmental change as well as for determining lake morphology,
estimating primary productivity, habitat features, and water quality.

\begin{center}\rule{0.5\linewidth}{\linethickness}\end{center}

\textbf{Abstract}

Bathymetric maps are obtained using hydroacoustic (sonar)
instrumentation and interfaced with differential global positioning
system (DGPS) mounted on a boat. Hydroacoustics are utilized to detect
the depth of a water body, sediment characteristics as well as the
presence or absence, approximate abundance, distribution, size, and
behavior of underwater biota. For additional details, see
{[}NEON.DOC.001197: AOS Protocol and Procedure: Bathymetry and
Morphology of Lakes and Non-Wadeable
Streams{]}((\url{http://data.neonscience.org/api/v0/documents/NEON.DOC.001197vE}).

\begin{center}\rule{0.5\linewidth}{\linethickness}\end{center}

\textbf{Usage Notes}

Queries for this data product will return data from bat\_fieldData,
bat\_pointCollection, bat\_sonarRecord and bat\_resultsFile for any lake
or non-wadeable (river) site within the user specified range. The
bat\_fieldData table will have one record per site per year of
collection, with up to 5 years between collections. The
bat\_pointcollection table may have several records per site per year of
collection, depending on how many waypoints were collected (could be
zero). The bat\_sonarRecord table will have several records per site per
year of collection, one per recording number along the survey track. The
bat\_resultsFile will have one record per site per year of collection,
and will include URLs to download a .zip folder containing maps and
calculated statistics. Clicking on the URL will start a direct download
of this .zip file from a cloud storage location. Users should check data
carefully for anomalies before joining tables. \newpage
.

\section{DP4.00133.001 Stream discharge rating
curve}\label{dp4.00133.001-stream-discharge-rating-curve}

\begin{center}\rule{0.5\linewidth}{\linethickness}\end{center}

\textbf{Subsystem}

Aquatic Instrument System (AIS)

\begin{center}\rule{0.5\linewidth}{\linethickness}\end{center}

\textbf{Coverage}

This data product is measured at NEON aquatic wadeable stream and river
sites.

\begin{center}\rule{0.5\linewidth}{\linethickness}\end{center}

\textbf{Description}

Rating curve generated from manual wading surveys of stream discharge.
Used to calculate continuous measurements of stream discharge.

\begin{center}\rule{0.5\linewidth}{\linethickness}\end{center}

\textbf{Abstract}

This data product provides parameters that describe the relationship
between staff gauge readings and stream discharge measurements. The
parameters provided are the coefficients defining an exponential curve,
and are derived from manually measured discharge and staff gauge
readings by a Bayesian model. Rating curve parameters published in this
product are used together with sensor measurements of surface water
pressure to calculate the continuous stream discharge data product
(DP4.00130). Data users should refer to the user guide for stream
discharge rating curve (NEON\_ratingCurve\_userGuide\_vA) for more
detailed information on the algorithm used to develop a rating curve.

\begin{center}\rule{0.5\linewidth}{\linethickness}\end{center}

\textbf{Usage Notes}

Queries for this data product that include September will return rating
curve data for the site(s). Queries that do not include September will
not return data since the rating curve is made on an annual basis
following the end of the water year (Oct 1st - Sept 30th). The number of
records in sdrc\_gaugeDischargeMeas will, for each site, match the
number for the past water year in the dsc\_fieldData table of Stream
discharge field collection (NEON.DP1.20048). Values may differ between
the two tables. Discharge values are recalculated from individual point
measurements and stage values are offset using information from the NEON
geolocation database. One record is created for each hydraulic control
at a site for the sdrc\_posteriorParameters table. One record is created
for each rating curve (usually one per year) in the
sdrc\_stageDischargeCurveInfo table. 500 records per hydraulic control
per site are created for the sdrc\_sampledParameters, which represent
the Markov Chain Monte Carlo (MCMC) samples of the range of model
parameters. One record per staff gauge and discharge measurement set
used to develop the rating curve is created in the
sdrc\_resultsResiduals table, this is often more than the number of
records in the sdrc\_gaugeDischargeMeas due to the inclusion of
measurements from the previous water year in the development of the
curve. \newpage
.

\section{DP4.00137.001 Latent heat
flux}\label{dp4.00137.001-latent-heat-flux}

\begin{center}\rule{0.5\linewidth}{\linethickness}\end{center}

\textbf{Subsystem}

Terrestrial Instrument System (TIS)

\begin{center}\rule{0.5\linewidth}{\linethickness}\end{center}

\textbf{Coverage}

Data are collected at all terrestrial sites, along the tower profile
from the ground to the tower top above the canopy, in order to study the
ecosystem exchange of scalars (CO2, H2O, etc.) and energy between the
atmosphere and the ecosystem of interest.

\begin{center}\rule{0.5\linewidth}{\linethickness}\end{center}

\textbf{Description}

Latent heat flux is estimated based on the eddy covariance technique
using a sonic anemometer to measure vertical winds and an IRGA sensor to
measure water vapor and tower profile measurements of water vapor. This
data product is bundled into DP4.00200, Bundled data products - eddy
covariance, and is not available as a stand-alone download.

\begin{center}\rule{0.5\linewidth}{\linethickness}\end{center}

\textbf{Abstract}

Latent heat flux is estimated based on the eddy-covariance technique
using high frequency sonic anemometer measurements of vertical wind
velocity and a infrared gas analyzer (IRGA) measurements of water vapor
to calculate turbulent flux, and tower profile measurements of water
vapor to calculate storage flux. This data product contains the
measurement data and associated metadata in HDF5 format. The key
sub-data products include storage flux, turbulent flux, and net
surface-atmosphere exchange (NSAE) which is defined as the sum of
storage flux and turbulent flux, on a 30 min basis. The data are
delivered with the Bundled data products - eddy covariance data product
(DP4.00200.001).

\begin{center}\rule{0.5\linewidth}{\linethickness}\end{center}

\textbf{Usage Notes}

During subsequent nominal operations, we plan to produce and publish the
data products in three phases, to accommodate a variety of use cases:
the initial near-real-time transition, a science reviewed quality
transition, and the epoch yearly transition. The initial near-real-time
transition is scheduled to process daily files at a 5-day delay after
data collection to accommodate a 9-day centered planar-fit window. If
the data has not been received from the field it will attempt to process
daily for 30\,days, and if not all data is available after this window a
force execution is performed populating a HDF5 file with metadata and
filling data with NaN's. The monthly file will be produced after all
daily files are available, no later than 30 days after the last daily
file was initially attempted to be processed. After the initial
transition, the NEON science team has a one month window to manually
flag data that were identified as suspect through field-based problem
tracking and resolution tickets or through additional manual data
quality analysis. Then, the science-reviewed transition will occur, and
the data will be republished to the data portal. The last transition
type is part of the yearly epoch versioning, which provides a fully
quality assured and quality controlled version of the data using the
latest full release of the processing code. This transition is scheduled
to occur 18 months after the initial data collection. \newpage
.

\section{DP4.00200.001 Bundled data products - eddy
covariance}\label{dp4.00200.001-bundled-data-products---eddy-covariance}

\begin{center}\rule{0.5\linewidth}{\linethickness}\end{center}

\textbf{Subsystem}

Terrestrial Instrument System (TIS)

\begin{center}\rule{0.5\linewidth}{\linethickness}\end{center}

\textbf{Coverage}

Data are collected at all terrestrial sites, along the vertical tower
profile from the ground to the tower top above the canopy. These data
are used to determine the net ecosystem exchange of heat and gases (CO2,
H2O, etc.) between the atmosphere and the ecosystem of interest.

\begin{center}\rule{0.5\linewidth}{\linethickness}\end{center}

\textbf{Description}

Bundle of eddy-covariance data products, including related
meteorological and soil data products.

\begin{center}\rule{0.5\linewidth}{\linethickness}\end{center}

\textbf{Abstract}

Net surface-atmosphere exchange, or ``flux'' quantifies how much heat,
H2O and CO2 are transferred between an ecosystem and the atmosphere.
Fluxes are useful in a variety scientific applications, including to
study ecosystem processes, to interpret and calibrate satellite
observations of the earth system, and to constrain ecosystem and earth
system models. One of the most direct approaches to observe the net
surface-atmosphere exchange is the in-situ
\href{https://youtu.be/CR4Anc8Mkas}{eddy-covariance method}. Calculation
of the net surface-atmosphere exchange involves the estimation of at
least two major terms (assuming horizontally homogenous surface
conditions): the turbulent flux and the storage flux. In addition,
stable isotope measurements of CO2 and H2O within and above the
ecosystem canopy can support the subsequent partitioning of the net
surface-atmosphere exchange into ecosystem constituent fluxes. For
example, partitioning CO2 into photosynthesis and respiration, or
evaporation and transpiration in the case of H2O.

For data product and algorithm details please see
\href{http://data.neonscience.org/documents}{NEON.DOC.004571}; in short:
this data product bundle contains derived eddy-covariance data products
and associated metadata in HDF5 format. Each file contains metadata
about the file structure, table formats, and attributes. For more
information on using HDF5 files, please visit The HDF Group website at
\url{https://www.hdfgroup.org/}. This is a provisional product and query
reproducibility cannot be guaranteed. During nominal Operations,
earliest anticipated availability of the provisional product is 5 days
after data acquisition, with planned annual re-processing and
publication of consistent, versioned datasets. Data, quality flags and
metrics (qfqm), and uncertainty metrics (ucrt) are currently provided in
folders using the following naming convention within the HDF5 file
structure:
data\_product\_level/type\_of\_data\_available/data\_product\_abbreviation
(e.g., ``dp01/data/soni''). Empty folders within the file structure are
being incrementally filled in future publications. The data products
embedded in this bundle currently include the following:

Data Product \textbar{} Type of data available \textbar{} Abbreviation
\textbar{} Temporal Resolution

DP1.00002 Single aspirated air temperature \textbar{} data, qfqm, ucrt
\textbar{} tempAirLvl \textbar{} 1-min, 30-min

DP1.00003 Triple aspirated air temperature \textbar{} data, qfqm, ucrt
\textbar{} tempAirTop \textbar{} 1-min, 30-min

DP1.00007 3D wind speed, direction and sonic temperature \textbar{}
data, qfqm, ucrt \textbar{} soni \textbar{} 1-min, 30-min

DP1.00010 3D wind attitude and motion reference \textbar{} data, qfqm,
ucrt \textbar{} amrs \textbar{} 1-min, 30-min

DP1.00034 CO2 concentration - turbulent \textbar{} data, qfqm, ucrt
\textbar{} co2Turb \textbar{} 1-min, 30-min

DP1.00035 H2O concentration - turbulent \textbar{} data, qfqm, ucrt
\textbar{} h2oTurb \textbar{} 1-min, 30-min

DP1.00036 Atmospheric CO2 isotopes \textbar{} data, qfqm, ucrt
\textbar{} isoCo2 \textbar{} 9-min, 30-min

DP1.00037 Atmospheric H2O isotopes \textbar{} data, qfqm, ucrt
\textbar{} isoH2o \textbar{} 9-min, 30-min

DP1.00099 CO2 concentration - storage \textbar{} data, qfqm, ucrt
\textbar{} co2Stor \textbar{} 2-min, 30-min

DP1.00100 H2O concentration - storage \textbar{} data, qfqm, ucrt
\textbar{} h2oStor \textbar{} 2-min, 30-min

DP2.00008 CO2 concentration rate of change \textbar{} data, qfqm
\textbar{} co2Stor \textbar{} 30-min

DP2.00009 H2O concentration rate of change \textbar{} data, qfqm
\textbar{} h2oStor \textbar{} 30-min

DP2.00024 Temperature rate of change \textbar{} data, qfqm \textbar{}
tempStor \textbar{} 30-min

DP3.00008 Temperature rate of change profile \textbar{} data, qfqm
\textbar{} tempStor \textbar{} 30-min

DP3.00009 CO2 concentration rate of change profile \textbar{} data, qfqm
\textbar{} co2Stor \textbar{} 30-min

DP3.00010 H2O concentration rate of change profile data \textbar{} qfqm
\textbar{} h2oStor \textbar{} 30-min

DP4.00002 Sensible heat flux \textbar{} data, qfqm \textbar{} fluxTemp
\textbar{} 30-min

DP4.00007 Momentum flux \textbar{} data, qfqm \textbar{} fluxMome
\textbar{} 30-min

DP4.00067 Carbon dioxide flux \textbar{} data, qfqm \textbar{} fluxCo2
\textbar{} 30-min

DP4.00137 Latent heat flux \textbar{} data, qfqm \textbar{} fluxH2o
\textbar{} 30-min

DP4.00201 Flux footprint characteristics \textbar{} data, qfqm
\textbar{} foot \textbar{} 30-min

\begin{center}\rule{0.5\linewidth}{\linethickness}\end{center}

\textbf{Usage Notes}

During nominal operations, we plan to produce and publish the data
products in three phases, to accommodate a variety of use cases: the
initial near-real-time transition, a science reviewed quality
transition, and the epoch yearly transition. The initial near-real-time
transition is scheduled to process daily expanded files at a 5-day delay
after data collection to accommodate a 9-day centered planar-fit window.
If the data has not been received from the field it will attempt to
process daily for 30\,days, and if not all data is available after this
window a force execution is performed populating a HDF5 file with
metadata and filling data with NaN's. The monthly basic file will be
produced after all daily files are available, as early as 5 days after
the end of the month, and no later than 30 days after the last daily
file was initially attempted to be processed. Daily basic files are not
produced. After the initial transition, the NEON science team has a one
month window to manually flag data that were identified as suspect
through field-based problem tracking and resolution tickets or through
additional manual data quality analysis. Then, the science-reviewed
transition will occur, and the data will be republished to the data
portal. The last transition type is part of the yearly epoch versioning,
which provides a fully quality assured and quality controlled version of
the data using the latest full release of the processing code. This
transition is scheduled to occur 18 months after the initial data
collection. For additional details please see
\href{http://data.neonscience.org/documents}{NEON.DOC.004571}. \newpage
.

\section{DP4.00201.001 Flux footprint
characteristics}\label{dp4.00201.001-flux-footprint-characteristics}

\begin{center}\rule{0.5\linewidth}{\linethickness}\end{center}

\textbf{Subsystem}

Terrestrial Instrument System (TIS)

\begin{center}\rule{0.5\linewidth}{\linethickness}\end{center}

\textbf{Coverage}

Data are collected at all terrestrial sites, along the tower profile
from the ground to the tower top above the canopy, in order to study the
ecosystem exchange of scalars (CO2, H2O, etc.) and energy between the
atmosphere and the ecosystem of interest.

\begin{center}\rule{0.5\linewidth}{\linethickness}\end{center}

\textbf{Description}

The eddy-covariance flux measurement sources its information from an
upstream surface, the footprint. Footprint characteristics provide the
biophysical surface information of this time-varying area, necessary to
distinguish temporal effects (e.g., biological activity) from spatial
effects (e.g., changing wind direction). This data product is bundled
into DP4.00200, Bundled data products - eddy covariance, and is not
available as a stand-alone download.

\begin{center}\rule{0.5\linewidth}{\linethickness}\end{center}

\textbf{Abstract}

A footprint model as described by Metzger et al. (2012) is used to
determine where on the ground surface emissions measured by the
eddy-covariance turbulent exchange system originated from. This allows
interpretation of observed emission rates against hour-to-hour
variations in flux footprint over surface properties such as land cover,
soil moisture etc. e.g.~from gridded remote-sensing data products. This
data product contains the quality-controlled measurement data and
associated metadata in HDF5 format. The key sub-data products include
model inputs, footprint statistics, and footprint weight matrices, on a
30 min basis. The data are delivered with the Bundled data products -
eddy covariance data product (DP4.00200.001).

\begin{center}\rule{0.5\linewidth}{\linethickness}\end{center}

\textbf{Usage Notes}

During subsequent nominal operations, we plan to produce and publish the
data products in three phases, to accommodate a variety of use cases:
the initial near-real-time transition, a science reviewed quality
transition, and the epoch yearly transition. The initial near-real-time
transition is scheduled to process daily files at a 5-day delay after
data collection to accommodate a 9-day centered planar-fit window. If
the data has not been received from the field it will attempt to process
daily for 30\,days, and if not all data is available after this window a
force execution is performed populating a HDF5 file with metadata and
filling data with NaN's. The monthly file will be produced after all
daily files are available, no later than 30 days after the last daily
file was initially attempted to be processed. After the initial
transition, the NEON science team has a one month window to manually
flag data that were identified as suspect through field-based problem
tracking and resolution tickets or through additional manual data
quality analysis. Then, the science-reviewed transition will occur, and
the data will be republished to the data portal. The last transition
type is part of the yearly epoch versioning, which provides a fully
quality assured and quality controlled version of the data using the
latest full release of the processing code. This transition is scheduled
to occur 18 months after the initial data collection. \newpage
.

\section{DP4.50036.001 Soil CO2 flux - MDP
sensor}\label{dp4.50036.001-soil-co2-flux---mdp-sensor}

\begin{center}\rule{0.5\linewidth}{\linethickness}\end{center}

\textbf{Subsystem}

Terrestrial Instrument System (TIS)

\begin{center}\rule{0.5\linewidth}{\linethickness}\end{center}

\textbf{Coverage}

NA

\begin{center}\rule{0.5\linewidth}{\linethickness}\end{center}

\textbf{Description}

Flux of carbon dioxide from the soil surface. Generated only by the
mobile deployment platforms (MDP); for soil CO2 concentration at NEON
towers see NEON.DOM.SITE.DP1.00095.

\begin{center}\rule{0.5\linewidth}{\linethickness}\end{center}

\textbf{Abstract}

NA


\end{document}
